\documentclass{beamer}
\usepackage{etex} % fixes new-dimension error
\usepackage{lmodern}
\usepackage[T1]{fontenc}
\usepackage{mathtools}
%%%%%%%%%%%%% Macros
%%%% Categories
\newcommand{\catfont}[1]{\mathsf{#1}}
\newcommand{\cop}{\catfont{op}}
\newcommand{\Law}{\catfont{Law}}
\newcommand{\catV}{\catfont{V}}
\newcommand{\catX}{\catfont{X}}
\newcommand{\catC}{\catfont{C}}
\newcommand{\catD}{\catfont{D}}
\newcommand{\catA}{\catfont{A}}
\newcommand{\catB}{\catfont{B}}
\newcommand{\catI}{\catfont{I}}
\newcommand{\Set}{\catfont{Set}}
\newcommand{\Top}{\catfont{Top}}
\newcommand{\Pos}{\catfont{Pos}}
\newcommand{\Inj}{\catfont{Inj}}
\newcommand{\Det}{\catfont{RMhat}}
\newcommand{\CoAlg}[1]{\catfont{CoAlg}\left (#1 \right )}
\newcommand{\Mon}{\catfont{Mon}}
\newcommand{\Mnd}{\catfont{Mnd}(\catC)}
\newcommand{\SMnd}{\catfont{Mnd}(\Set)}
\newcommand{\CLat}{\catfont{CLat}}
\newcommand{\Stone}{\catfont{Stone}}
\newcommand{\Spectral}{\catfont{Spectral}}
\newcommand{\CompHaus}{\catfont{CompHaus}}
\newcommand{\Subs}[2]{\catfont{Sub}_{}}
\newcommand{\Cone}{\catfont{Cone}}
\newcommand{\StComp}{\catfont{StablyComp}}
\newcommand{\PosC}{\catfont{PosComp}}
\newcommand{\Haus}{\catfont{Haus}}
\newcommand{\Meas}{\catfont{Meas}}
\newcommand{\Ord}{\catfont{Ord}}
\newcommand{\EndoC}{[\catC,\catC]}
%% General functors
\newcommand{\funfont}[1]{#1}
\newcommand{\funF}{\funfont{F}}
\newcommand{\funU}{\funfont{U}}
\newcommand{\funG}{\funfont{G}}
\newcommand{\funT}{\funfont{T}}
\newcommand{\funI}{\funfont{I}}
%% Particular kinds of functors
\newcommand{\sfunfont}[1]{\mathrm{#1}}
\newcommand{\Pow}{\sfunfont{P}}
\newcommand{\Dist}{\sfunfont{D}}
\newcommand{\Maybe}{\sfunfont{M}}
\newcommand{\List}{\sfunfont{L}}
\newcommand{\UForg}{\sfunfont{U}}
\newcommand{\Forg}[1]{\sfunfont{U}_{#1}}
\newcommand{\Id}{\sfunfont{Id}}
\newcommand{\Vie}{\sfunfont{V}}
\newcommand{\Disc}{\funfont{D}}
\newcommand{\Weight}{\sfunfont{W}}
\newcommand{\homf}{\sfunfont{hom}}
\newcommand{\Yoneda}{\sfunfont{Y}}
%% Diagram functors
\newcommand{\Diag}{\mathscr{D}}
\newcommand{\KDiag}{\mathscr{K}}
\newcommand{\LDiag}{\mathscr{L}}
%% Monads
\newcommand{\monadfont}[1]{\mathbb{#1}}
\newcommand{\monadT}{\monadfont{T}}
\newcommand{\monadS}{\monadfont{S}}
\newcommand{\monadU}{\monadfont{U}}
\newcommand{\monadH}{\monadfont{H}}
\newcommand{\str}{\mathrm{str}}
%% Adjunctions
\newcommand\adjunct[2]{\xymatrix@=8ex{\ar@{}[r]|{\top}\ar@<1mm>@/^2mm/[r]^{{#2}}
& \ar@<1mm>@/^2mm/[l]^{{#1}}}}
\newcommand\adjunctop[2]{\xymatrix@=8ex{\ar@{}[r]|{\bot}\ar@<1mm>@/^2mm/[r]^{{#2}}
& \ar@<1mm>@/^2mm/[l]^{{#1}}}}
%% Retractions
\newcommand\retract[2]{\xymatrix@=8ex{\ar@{}[r]|{}\ar@<1mm>@/^2mm/@{^{(}->}[r]^{{#2}}
& \ar@<1mm>@/^2mm/@{->>}[l]^{{#1}}}}
%% Limits
\newcommand{\pv}[2]{\langle #1, #2 \rangle}
\newcommand{\limt}{\mathrm{lim}}
\newcommand{\pullbackcorner}[1][dr]{\save*!/#1+1.2pc/#1:(1,-1)@^{|-}\restore}
\newcommand{\pushoutcorner}[1][dr]{\save*!/#1-1.2pc/#1:(-1,1)@^{|-}\restore}
%% Colimits
\newcommand{\colim}{\mathrm{colim}}
\newcommand{\inl}{\mathrm{inl}}
\newcommand{\inr}{\mathrm{inr}}
%% Distributive categories
\newcommand{\distr}{\mathrm{dist}}
\newcommand{\undistr}{\mathrm{undist}}
%% Closedness
\newcommand{\curry}[1]{\mathrm{curry}{#1}}
\newcommand{\app}{\mathrm{app}}
%% Misc. operations
\newcommand{\const}[1]{\underline{#1}}
\newcommand{\comp}{\cdot}
\newcommand{\id}{\mathrm{id}}
%% Factorisations
\newcommand{\EClass}{E}
\newcommand{\MClass}{M}
\newcommand{\MConeClass}{\mathcal{M}}
%%%%%%%%%%%%%%%% End of Categorical Stuff

%%%% Misc
%% Operations
\newcommand{\blank}{\, - \,}
\newcommand{\sem}[1]{\llbracket #1 \rrbracket}
\newcommand{\closure}[1]{\overline{#1}}
\DeclareMathOperator{\img}{\mathrm{im}}
\DeclareMathOperator{\dom}{\mathrm{dom}}
\DeclareMathOperator{\codom}{\mathrm{codom}}
%% Sets of numbers
\newcommand{\Nats}{\mathbb{N}}
\newcommand{\Reals}{\mathbb{R}}
\newcommand{\Rz}{\Reals_{\geq 0}}
%% Writing
\newcommand{\cf}{\emph{cf.}}
\newcommand{\ie}{\emph{i.e.}}
\newcommand{\eg}{\emph{e.g.}}
\newcommand{\df}[1]{\emph{\textbf{#1}}}
%%%%%%%%%%%%%%%% End of Misc

%%%% Programming Stuff
%% Types
\newcommand{\typefont}[1]{\mathbb{#1}}
\newcommand{\typeOne}{1}
\newcommand{\typeTwo}{2}
\newcommand{\typeA}{\typefont{A}}
\newcommand{\typeB}{\typefont{B}}
\newcommand{\typeC}{\typefont{C}}
\newcommand{\typeV}{\typefont{V}}
\newcommand{\typeD}{\typefont{D}}
%% RuleName
\newcommand{\rulename}[1]{(\mathrm{#1})}
%% Sequents
\newcommand{\jud}{\vdash}
\newcommand{\vljud}{\vdash}
\newcommand{\cojud}{\vdash_{\co}}
\newcommand{\vl}{\mathtt{v}}
\newcommand{\co}{\mathtt{c}}
% Program font
\newcommand{\prog}[1]{\mathtt{#1}}
\newcommand{\pseq}[3]{#1 \leftarrow #2; #3}
\newcommand{\ppm}[4]{(#1,#2) \leftarrow #3; #4}
\newcommand{\pinl}[1]{\prog{inl}(#1)}
\newcommand{\pinr}[1]{\prog{inr}(#1)}
\newcommand{\pcase}[4]{\prog{ case } #1 \prog{ of } \pinl{#2} \Rightarrow #3 ; \pinr{#2} \Rightarrow #4}
%% Sets of terms
\newcommand{\ValuesBP}[2]{\mathsf{Values}(#1, #2)}
\newcommand{\TermsBP}[2]{\mathsf{Terms}(#1, #2)}
\newcommand{\closValP}[1]{\ValuesBP{\emptyset}{#1}}
\newcommand{\closTermP}[1]{\TermsBP{\emptyset}{#1}}
\newcommand{\closVal}{\closValP{\typeA}}
\newcommand{\closTerm}{\closTermP{\typeA}}
%% Contextual equivalence
\newcommand{\ctxeq}{\equiv_{\prog{ctx}}}
%%%% End of Programming Stuff

%-------------- template --------------------------------------------------
\usetheme{metropolis}
\metroset{block=fill}
%\usetheme{Boadilla}
% Configuring the foot line
\setbeamertemplate{footline}
{
  \leavevmode%
  \hbox{%
  \begin{beamercolorbox}[wd=.4\paperwidth,ht=2.25ex,dp=1ex,center]{author in head/foot}%
    \usebeamerfont{author in head/foot}\insertshortauthor
  \end{beamercolorbox}%
  \begin{beamercolorbox}[wd=.5\paperwidth,ht=2.25ex,dp=1ex,center]{title in head/foot}%
    \usebeamerfont{title in head/foot}\insertsection
  \end{beamercolorbox}%
  \begin{beamercolorbox}[wd=.1\paperwidth,ht=2.25ex,dp=1ex,right]{date in head/foot}%
    \insertframenumber{} / \inserttotalframenumber\hspace*{2ex} 
  \end{beamercolorbox}}%
  \vskip0pt%
}
% No configuration symbols
\setbeamertemplate{navigation symbols}{}
%--------------- packages -------------------------------------------------
\usepackage{graphicx,amsmath}
\usepackage{stmaryrd} % cf. interleave
\usepackage{booktabs}
\usepackage{amscd}
\usepackage{multicol}
\usepackage[absolute,overlay]{textpos}
\usepackage{alltt}
\usepackage{proof}
\usepackage{subcaption}
\usepackage{tikz}
\usepackage{tikz-cd}
\usepackage{quantikz}
\usepackage[new]{old-arrows}
\usepackage[all]{xy}
\usepackage{pgfplots}
\usepackage{textcomp}
\usetikzlibrary{arrows.meta, calc, fit, tikzmark}
\usepackage{pstricks,pst-node,pst-text,pst-3d}
% context
\AtBeginSection[]
{
    \begin{frame}
        \frametitle{Table of Contents}
        \tableofcontents[currentsection]
    \end{frame}
}

\author[Renato Neves]{Renato Neves}

% logos of institutions
\titlegraphic{
  \begin{textblock*}{5cm}(6.7cm,7.45cm)
     \includegraphics[scale=0.06]{logos/uminho.png}\hspace*{.85cm}~%
  \end{textblock*}
  \begin{textblock*}{5cm}(9.4cm,7.42cm)
    \includegraphics[scale=0.50]{logos/haslab.pdf}
  \end{textblock*}
}

% No date
\date{}

\begin{document}

\title{A First View of Quantum Algorithmics}

\frame[plain]{\titlepage}

\section{The Problem}

\begin{frame}{Our First Encounter with `Quantum Advantage'} 

        \begin{block}{The Problem}
        Receive a `single-bit' function $f : \{0,1\} \to \{0,1\}$

        Either $f(0) = f(1)$ or $f(0) \not = f(1)$

        Tell us whether the first or second case hold
        \end{block}

        \pause
        Classically, to determine which case holds requires 
        running $f$ twice 

        \pause
        Quantumly, it suffices to run $f$ once \dots

        \pause
        due to two quantum effects
        \alert{superposition} and 
        \alert{interference}
\end{frame}

\begin{frame}{The Need for a Quantum Computational Language}

        Quantum solution to the previous problem given as
        an \alert{algorithm}
        
        In order to describe it, it is convenient to use a
        computational language \dots
        
        \dots\ just like in the classical case
\end{frame}

\section{Recap}

\begin{frame}{Postulates of Pure Quantum Systems}

        \begin{block}{States}
                State of $n$-qubits encoded as a \alert{unit} vector
                \[
                        v \in \underbrace{\Complex^2 \otimes \dots \otimes
                        \Complex^2}_{\text{$n$ times}} \cong \Complex^{2^n}
                \]
        \end{block}

        \begin{block}{State operations}
                $n$-qubit operation encoded as an \alert{isometry}
                \[
                        \Complex^{2^n} \longrightarrow \Complex^{2^n}
                \]
                \ie\ a linear map that preserves norms
        \end{block}

        \scriptsize{Recall: we can sequentially compose and tensor
        isometries}

\end{frame}

\section{The Quantum Circuit Formalism}

\begin{frame}{Genesis}
 We start with a set of \alert{quantum operations}

 \begin{center}
        $\left \{ \begin{quantikz}
               &[4mm] \gate{U_1} \qwbundle{n_1} &[3mm] \qw \qwbundle{n_1}
        \end{quantikz}
        , \quad \dots \quad  , 
        \begin{quantikz}
               &[4mm] \gate{U_k} \qwbundle{n_k} &[3mm] \qw \qwbundle{n_k}
        \end{quantikz}
        \right \}$
 \end{center}

 Each operation $U_i$ \alert{manipulates the state} of $n_i$-qubits received
 from its left-hand side \dots

 \dots\  and returns the result on its right-hand side

\end{frame}

\begin{frame}{Examples of Quantum Operations}
        \hspace{0.5cm}
        $\left \sem{ 
                \tikzmark{x1}
                \begin{quantikz}
                &\gate{X} &\qw    
                \end{quantikz}
                \tikzmark{x2} 
        \right } = X : \Complex^2 \to \Complex^2$ (\alert{not} operation)


        \vspace{2cm}  
        \hspace{0.5cm}
        $\left \sem{ 
                \begin{quantikz}
                &\gate{H} &\qw    
                \end{quantikz}
        \right } 
        = H : \Complex^2 \to \Complex^2$ (\alert{Hadamard} operation)

        \begin{tikzpicture}[overlay,remember picture,
        box/.style = {rounded corners},
        pin edge={-Stealth,thick, red}]
        \coordinate (x1) at ($({pic cs:x1})+(+0.5ex, 1.5ex)$);
        \coordinate (x2) at ($({pic cs:x2})+(-0.5ex,-2.8ex)$);
        \node[semitransparent, 
             fit=(x1) (x2),
             pin=below:\tiny{$\sem{x}$ reads as "the mathematical meaning of $x$"}]  {};
        \end{tikzpicture}

\end{frame}

\begin{frame}{New Operations from Old Ones}

        \begin{block}{Sequential Composition}
                      \[
                              \small{
                                \begin{quantikz}[transparent]
                                        &[4mm] \gate{U_1} \qwbundle{n} &[3mm] \qw \qwbundle{n}
                                \end{quantikz} 
                                \begin{quantikz}[transparent]
                                        &[4mm] \gate{U_2} \qwbundle{n} &[3mm] \qw \qwbundle{n}
                                \end{quantikz} 
                                \Longrightarrow
                                \begin{quantikz}[transparent]
                                        &[7mm]  
                                        \gate{U_1} \qwbundle{n}
                                        \gategroup[wires=1,steps=2,style={dashed,
                                        rounded corners,fill=blue!20},background]{}
                                        & \gate{U_2} 
                                        & [7mm] \qw \qwbundle{n}
                                \end{quantikz} 
                        }
                      \]
        \end{block}
        \begin{block}{Parallel Composition}
                      \[
                              \small{
                                \begin{quantikz}[transparent]
                                        &[4mm] \gate{U_1} \qwbundle{n_1} &[3mm] \qw \qwbundle{n_1}
                                \end{quantikz} 
                                \begin{quantikz}[transparent]
                                        &[4mm] \gate{U_2} \qwbundle{n_2} &[3mm] \qw \qwbundle{n_2}
                                \end{quantikz} 
                                \Longrightarrow
                                \begin{quantikz}[transparent]
                                        &[10mm]  
                                        \gate{U_1} \qwbundle{n_1}
                                        \gategroup[wires=2,steps=1,style={dashed,
                                        rounded corners,fill=blue!20},background]{}
                                        & [8mm] \qw \qwbundle{n_1} \\
                                        &[10mm]  
                                        \gate{U_2} \qwbundle{n_2}
                                        & [8mm] \qw \qwbundle{n_2} 
                                \end{quantikz} 
                        }
                      \]
        \end{block}
\end{frame}

\begin{frame}{Mathematical Meaning of Sequential Composition}

        $\left \sem{\begin{quantikz}[transparent]
                &[4mm] \gate{U_1} \qwbundle{n} &[3mm] \qw \qwbundle{n}
        \end{quantikz} \right } 
        = f : \Complex^{2^n} \to \Complex^{2^n}$ and

        $\left \sem{
        \begin{quantikz}[transparent]
                &[4mm] \gate{U_2} \qwbundle{n} &[3mm] \qw \qwbundle{n}
        \end{quantikz} \right }
        = g : \Complex^{2^n} \to \Complex^{2^n}$ entails  \dots

        $\left \sem{
        \begin{quantikz}[transparent]
                &[7mm]  
                \gate{U_1} \qwbundle{n}
                \gategroup[wires=1,steps=2,style={dashed,
                rounded corners,fill=blue!20},background]{}
                & \gate{U_2} 
                & [7mm] \qw \qwbundle{n}
        \end{quantikz}  \right }
        = g \comp f : \Complex^{2^n} \to \Complex^{2^n}$
\end{frame}
\begin{frame}{Mathematical Meaning of Parallel Composition}
        $\left \sem{\begin{quantikz}[transparent]
                &[4mm] \gate{U_1} \qwbundle{n_1} &[3mm] \qw \qwbundle{n_1}
        \end{quantikz} \right } 
        = f : \Complex^{2^{n_1}} \to \Complex^{2^{n_1}}$ and

        $\left \sem{
        \begin{quantikz}[transparent]
                &[4mm] \gate{U_2} \qwbundle{n_2} &[3mm] \qw \qwbundle{n_2}
        \end{quantikz} \right }
        = g : \Complex^{2^{n_2}} \to \Complex^{2^{n_2}}$ entails  \dots

        $\left \sem{
        \begin{quantikz}[transparent]
                                &[10mm]  
                                \gate{U_1} \qwbundle{n_1}
                                \gategroup[wires=2,steps=1,style={dashed,
                                rounded corners,fill=blue!20},background]{}
                                & [8mm] \qw \qwbundle{n_1} \\
                                &[10mm]  
                                \gate{U_1} \qwbundle{n_2}
                                & [8mm] \qw \qwbundle{n_2} 
        \end{quantikz} 
        \right }
        = f \otimes g : \underbrace{\Complex^{2^{n_1}} \otimes  \Complex^{2^{n_2}}}_{\cong\
                \Complex^{2^{n_1 + n_2}}} \to 
                \underbrace{\Complex^{2^{n_1}} \otimes  \Complex^{2^{n_2}}}_{\cong\
                \Complex^{2^{n_1 + n_2}}}$ 
\end{frame}

\begin{frame}{Two Warm-up Exercises}
        \begin{enumerate}
                \item Show that
                        \[
                        \begin{quantikz}
                                & \qw & \qw 
                        \end{quantikz}\
                        \tikzmark{y1} = \tikzmark{y2}
                        \begin{quantikz}
                                & \gate{H} & \gate{H} & \qw 
                        \end{quantikz}
                        \]

                        \vspace{1.2cm}
                \item Prove that the circuit
                        \[
                        \begin{quantikz}
                                & \gate{X} & \gate{H} & \qw \\
                                & \gate{X} & \gate{H} & \qw
                        \end{quantikz}
                        \]
                       can be built in two different ways and that despite that
                       its mathematical meaning is unambiguous
       \end{enumerate}

        \begin{tikzpicture}[overlay,remember picture,
        box/.style = {rounded corners},
        pin edge={-Stealth,thick, red}]
        \coordinate (y1) at ($({pic cs:y1})+(+0.5ex, 1.5ex)$);
        \coordinate (y2) at ($({pic cs:y2})+(-0.5ex,-1.5ex)$);
        \node[semitransparent, 
             fit=(y1) (y2),
             pin=below:\tiny{Our first case of quantum interference}]  {};
        \end{tikzpicture}

\end{frame}

\section{Back to our Problem}

\begin{frame}{Turning $f$ into a Quantum Operation pt. I}
        $f : \{0,1\} \to \{0,1\}$ extends to a linear map $\Complex^2 \to
        \Complex^2$

        \dots\ but not necessarily to an isometry

        \vfill
        \begin{example}
                When $f$ is constant on $0$ we obtain $f \ket{0} = \ket{0}$ and
                $f \ket{1} = \ket{0}$. Then we know that
                $\norm{ \frac{1}{\sqrt{2}} (\ket{0} + \ket{1}) } = 1$ and calculate,
                \[
                        \textstyle{ \norm{f \left 
                        ( \frac{1}{\sqrt{2}} (\ket{0} + \ket{1})\right ) } }
                        = \textstyle{ \norm{ 
                        \frac{1}{\sqrt{2}} (\ket{0} + \ket{0}) } }
                        = \textstyle{ \norm{ 
                        \frac{2}{\sqrt{2}} \ket{0} } } =
                        \textstyle{\frac{2}{\sqrt{2}}} 
                \]
        \end{example}
\end{frame}

\begin{frame}{Turning $f$ into a Quantum Operation pt. II}

        What is the problem intuitively?
        \pause

        $f$ potentially \alert{\underline{loses information}} and it is general
        consensus that pure quantum operations are \tikzmark{z1}reversible
        \tikzmark{z2}

        \begin{tikzpicture}[overlay,remember picture,
        box/.style = {rounded corners},
        pin edge={-Stealth,thick, red}]
        \coordinate (z1) at ($({pic cs:z1})+(+0.5ex, 1.5ex)$);
        \coordinate (z2) at ($({pic cs:z2})+(-0.5ex,-1.5ex)$);
        \node[semitransparent, 
             fit=(z1) (z2),
             pin=below:\tiny{Charles Bennett, 1973}]  {};
        \end{tikzpicture}

        \vspace{0.8cm}
        \textbf{N.b.}: isometricity implies injectivity so if a map loses
        information it cannot be isometric
\end{frame}

\begin{frame}{Turning $f$ into a Quantum Operation pt. III}
 
        \begin{block}{Proposed Solution}
                \begin{center}
                $\left \sem{
                \begin{quantikz}[transparent]
                &[4mm] \gate{U_f} \qwbundle{2} &[3mm] \qw \qwbundle{2}
                \end{quantikz} \right }
                =  
                \ket{x}\otimes \ket{y} \mapsto \ket{x} \otimes 
                \ket{y \tikzmark{a1}\oplus\tikzmark{a2} f(x)}$
                \end{center}
        \end{block}

        \begin{tikzpicture}[overlay,remember picture,
        box/.style = {rounded corners},
        pin edge={-Stealth,thick, red}]
        \coordinate (a1) at ($({pic cs:a1})+(+0.5ex, 1.5ex)$);
        \coordinate (a2) at ($({pic cs:a2})+(-0.5ex,-1.5ex)$);
        \node[semitransparent, 
             fit=(a1) (a2),
             pin=below:\tiny{Addition modulo 2}]  {};
        \end{tikzpicture}

        $U_f$ encodes $f$: $U_f(\ket{x} \otimes \ket{0}) = \ket{x} \otimes
        \ket{0 \oplus f(x)} = \ket{x} \otimes \ket{f(x)}$

        $U_f$ is reversible, in particular
        \[
                \begin{quantikz}[transparent]
                        &[4mm] \gate{U_f} \qwbundle{2} &[3mm]  \gate{U_f} &[4mm] \qw \qwbundle{2}
                \end{quantikz}
                = 
                \begin{quantikz}
                                & \qw & \qw 
                \end{quantikz}\
        \]
\end{frame}

\begin{frame}{Tackling the Problem via Quantum Parallelism pt. I}

        Need to somehow evaluate $f$ with $\ket{0}$ and $\ket{1}$
        in one step
        
        So take the circuit
        \begin{center} 
                $\left \sem{ 
                        \begin{quantikz}[transparent]
                        &\gate{H} & \gate[wires=2]{U_f} & \qw \\
                        & \qw & & \qw 
                        \end{quantikz}
                \right } = U_f (H \otimes I)$
        \end{center} 
        and calculate \dots
        \end{frame}

\begin{frame}{Tackling the Problem via Quantum Parallelism pt. II}

        \begin{flalign*}
             & \, U_f (H \otimes I) \ket{0} \otimes \ket{0} & \\
             & =  U_f \left ( \textstyle{\frac{1}{\sqrt{2}}(\ket{0} + \ket{1})}
               \otimes \ket{0}  \right ) & \text{\{Defn. of $H$ and $I$\}}  \\
             & =  U_f \left ( \textstyle{ \frac{1}{\sqrt{2}}(\ket{00} + \ket{10})}
                 \right ) & \text{\{$\otimes$ distributes over $+$\}} \\
             & = \textstyle{\frac{1}{\sqrt{2}}(\ket{0} \ket{0 \oplus \alert{f(0)}} 
               + \ket{1} \ket{0 \oplus \alert{f(1)}})} & \text{\{Defn. of $U_f$\}} \\
             & = \underbrace{\textstyle{\frac{1}{\sqrt{2}}(\ket{0} \ket{\alert{f(0)}} 
               + \ket{1} \ket{\alert{f(1)}})}}_{\text{$f(0)$ and $f(1)$ in a single run}} & \{0 \oplus x = x\}
        \end{flalign*}

        \dots\ cannot extract this information from the resultant state :(

        but fortunately no need to know the values of $f(0)$ and $f(1)$ -- only
        whether they are equal or not
\end{frame}

\begin{frame}{Tackling the Problem via Parallelism and Interference pt. I}

        We create an \alert{\underline{interference pattern}} dependent on this property
        \begin{center}
                        \begin{quantikz}[transparent]
                                \lstick{\ket{0}} &\gate{H} 
                                \gategroup[wires=1,steps=1,style={dashed,
                                rounded corners,fill=blue!10, inner xsep=0.2pt},background]
                                {\tiny{parallelism}}
                                & \gate[wires=2]{U_f} \gategroup[wires=2,steps=1,style={dashed,
                                rounded corners,fill=blue!10, inner xsep=0.2pt}, 
                                label style={label position = 
                                below, yshift=-0.4cm},background]
                                {\tiny{interference pattern}}
                                & \gate{H} 
                                \gategroup[wires=1,steps=1,style={dashed,
                                rounded corners,fill=blue!10, inner xsep=0.2pt},background]{
                                \tiny{wave collapse}}
                                & \qw \\
                                \lstick{$\frac{1}{\sqrt{2}} (\ket{0} - \ket{1})$}
                                                 & \qw & & \qw & \qw 
                        \end{quantikz}
        \end{center} 
        \dots\ and the wave collapse informs us 
\end{frame}

\begin{frame}{An Auxiliary Computation}
        \small{
        \begin{flalign*}
                & \, U_f \left (\ket{x} \otimes 
                (\ket{0} - \ket{1}) \right ) &  \\
                & =  
                U_f \left (\ket{x}\ket{0} - \ket{x}\ket{1} \right ) &
                \{\text{$\otimes$ distributes over $+$ }\} \\
                & =  
                \ket{x}\ket{0 \oplus f(x)} - \ket{x}\ket{1 \oplus f(x)} &
                \{\text{Defn. of $f$}\} \\
                & =   
                \ket{x}\ket{\alert{f(x)}} - 
                \ket{x}\ket{\alert{\neg f(x)}}  &  
                \{0 \oplus x = x, 1 \oplus x = \neg x \} \\
                & =  
                \ket{x} \otimes (\ket{\alert{f(x)}} - 
                \ket{\alert{\neg f(x)}}) &
                \{ \text{$\otimes$ distributes over $+$} \}
        \end{flalign*}
        
        We then proceed by case distinction
        \[
        \ket{x} \otimes (\ket{f(x)} - 
                \ket{\neg f(x)}) = 
                \begin{cases}
                        \ket{x} \otimes (\ket{0} - \ket{1}) & \text{ if } f (x) = 0 \\
                        \ket{x} \otimes (\ket{1} - \ket{0}) & \text{ if } f (x) = 1
                \end{cases}
        \]
        and conclude
        \[
           \ket{x} \otimes (\ket{f(x)} -  \ket{\neg f(x)}) =      
           (-1)^{f(x)} \ket{x} \otimes (\ket{0} - \ket{1})
        \]
        }
\end{frame}

\begin{frame}{Tackling the Problem via Parallelism and Interference pt. II}
        \small{
        \begin{flalign*}
              & \,  (H \otimes I)  U_f (H \otimes I)
              \left ( \ket{0} \otimes \ket{-} 
              \right ) & \\[1mm]
              & = (H \otimes I)  U_f
              \left ( \ket{+} \otimes \ket{-} 
      \right ) & \text{\{\dots\}} \\[1mm]
              & = \textstyle{\frac{1}{\sqrt{2}}} (H \otimes I)  U_f
              \left ( (\ket{0} + \ket{1})
                      \otimes \ket{-} 
\right ) & \text{\{\dots\}} \\[1mm]
              & = \textstyle{\frac{1}{\sqrt{2}}} (H \otimes I) 
              \left ( U_f \ket{0} \otimes \ket{-} 
                      + 
                      U_f \ket{1} \otimes \ket{-} 
\right ) & \text{\{\dots\}} \\[1mm]
              & = \textstyle{\frac{1}{\sqrt{2}}} (H \otimes I) 
              \left ( (-1)^{f(0)} \ket{0} \otimes \ket{-} 
                      + 
                      (-1)^{f(1)} \ket{1} \otimes \ket{-} 
\right ) & \text{\{Previous slide\}}\\[1mm]
              & = \begin{cases}
                      (H \otimes I) (\pm 1) \alert{\ket{+}} 
                        \otimes \ket{-} & \text{if } f(0) = f(1) \\
                        (H \otimes I) (\pm 1) \alert{\ket{-}} \otimes \ket{-} 
                                        & \text{if } f(0) \not = f(1)
                \end{cases} & \text{\{Case distinction\}} \\[1mm]
              & = \begin{cases}
                      (\pm 1) \alert{\ket{0}} \otimes \ket{-} & \text{if } f(0) = f(1) \\
                      (\pm 1) \alert{\ket{1}} \otimes \ket{-} & \text{if } f(0) \not = f(1)
              \end{cases}& \text{\{\dots\}} 
      \end{flalign*}
      }
\end{frame}

\begin{frame}{What's Next?}

        A simplification of the first algorithm with `quantum advantage'
        presented in literature [Deutsch, 1985]

        All other quantum algorithms crucially rely on similar ideas
        of quantum interference 

        We will study them in the following lectures
\end{frame}
\end{document}
