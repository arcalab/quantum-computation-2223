\documentclass{beamer}
\usepackage{etex} % fixes new-dimension error
\usepackage{lmodern}
\usepackage[T1]{fontenc}
%%%%%%%%%%%%% Macros
%%%% Categories
\newcommand{\catfont}[1]{\mathsf{#1}}
\newcommand{\cop}{\catfont{op}}
\newcommand{\Law}{\catfont{Law}}
\newcommand{\catV}{\catfont{V}}
\newcommand{\catX}{\catfont{X}}
\newcommand{\catC}{\catfont{C}}
\newcommand{\catD}{\catfont{D}}
\newcommand{\catA}{\catfont{A}}
\newcommand{\catB}{\catfont{B}}
\newcommand{\catI}{\catfont{I}}
\newcommand{\Set}{\catfont{Set}}
\newcommand{\Top}{\catfont{Top}}
\newcommand{\Pos}{\catfont{Pos}}
\newcommand{\Inj}{\catfont{Inj}}
\newcommand{\Det}{\catfont{RMhat}}
\newcommand{\CoAlg}[1]{\catfont{CoAlg}\left (#1 \right )}
\newcommand{\Mon}{\catfont{Mon}}
\newcommand{\Mnd}{\catfont{Mnd}(\catC)}
\newcommand{\SMnd}{\catfont{Mnd}(\Set)}
\newcommand{\CLat}{\catfont{CLat}}
\newcommand{\Stone}{\catfont{Stone}}
\newcommand{\Spectral}{\catfont{Spectral}}
\newcommand{\CompHaus}{\catfont{CompHaus}}
\newcommand{\Subs}[2]{\catfont{Sub}_{}}
\newcommand{\Cone}{\catfont{Cone}}
\newcommand{\StComp}{\catfont{StablyComp}}
\newcommand{\PosC}{\catfont{PosComp}}
\newcommand{\Haus}{\catfont{Haus}}
\newcommand{\Meas}{\catfont{Meas}}
\newcommand{\Ord}{\catfont{Ord}}
\newcommand{\EndoC}{[\catC,\catC]}
%% General functors
\newcommand{\funfont}[1]{#1}
\newcommand{\funF}{\funfont{F}}
\newcommand{\funU}{\funfont{U}}
\newcommand{\funG}{\funfont{G}}
\newcommand{\funT}{\funfont{T}}
\newcommand{\funI}{\funfont{I}}
%% Particular kinds of functors
\newcommand{\sfunfont}[1]{\mathrm{#1}}
\newcommand{\Pow}{\sfunfont{P}}
\newcommand{\Dist}{\sfunfont{D}}
\newcommand{\Maybe}{\sfunfont{M}}
\newcommand{\List}{\sfunfont{L}}
\newcommand{\UForg}{\sfunfont{U}}
\newcommand{\Forg}[1]{\sfunfont{U}_{#1}}
\newcommand{\Id}{\sfunfont{Id}}
\newcommand{\Vie}{\sfunfont{V}}
\newcommand{\Disc}{\funfont{D}}
\newcommand{\Weight}{\sfunfont{W}}
\newcommand{\homf}{\sfunfont{hom}}
\newcommand{\Yoneda}{\sfunfont{Y}}
%% Diagram functors
\newcommand{\Diag}{\mathscr{D}}
\newcommand{\KDiag}{\mathscr{K}}
\newcommand{\LDiag}{\mathscr{L}}
%% Monads
\newcommand{\monadfont}[1]{\mathbb{#1}}
\newcommand{\monadT}{\monadfont{T}}
\newcommand{\monadS}{\monadfont{S}}
\newcommand{\monadU}{\monadfont{U}}
\newcommand{\monadH}{\monadfont{H}}
\newcommand{\str}{\mathrm{str}}
%% Adjunctions
\newcommand\adjunct[2]{\xymatrix@=8ex{\ar@{}[r]|{\top}\ar@<1mm>@/^2mm/[r]^{{#2}}
& \ar@<1mm>@/^2mm/[l]^{{#1}}}}
\newcommand\adjunctop[2]{\xymatrix@=8ex{\ar@{}[r]|{\bot}\ar@<1mm>@/^2mm/[r]^{{#2}}
& \ar@<1mm>@/^2mm/[l]^{{#1}}}}
%% Retractions
\newcommand\retract[2]{\xymatrix@=8ex{\ar@{}[r]|{}\ar@<1mm>@/^2mm/@{^{(}->}[r]^{{#2}}
& \ar@<1mm>@/^2mm/@{->>}[l]^{{#1}}}}
%% Limits
\newcommand{\pv}[2]{\langle #1, #2 \rangle}
\newcommand{\limt}{\mathrm{lim}}
\newcommand{\pullbackcorner}[1][dr]{\save*!/#1+1.2pc/#1:(1,-1)@^{|-}\restore}
\newcommand{\pushoutcorner}[1][dr]{\save*!/#1-1.2pc/#1:(-1,1)@^{|-}\restore}
%% Colimits
\newcommand{\colim}{\mathrm{colim}}
\newcommand{\inl}{\mathrm{inl}}
\newcommand{\inr}{\mathrm{inr}}
%% Distributive categories
\newcommand{\distr}{\mathrm{dist}}
\newcommand{\undistr}{\mathrm{undist}}
%% Closedness
\newcommand{\curry}[1]{\mathrm{curry}{#1}}
\newcommand{\app}{\mathrm{app}}
%% Misc. operations
\newcommand{\const}[1]{\underline{#1}}
\newcommand{\comp}{\cdot}
\newcommand{\id}{\mathrm{id}}
%% Factorisations
\newcommand{\EClass}{E}
\newcommand{\MClass}{M}
\newcommand{\MConeClass}{\mathcal{M}}
%%%%%%%%%%%%%%%% End of Categorical Stuff

%%%% Misc
%% Operations
\newcommand{\blank}{\, - \,}
\newcommand{\sem}[1]{\llbracket #1 \rrbracket}
\newcommand{\closure}[1]{\overline{#1}}
\DeclareMathOperator{\img}{\mathrm{im}}
\DeclareMathOperator{\dom}{\mathrm{dom}}
\DeclareMathOperator{\codom}{\mathrm{codom}}
%% Sets of numbers
\newcommand{\Nats}{\mathbb{N}}
\newcommand{\Reals}{\mathbb{R}}
\newcommand{\Rz}{\Reals_{\geq 0}}
%% Writing
\newcommand{\cf}{\emph{cf.}}
\newcommand{\ie}{\emph{i.e.}}
\newcommand{\eg}{\emph{e.g.}}
\newcommand{\df}[1]{\emph{\textbf{#1}}}
%%%%%%%%%%%%%%%% End of Misc

%%%% Programming Stuff
%% Types
\newcommand{\typefont}[1]{\mathbb{#1}}
\newcommand{\typeOne}{1}
\newcommand{\typeTwo}{2}
\newcommand{\typeA}{\typefont{A}}
\newcommand{\typeB}{\typefont{B}}
\newcommand{\typeC}{\typefont{C}}
\newcommand{\typeV}{\typefont{V}}
\newcommand{\typeD}{\typefont{D}}
%% RuleName
\newcommand{\rulename}[1]{(\mathrm{#1})}
%% Sequents
\newcommand{\jud}{\vdash}
\newcommand{\vljud}{\vdash}
\newcommand{\cojud}{\vdash_{\co}}
\newcommand{\vl}{\mathtt{v}}
\newcommand{\co}{\mathtt{c}}
% Program font
\newcommand{\prog}[1]{\mathtt{#1}}
\newcommand{\pseq}[3]{#1 \leftarrow #2; #3}
\newcommand{\ppm}[4]{(#1,#2) \leftarrow #3; #4}
\newcommand{\pinl}[1]{\prog{inl}(#1)}
\newcommand{\pinr}[1]{\prog{inr}(#1)}
\newcommand{\pcase}[4]{\prog{ case } #1 \prog{ of } \pinl{#2} \Rightarrow #3 ; \pinr{#2} \Rightarrow #4}
%% Sets of terms
\newcommand{\ValuesBP}[2]{\mathsf{Values}(#1, #2)}
\newcommand{\TermsBP}[2]{\mathsf{Terms}(#1, #2)}
\newcommand{\closValP}[1]{\ValuesBP{\emptyset}{#1}}
\newcommand{\closTermP}[1]{\TermsBP{\emptyset}{#1}}
\newcommand{\closVal}{\closValP{\typeA}}
\newcommand{\closTerm}{\closTermP{\typeA}}
%% Contextual equivalence
\newcommand{\ctxeq}{\equiv_{\prog{ctx}}}
%%%% End of Programming Stuff

%-------------- template --------------------------------------------------
\usetheme{metropolis}
\metroset{block=fill}
%\usetheme{Boadilla}
% Configuring the foot line
\setbeamertemplate{footline}
{
  \leavevmode%
  \hbox{%
  \begin{beamercolorbox}[wd=.4\paperwidth,ht=2.25ex,dp=1ex,center]{author in head/foot}%
    \usebeamerfont{author in head/foot}\insertshortauthor
  \end{beamercolorbox}%
  \begin{beamercolorbox}[wd=.5\paperwidth,ht=2.25ex,dp=1ex,center]{title in head/foot}%
    \usebeamerfont{title in head/foot}\insertsection
  \end{beamercolorbox}%
  \begin{beamercolorbox}[wd=.1\paperwidth,ht=2.25ex,dp=1ex,right]{date in head/foot}%
    \insertframenumber{} / \inserttotalframenumber\hspace*{2ex} 
  \end{beamercolorbox}}%
  \vskip0pt%
}
% No configuration symbols
\setbeamertemplate{navigation symbols}{}
%--------------- packages -------------------------------------------------
\usepackage{graphicx,amsmath}
\usepackage{stmaryrd} % cf. interleave
\usepackage{booktabs}
\usepackage{amscd}
\usepackage{multicol}
\usepackage[absolute,overlay]{textpos}
\usepackage{alltt}
\usepackage{proof}
\usepackage{subcaption}
\usepackage{tikz}
\usepackage{tikz-cd}
\usepackage[new]{old-arrows}
\usepackage[all]{xy}
\usepackage{pgfplots}
\usepackage{textcomp}
\usetikzlibrary{arrows.meta, calc, fit, tikzmark}
\usepackage{pstricks,pst-node,pst-text,pst-3d}

% context
\AtBeginSection[]
{
    \begin{frame}
        \frametitle{Table of Contents}
        \tableofcontents[currentsection]
    \end{frame}
}

\author[Renato Neves]{Renato Neves}

% logos of institutions
\titlegraphic{
  \begin{textblock*}{5cm}(6.7cm,7.45cm)
     \includegraphics[scale=0.06]{logos/uminho.png}\hspace*{.85cm}~%
  \end{textblock*}
  \begin{textblock*}{5cm}(9.4cm,7.42cm)
    \includegraphics[scale=0.50]{logos/haslab.pdf}
  \end{textblock*}
}

% No date
\date{}

\begin{document}

\title{Quantum Computation 2022/23}

\frame[plain]{\titlepage}

\section{The Context}

\begin{frame}{Context}

  Quantum Computing is coming of age

  \dots\ moving from a potential far-future technology
  to a stage where prototypes become available and
  \alert{major investments} arise
  \begin{itemize}
  \item Companies (IBM, Google, Microsoft, and Intel)
  \item Public investment (UK, Sweden, Canada, Australia, Portugal)
  \item EU Flagship initiative with a 10 year timespan and an
  estimated budget of over one billion euros
  \end{itemize}
\end{frame}

\begin{frame}{Why the Big Interest?}
        
  A strategic use of quantum effects potentially provides remarkable speedups
  to certain kinds of \alert{computational tasks} 

  \begin{itemize}
  \item Cryptography
  \item Molecular simulation and weather prediction
  \item Processing of large data
  \end{itemize}

\end{frame}

\begin{frame}{A Concrete Example}

  Cryptographic schemes often assume that factoring large integers is
  computationally intractable
  
  In 1994 Peter Shor presented a quantum \alert{algorithm} for factoring
  integers that runs in \alert{polynomial time} 

  \vspace{2.5cm}

  \begin{textblock*}{4cm}(8.1cm,5.25cm)
    \includegraphics[scale=0.2]{images/size.jpg}
    \tiny{smbc-comics}
  \end{textblock*}

\end{frame}


\section{A Brief History of Quantum Computation}

\begin{frame}{Modern Reincarnation of Computing Science}

  \begin{minipage}[0.3\textheight]{\textwidth}
  \begin{columns}[c]
  \begin{column}{0.7\textwidth}
          On Computable Numbers, with an Application to the 
          \tikzmark{z1}\emph{Entscheidungsproblem}\tikzmark{z2}, \textbf{1936}
  \end{column}
  \begin{column}{0.3\textwidth}
    \includegraphics[scale=2]{images/turing.jpg}
    \tiny{Alan Turing}
  \end{column}
  \end{columns}
  \end{minipage}

  \begin{tikzpicture}[overlay,remember picture,
       box/.style = {rounded corners},
       pin edge={-Stealth,thick, red}]
       \coordinate (z1) at ($({pic cs:z1})+(+0.5ex, 1.5ex)$);
       \coordinate (z2) at ($({pic cs:z2})+(-0.5ex,-0.5ex)$);
       \node[semitransparent, 
             fit=(z1) (z2),
             pin=below:\tiny{Homework: See/read 
             \emph{The Hitchhiker's Guide to the Galaxy}}]  {};
   \end{tikzpicture}

  \vfill
  \begin{center}
  \scriptsize{
        A computational model and  computability
  }
  \end{center}
\end{frame}

\begin{frame}{A Case for Quantum Computing}

  \begin{minipage}[0.3\textheight]{\textwidth}
  \begin{columns}[c]
  \begin{column}{0.62\textwidth}
          Simulating Physics with Computers, \textbf{1982}
  \end{column}
  \begin{column}{0.39\textwidth}
    \includegraphics[scale=1]{images/feynman.jpg}
    \tiny{Richard Feynman}
  \end{column}
  \end{columns}
  \end{minipage}

  \vfill
  \begin{center}
  \scriptsize{
          "\dots \emph{because nature isn't
                  classical, dammit, and if you want to make a simulation of nature, you'd
                  better make it quantum mechanical, and by golly it's a wonderful problem,
                  because it doesn't look so easy.}"
  }
  \end{center}
\end{frame}

\begin{frame}{Mathematical Demonstration of `Quantum Advantage'}

  \begin{minipage}[0.3\textheight]{\textwidth}
  \begin{columns}[c]
  \begin{column}{0.6\textwidth}
        Quantum theory, the Church-Turing principle and the universal quantum computer,
        \textbf{1985}
  \end{column}
  \begin{column}{0.44\textwidth}
    \includegraphics[scale=0.2]{images/deutsch.jpg}
    \tiny{David Deutsch}
  \end{column}
  \end{columns}
  \end{minipage}

  \vfill
  \begin{center}
  \scriptsize{
        A quantum computational model and quantum computability: first example of a quantum
        algorithm that is remarkably faster than any known classical counterpart
  }
  \end{center}
\end{frame}

\begin{frame}{The Field of Quantum Computation}
  \begin{columns}[c]
  \begin{column}{0.33\textwidth}
          \centering
          \small{Computability} \\
          \includegraphics[height=2.4cm]{images/turing.jpg}
  \end{column}
  \begin{column}{0.33\textwidth}
          \centering
          \small{Quantum Computing} \\
          \includegraphics[height=2.4cm]{images/feynman.jpg}
  \end{column}
  \begin{column}{0.4\textwidth}
          \centering
          \small{Quantum `Advantage'} \\
          \includegraphics[height=2.4cm]{images/deutsch.jpg}
  \end{column}
  \end{columns}

  \vspace{1cm}
  \dots\ and of course many other pioneers
\end{frame}

\section{Currently\dots}

\begin{frame}{At Present \dots}

        Prototype quantum computers are already available 

        Their viability demonstrated in problems that are difficult to
        handle classically 

        \begin{itemize}
                \item Sycamore, \textbf{2019}
                \item Zuchongzhi, \textbf{2021}
                \item \dots
        \end{itemize}

\end{frame}

\begin{frame}{However \dots}

        The quantum race has just started
        
        \begin{itemize}
                \item quantum computers currently \alert{unreliable} for
                        performing useful computational tasks

                \item difficult to anticipate 
                        their  evolution and future applications
                \item commercial/military potential in the short term (5 to 10 yrs) 
                        is still highly debatable
        \end{itemize}
\end{frame}

\section{Course's Structure and Pragmatics}

\begin{frame}{Learning Outcomes}

On successful completion of the course students will be able 

\begin{itemize}
\item to understand basic concepts of computability and computational complexity
\item to understand basic concepts and techniques in quantum algorithmics
\item to design and analyse quantum algorithms
\item to implement and run quantum algorithms in the Qiskit 
        open-source software development kit
\end{itemize}

\end{frame}


\begin{frame}{Course Information and Pragmatics}

Refer to the course's website at:
\begin{center}
        \url{lmf.di.uminho.pt/quantum-computation-2223/}
\end{center}

\end{frame}

\begin{frame}[plain]
  \begin{textblock*}{4cm}(2cm,2.25cm)
          \includegraphics[scale=0.5]{images/mot.jpg} 
          \tiny{PhD Comics}
  \end{textblock*}

\end{frame}

\end{document}
