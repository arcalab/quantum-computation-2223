\documentclass{beamer}
\usepackage{etex} % fixes new-dimension error
\usepackage{lmodern}
\usepackage[T1]{fontenc}
\usepackage{mathtools}
%%%%%%%%%%%%% Macros
%%%% Categories
\newcommand{\catfont}[1]{\mathsf{#1}}
\newcommand{\cop}{\catfont{op}}
\newcommand{\Law}{\catfont{Law}}
\newcommand{\catV}{\catfont{V}}
\newcommand{\catX}{\catfont{X}}
\newcommand{\catC}{\catfont{C}}
\newcommand{\catD}{\catfont{D}}
\newcommand{\catA}{\catfont{A}}
\newcommand{\catB}{\catfont{B}}
\newcommand{\catI}{\catfont{I}}
\newcommand{\Set}{\catfont{Set}}
\newcommand{\Top}{\catfont{Top}}
\newcommand{\Pos}{\catfont{Pos}}
\newcommand{\Inj}{\catfont{Inj}}
\newcommand{\Det}{\catfont{RMhat}}
\newcommand{\CoAlg}[1]{\catfont{CoAlg}\left (#1 \right )}
\newcommand{\Mon}{\catfont{Mon}}
\newcommand{\Mnd}{\catfont{Mnd}(\catC)}
\newcommand{\SMnd}{\catfont{Mnd}(\Set)}
\newcommand{\CLat}{\catfont{CLat}}
\newcommand{\Stone}{\catfont{Stone}}
\newcommand{\Spectral}{\catfont{Spectral}}
\newcommand{\CompHaus}{\catfont{CompHaus}}
\newcommand{\Subs}[2]{\catfont{Sub}_{}}
\newcommand{\Cone}{\catfont{Cone}}
\newcommand{\StComp}{\catfont{StablyComp}}
\newcommand{\PosC}{\catfont{PosComp}}
\newcommand{\Haus}{\catfont{Haus}}
\newcommand{\Meas}{\catfont{Meas}}
\newcommand{\Ord}{\catfont{Ord}}
\newcommand{\EndoC}{[\catC,\catC]}
%% General functors
\newcommand{\funfont}[1]{#1}
\newcommand{\funF}{\funfont{F}}
\newcommand{\funU}{\funfont{U}}
\newcommand{\funG}{\funfont{G}}
\newcommand{\funT}{\funfont{T}}
\newcommand{\funI}{\funfont{I}}
%% Particular kinds of functors
\newcommand{\sfunfont}[1]{\mathrm{#1}}
\newcommand{\Pow}{\sfunfont{P}}
\newcommand{\Dist}{\sfunfont{D}}
\newcommand{\Maybe}{\sfunfont{M}}
\newcommand{\List}{\sfunfont{L}}
\newcommand{\UForg}{\sfunfont{U}}
\newcommand{\Forg}[1]{\sfunfont{U}_{#1}}
\newcommand{\Id}{\sfunfont{Id}}
\newcommand{\Vie}{\sfunfont{V}}
\newcommand{\Disc}{\funfont{D}}
\newcommand{\Weight}{\sfunfont{W}}
\newcommand{\homf}{\sfunfont{hom}}
\newcommand{\Yoneda}{\sfunfont{Y}}
%% Diagram functors
\newcommand{\Diag}{\mathscr{D}}
\newcommand{\KDiag}{\mathscr{K}}
\newcommand{\LDiag}{\mathscr{L}}
%% Monads
\newcommand{\monadfont}[1]{\mathbb{#1}}
\newcommand{\monadT}{\monadfont{T}}
\newcommand{\monadS}{\monadfont{S}}
\newcommand{\monadU}{\monadfont{U}}
\newcommand{\monadH}{\monadfont{H}}
\newcommand{\str}{\mathrm{str}}
%% Adjunctions
\newcommand\adjunct[2]{\xymatrix@=8ex{\ar@{}[r]|{\top}\ar@<1mm>@/^2mm/[r]^{{#2}}
& \ar@<1mm>@/^2mm/[l]^{{#1}}}}
\newcommand\adjunctop[2]{\xymatrix@=8ex{\ar@{}[r]|{\bot}\ar@<1mm>@/^2mm/[r]^{{#2}}
& \ar@<1mm>@/^2mm/[l]^{{#1}}}}
%% Retractions
\newcommand\retract[2]{\xymatrix@=8ex{\ar@{}[r]|{}\ar@<1mm>@/^2mm/@{^{(}->}[r]^{{#2}}
& \ar@<1mm>@/^2mm/@{->>}[l]^{{#1}}}}
%% Limits
\newcommand{\pv}[2]{\langle #1, #2 \rangle}
\newcommand{\limt}{\mathrm{lim}}
\newcommand{\pullbackcorner}[1][dr]{\save*!/#1+1.2pc/#1:(1,-1)@^{|-}\restore}
\newcommand{\pushoutcorner}[1][dr]{\save*!/#1-1.2pc/#1:(-1,1)@^{|-}\restore}
%% Colimits
\newcommand{\colim}{\mathrm{colim}}
\newcommand{\inl}{\mathrm{inl}}
\newcommand{\inr}{\mathrm{inr}}
%% Distributive categories
\newcommand{\distr}{\mathrm{dist}}
\newcommand{\undistr}{\mathrm{undist}}
%% Closedness
\newcommand{\curry}[1]{\mathrm{curry}{#1}}
\newcommand{\app}{\mathrm{app}}
%% Misc. operations
\newcommand{\const}[1]{\underline{#1}}
\newcommand{\comp}{\cdot}
\newcommand{\id}{\mathrm{id}}
%% Factorisations
\newcommand{\EClass}{E}
\newcommand{\MClass}{M}
\newcommand{\MConeClass}{\mathcal{M}}
%%%%%%%%%%%%%%%% End of Categorical Stuff

%%%% Misc
%% Operations
\newcommand{\blank}{\, - \,}
\newcommand{\sem}[1]{\llbracket #1 \rrbracket}
\newcommand{\closure}[1]{\overline{#1}}
\DeclareMathOperator{\img}{\mathrm{im}}
\DeclareMathOperator{\dom}{\mathrm{dom}}
\DeclareMathOperator{\codom}{\mathrm{codom}}
%% Sets of numbers
\newcommand{\Nats}{\mathbb{N}}
\newcommand{\Reals}{\mathbb{R}}
\newcommand{\Rz}{\Reals_{\geq 0}}
%% Writing
\newcommand{\cf}{\emph{cf.}}
\newcommand{\ie}{\emph{i.e.}}
\newcommand{\eg}{\emph{e.g.}}
\newcommand{\df}[1]{\emph{\textbf{#1}}}
%%%%%%%%%%%%%%%% End of Misc

%%%% Programming Stuff
%% Types
\newcommand{\typefont}[1]{\mathbb{#1}}
\newcommand{\typeOne}{1}
\newcommand{\typeTwo}{2}
\newcommand{\typeA}{\typefont{A}}
\newcommand{\typeB}{\typefont{B}}
\newcommand{\typeC}{\typefont{C}}
\newcommand{\typeV}{\typefont{V}}
\newcommand{\typeD}{\typefont{D}}
%% RuleName
\newcommand{\rulename}[1]{(\mathrm{#1})}
%% Sequents
\newcommand{\jud}{\vdash}
\newcommand{\vljud}{\vdash}
\newcommand{\cojud}{\vdash_{\co}}
\newcommand{\vl}{\mathtt{v}}
\newcommand{\co}{\mathtt{c}}
% Program font
\newcommand{\prog}[1]{\mathtt{#1}}
\newcommand{\pseq}[3]{#1 \leftarrow #2; #3}
\newcommand{\ppm}[4]{(#1,#2) \leftarrow #3; #4}
\newcommand{\pinl}[1]{\prog{inl}(#1)}
\newcommand{\pinr}[1]{\prog{inr}(#1)}
\newcommand{\pcase}[4]{\prog{ case } #1 \prog{ of } \pinl{#2} \Rightarrow #3 ; \pinr{#2} \Rightarrow #4}
%% Sets of terms
\newcommand{\ValuesBP}[2]{\mathsf{Values}(#1, #2)}
\newcommand{\TermsBP}[2]{\mathsf{Terms}(#1, #2)}
\newcommand{\closValP}[1]{\ValuesBP{\emptyset}{#1}}
\newcommand{\closTermP}[1]{\TermsBP{\emptyset}{#1}}
\newcommand{\closVal}{\closValP{\typeA}}
\newcommand{\closTerm}{\closTermP{\typeA}}
%% Contextual equivalence
\newcommand{\ctxeq}{\equiv_{\prog{ctx}}}
%%%% End of Programming Stuff

%-------------- template --------------------------------------------------
\usetheme{metropolis}
\metroset{block=fill}
%\usetheme{Boadilla}
% Configuring the foot line
\setbeamertemplate{footline}
{
  \leavevmode%
  \hbox{%
  \begin{beamercolorbox}[wd=.4\paperwidth,ht=2.25ex,dp=1ex,center]{author in head/foot}%
    \usebeamerfont{author in head/foot}\insertshortauthor
  \end{beamercolorbox}%
  \begin{beamercolorbox}[wd=.5\paperwidth,ht=2.25ex,dp=1ex,center]{title in head/foot}%
    \usebeamerfont{title in head/foot}\insertsection
  \end{beamercolorbox}%
  \begin{beamercolorbox}[wd=.1\paperwidth,ht=2.25ex,dp=1ex,right]{date in head/foot}%
    \insertframenumber{} / \inserttotalframenumber\hspace*{2ex} 
  \end{beamercolorbox}}%
  \vskip0pt%
}
% No configuration symbols
\setbeamertemplate{navigation symbols}{}
%--------------- packages -------------------------------------------------
\usepackage{graphicx,amsmath}
\usepackage{stmaryrd} % cf. interleave
\usepackage{booktabs}
\usepackage{amscd}
\usepackage{multicol}
\usepackage[absolute,overlay]{textpos}
\usepackage{alltt}
\usepackage{proof}
\usepackage{subcaption}
\usepackage{tikz}
\usepackage{tikz-cd}
\usepackage{quantikz}
\usepackage[new]{old-arrows}
\usepackage[all]{xy}
\usepackage{pgfplots}
\usepackage{textcomp}
\usetikzlibrary{arrows.meta, calc, fit, tikzmark, fillbetween}
\usepackage{pstricks,pst-node,pst-text,pst-3d}
% context
\AtBeginSection[]
{
    \begin{frame}
        \frametitle{Table of Contents}
        \tableofcontents[currentsection]
    \end{frame}
}

\author[Renato Neves]{Renato Neves}

% logos of institutions
\titlegraphic{
  \begin{textblock*}{5cm}(6.7cm,7.45cm)
     \includegraphics[scale=0.06]{logos/uminho.png}\hspace*{.85cm}~%
  \end{textblock*}
  \begin{textblock*}{5cm}(9.4cm,7.42cm)
    \includegraphics[scale=0.50]{logos/haslab.pdf}
  \end{textblock*}
}

% No date
\date{}

\begin{document}

\title{Quantum Phase Estimation}

\frame[plain]{\titlepage}

\section{Introduction}

\begin{frame}{Quantum Phase Estimation}
        \begin{block}{The Problem}
                A unitary operator on $n$ qubits

                Corresponding eigenvector with eigenvalue $\alert{e^{i 2 \pi
                \phi}}$ ($0 \leq \phi < 1$)

                Find out $\phi$
        \end{block}

        \pause
        Problem is present in different situations
        \begin{itemize}
                \item Shor's algorithm
                \item Determining $n$º of solutions in unstructured search
                \item \dots
        \end{itemize}
\end{frame}        

\begin{frame}{An Important Subroutine}
        Previous problem uses an important subroutine called
        \begin{center}
                \alert{Quantum Fourier Transform (QFT)}
        \end{center}

        Essentially a \alert{\underline{change-of-basis}} operation which
        stores information concerning the computational basis in
        \alert{\underline{local phases}}

        \pause
        We start by studying this operation in detail
\end{frame}

\section{Quantum Fourier Transform}

\begin{frame}{Quantum Fourier Transform: One Qubit}

        Let $\omega = e^{i 2 \pi \frac{1}{2}}$
        \begin{align*}
                QFT_1 \ket{0} & = \textstyle{\frac{1}{\sqrt{2}} \big (\ket{0} 
                + \alert{1} \ket{1} \big )} & =
                        \textstyle{\frac{1}{\sqrt{2}} \Big (\ket{0} 
                        + \alert{\omega^{1 \cdot 0}} \ket{1} \Big )} &
                \\
                QFT_1 \ket{1} & = \textstyle{\frac{1}{\sqrt{2}} \big (\ket{0} 
                + \alert{(-1)} \ket{1} \big )}  & =
                \textstyle{\frac{1}{\sqrt{2}} \Big (\ket{0} 
                + \alert{\omega^{1 \cdot 1}} \ket{1} \Big )} &
        \end{align*}

        \pause
        Thus $QFT_1 = H$ and operation $H^{-1}$ allows to extract information
        encoded in local phases

        \pause
        In general $QFT_1 \ket{x} = \textstyle{\frac{1}{\sqrt{2}} \Big (\ket{0} 
        + \alert{\omega^{1 \cdot x}} \ket{1} \Big )}$
\end{frame}

\begin{frame}{Quantum Fourier Transform: Two Qubits}

        Let $\omega = e^{i 2 \pi \frac{1}{\alert{4}}}$
        \begin{align*}
                QFT_2 \ket{00} & = \textstyle{\frac{1}{\sqrt{2}} \big (\ket{0} 
                + \omega^{2 \cdot \alert{0}} \ket{1} \big )}  \otimes  
                \textstyle{\frac{1}{\sqrt{2}} \big (\ket{0} 
                + \omega^{1 \cdot \alert{0}} \ket{1} \big )}   \\
                QFT_2 \ket{01} & = \textstyle{\frac{1}{\sqrt{2}} \big (\ket{0} 
                + \omega^{2 \cdot \alert{1}} \ket{1} \big )}   \otimes
                \textstyle{\frac{1}{\sqrt{2}} \big (\ket{0} 
                + \omega^{1 \cdot \alert{1}} \ket{1} \big )}  \\
                QFT_2 \ket{10} & = \textstyle{\frac{1}{\sqrt{2}} \big (\ket{0} 
                + \omega^{2 \cdot \alert{2}} \ket{1} \big )}   \otimes
                \textstyle{\frac{1}{\sqrt{2}} \big (\ket{0} 
                + \omega^{1 \cdot \alert{2}} \ket{1} \big )}  \\
                QFT_2 \ket{11} & = \textstyle{\frac{1}{\sqrt{2}} \big (\ket{0} 
                + \omega^{2 \cdot \alert{3}} \ket{1} \big )}   \otimes
                \textstyle{\frac{1}{\sqrt{2}} \big (\ket{0} 
                + \omega^{1 \cdot \alert{3}} \ket{1} \big )} 
        \end{align*}

        \pause
        Use Bloch sphere to visualise $QFT_2 \ket{x}$ and note that 
        \begin{itemize}
               \item previously, info. of $\ket{x}$ encoded in poles and
                       now encoded in the \alert{$xz$-plane}
               \item for every \alert{$\omega$-rotation} on the rightmost 
                       qubit \alert{$2$} such rotations on the leftmost 
        \end{itemize}
\end{frame}

\begin{frame}{Quantum Fourier Transform: Two Qubits}

        {\scriptsize Recall $\omega = e^{i 2 \pi \frac{1}{\alert{4}}}$}

        In order to derive a circuit for $QFT_2$ calculate
        \begin{align*}
                QFT_2 \ket{x} & = \textstyle{\frac{1}{\sqrt{2}} \big (\ket{0} 
                + \omega^{2 \cdot \alert{x}} \ket{1} \big )}   \otimes
                \textstyle{\frac{1}{\sqrt{2}} \big (\ket{0} 
                + \omega^{1 \cdot \alert{x}} \ket{1} \big )} \\
                              & =
                              \textstyle{\frac{1}{\sqrt{2}} \big (\ket{0} 
                              + \omega^{\alert{2(2x_1 + x_2)}} \ket{1} \big )}\otimes
                              \textstyle{\frac{1}{\sqrt{2}} \big (\ket{0} 
                              + \omega^{\alert{2x_1 + x_2}} \ket{1} \big )} 
                              \\
                              & =
                              \textstyle{\frac{1}{\sqrt{2}} \big (\ket{0} 
                              + \omega^{\alert{4x_1 + 2 x_2}} \ket{1} \big )}\otimes
                              \textstyle{\frac{1}{\sqrt{2}} \big (\ket{0} 
                              + \omega^{\alert{2x_1 + x_2}} \ket{1} \big )} 
                              \\
                              & =
                              \textstyle{\frac{1}{\sqrt{2}} \big (\ket{0} 
                              + \omega^{\alert{4x_1}} \omega^{\alert{2 x_2}} \ket{1} \big )}
                              \otimes
                              \textstyle{\frac{1}{\sqrt{2}} \big (\ket{0} 
                              + \omega^{\alert{2x_1}} \omega^{\alert{x_2}} \ket{1} \big )} 
                              \\
                              & =
                              \textstyle{\frac{1}{\sqrt{2}} \big (\ket{0} 
                              + \omega^{\alert{2 x_2}} \ket{1} \big )}
                              \otimes
                              \textstyle{\frac{1}{\sqrt{2}} \big (\ket{0} 
                              + \omega^{\alert{2x_1}} \omega^{\alert{x_2}} \ket{1} \big )} 
                              \\
                              & =
                              \underbrace{\textstyle{\frac{1}{\sqrt{2}} \big (\ket{0} 
                              + (-1)^{\alert{x_2}} \ket{1} \big )}}_{H\ket{x2}}
                              \otimes
                              \underbrace{\textstyle{\frac{1}{\sqrt{2}} \big (\ket{0} 
                              + (-1)^{\alert{x_1}} \omega^{\alert{x_2}} \ket{1} \big )}}_{
                              cR_2((H\ket{x1}) \otimes \ket{x2})}
        \end{align*}

        \pause
        What is $cR_2$?

\end{frame}

\begin{frame}{Quantum Fourier Transform: Two Qubits}

        $R_2\ket{0} = \ket{0}$ and $R_2\ket{1} = \omega \ket{1}$.
        Intuitively $R_2$ rotates a vector in the $xz$-plane
        \alert{$\frac{\pi}{2}$} radians

        \pause
        $cR_2$ is the \alert{controlled} $R_2$ operation defined by
        \[
                cR_2 \ket{x}\ket{0} = \ket{x} \ket{0} \hspace{1cm}
                cR_2 \ket{x}\ket{1} = R_2 \ket{x} \ket{1}
        \]
        equivalently $cR_2\ket{0}\ket{x_2} = \ket{0} \ket{x_2}$
        and $cR_2 \ket{1}\ket{x_2} = \omega^{\alert{x_2}} \ket{1} \ket{x_2}$

        \pause
        Putting all pieces together we derive the QFT circuit for 2 qubits:
        \begin{center}
                \begin{quantikz}
                        \lstick{$\ket{x_1}$} & \gate{H} & \gate{R_2} & \qw &\swap{1} & \qw  \\
                        \lstick{$\ket{x_2}$} & \qw & \ctrl{-1} & \gate{H} &\targX{} & \qw 
                \end{quantikz}

        \end{center}
\end{frame}

\begin{frame}{Quantum Fourier Transform: Three Qubits}
        Let $\omega_n = e^{i 2 \pi \cdot \frac{1}{2^n}}$. Note
        that $\omega_n^2 = \omega_{n -1}$ and thus
        $\omega_n^{2^{n-1}} = e^{i \pi}$
        \begin{flalign*}
                & QFT_3 \ket{x} \\ & = \textstyle{\frac{1}{\sqrt{2}} \big (\ket{0} 
                + \omega^{4 \cdot \alert{x}}_3 \ket{1} \big )} \otimes
                \textstyle{\frac{1}{\sqrt{2}} \big (\ket{0} 
                + \omega^{2 \cdot \alert{x}}_3 \ket{1} \big )}  \otimes
                \textstyle{\frac{1}{\sqrt{2}} \big (\ket{0} 
                + \omega^{1 \cdot \alert{x}}_3 \ket{1} \big )} \\
                              & = \textstyle{\frac{1}{\sqrt{2}} \big (\ket{0} 
                              + (-1)^{\alert{x}} \ket{1} \big )} \otimes
                              \textstyle{\frac{1}{\sqrt{2}} \big (\ket{0} 
                              + \omega^{2 \cdot \alert{x}}_3 \ket{1} \big )}  \otimes
                              \textstyle{\frac{1}{\sqrt{2}} \big (\ket{0} 
                              + \omega^{1 \cdot \alert{x}}_3 \ket{1} \big )} 
                              \\
                              & = \textstyle{\frac{1}{\sqrt{2}} \big (\ket{0} 
                              + (-1)^{\alert{x_3}} \ket{1} \big )} \otimes
                              \textstyle{\frac{1}{\sqrt{2}} \big (\ket{0} 
                              + \omega^{2 \cdot \alert{x}}_3 \ket{1} \big )}  \otimes
                              \textstyle{\frac{1}{\sqrt{2}} \big (\ket{0} 
                              + \omega^{1 \cdot \alert{x}}_3 \ket{1} \big )} 
                              \\
                              & = H \ket{x_3} \otimes
                              \textstyle{\frac{1}{\sqrt{2}} \big (\ket{0} 
                              + \omega^{2 \cdot \alert{(4x_1+2x_2+x_3)}}_3
                               \ket{1} \big )}  \otimes
                              \textstyle{\frac{1}{\sqrt{2}} \big (\ket{0} 
                              + \omega^{1 \cdot \alert{x}}_3 \ket{1} \big )}
                              \\
                              & = H \ket{x_3} \otimes
                              \textstyle{\frac{1}{\sqrt{2}} \big (\ket{0} 
                              + \omega^{2 \cdot \alert{(4x_1+2x_2+x_3)}}_3
                               \ket{1} \big )}  \otimes
                              \textstyle{\frac{1}{\sqrt{2}} \big (\ket{0} 
                              + \omega^{1 \cdot \alert{x}}_3 \ket{1} \big )}
                              \\
                              & = H \ket{x_3} \otimes
                              \textstyle{\frac{1}{\sqrt{2}} \big (\ket{0} 
                              + \omega^{2 \cdot \alert{(4x_1+2x_2)}}_3 
                              \omega^{2 \cdot \alert{x_3}}_3 
                              \ket{1} \big )}\otimes
                              \textstyle{\frac{1}{\sqrt{2}} \big (\ket{0} 
                              + \omega^{1 \cdot \alert{x}}_3 \ket{1} \big )}
                              \\
                              & = H \ket{x_3} \otimes
                              \textstyle{\frac{1}{\sqrt{2}} \big (\ket{0} 
                              + \omega^{2 \cdot \alert{(2x_1+x_2)}}_2 
                              \omega^{\alert{x_3}}_2 
                              \ket{1} \big )}\otimes
                              \textstyle{\frac{1}{\sqrt{2}} \big (\ket{0} 
                              + \omega^{\alert{4x_1 + 2x_2 + x_3}}_3 \ket{1} \big )}
        \end{flalign*}
\end{frame}

\begin{frame}{Quantum Fourier Transform: Three Qubits}
 \begin{flalign*}
                &  \\ & \dots \dots \\
                              & = H \ket{x_3} \otimes
                              \textstyle{\frac{1}{\sqrt{2}} \big (\ket{0} 
                              + \omega^{2 \cdot \alert{(2x_1+x_2)}}_2 
                              \omega^{\alert{x_3}}_2 
                              \ket{1} \big )}\otimes
                              \textstyle{\frac{1}{\sqrt{2}} \big (\ket{0} 
                              + \omega^{\alert{4x_1 + 2x_2 + x_3}}_3 \ket{1} \big )}
                              \\
                              & = H \ket{x_3} \otimes
                              \textstyle{\frac{1}{\sqrt{2}} \big (\ket{0} 
                              + \omega^{2 \cdot \alert{(2x_1+x_2)}}_2 
                              \omega^{\alert{x_3}}_2 
                              \ket{1} \big )}\otimes
                              \textstyle{\frac{1}{\sqrt{2}} \big (\ket{0} 
                              + \omega^{\alert{4x_1 + 2x_2}}_3 \omega^{\alert{x_3}}_3
                              \ket{1} \big )}
                              \\
                              & = H \ket{x_3} \otimes
                              \textstyle{\frac{1}{\sqrt{2}} \big (\ket{0} 
                              + \omega^{2 \cdot \alert{(2x_1+x_2)}}_2 
                              \omega^{\alert{x_3}}_2 
                              \ket{1} \big )}\otimes
                              \textstyle{\frac{1}{\sqrt{2}} \big (\ket{0} 
                              + \omega^{2 \cdot \alert{(2x_1 + x_2)}}_3 \omega^{\alert{x_3}}_3
                              \ket{1} \big )}
                              \\
                              & = H \ket{x_3} \otimes
                              \underbrace{ \textstyle{\frac{1}{\sqrt{2}} \big (\ket{0} 
                              + \omega^{2 \cdot \alert{(2x_1+x_2)}}_2 
                              \omega^{\alert{x_3}}_2 
                              \ket{1} \big )}\otimes
                              \textstyle{\frac{1}{\sqrt{2}} \big (\ket{0} 
                              + \omega^{\alert{2x_1 + x_2}}_2 \omega^{\alert{x_3}}_3
                              \ket{1} \big )}}_{= \, \text{some controlled-rotations on } 
                                 QFT_2 \ket{x_1 x_2}}
        \end{flalign*}
\end{frame}

\begin{frame}{Quantum Fourier Transform: Three Qubits}

        $R_n\ket{0} = \ket{0}$ and $R_n\ket{1} = \omega_n \ket{1}$

        $cR_n$ is the \alert{controlled} $R_n$ operation defined by
        \[
                cR_n \ket{x}\ket{0} = \ket{x} \ket{0} \hspace{1cm}
                cR_n \ket{x}\ket{1} = R_n \ket{x} \ket{1}
        \]
        equivalently $cR_n\ket{0}\ket{y} = \ket{0} \ket{y}$
        and $cR_n \ket{1}\ket{y} = \omega^{\alert{y}}_n \ket{1} \ket{y}$

        \pause
        Putting all pieces together we derive the QFT circuit for 3 qubits:
        \begin{center}
                \begin{quantikz}
                        \lstick{$\ket{x_1}$} & \gate[wires=2]{QFT_2} & \qw
                                             & \gate{R_2} & \qw & 
                        \gate[wires=3]{sw(+_{mod 3})} & \qw  \\
                        \lstick{$\ket{x_2}$} & & \gate{R_3} & \qw & \qw & \qw & \qw \\
                        \lstick{$\ket{x_3}$} & \qw & \ctrl{-1} & \ctrl{-2} & \gate{H} & \qw
                                             & \qw
                \end{quantikz}

        \end{center}
\end{frame}

\begin{frame}{Quantum Fourier Transform: $n$ Qubits}        
        Let $\omega_n = e^{i 2 \pi \cdot \frac{1}{2^n}}$
        \begin{flalign*}
                & QFT_n \ket{x} = \textstyle{\frac{1}{\sqrt{2}} \big (\ket{0} 
                + \omega^{2^{n-1} \cdot \alert{x}}_n \ket{1} \big )} \otimes
                \dots  \otimes
                \textstyle{\frac{1}{\sqrt{2}} \big (\ket{0} 
                + \omega^{2^{0} \cdot \alert{x}}_n \ket{1} \big )} 
        \end{flalign*}

        \pause
        The QFT circuit for n qubits:
        \begin{center}
                \begin{quantikz}
                        \lstick{$\ket{x_1}$} & \gate[wires=3]{QFT_{n}} & \qw
                                             & \gate{R_2} & \qw & 
                        \gate[wires=4]{sw(+_{mod {n+1}})} & \qw  \\
                        \dots \,  & \dots & \dots &  & \dots \,  & \dots & \dots &  \\
                        \lstick{$\ket{x_n}$} & & \gate{R_n} & \qw & \qw & \qw & \qw \\
                        \lstick{$\ket{x_{n+1}}$} & \qw & \ctrl{-1} & \ctrl{-3} & \gate{H} & \qw
                                             & \qw
                \end{quantikz}

        \end{center}
\end{frame}

\begin{frame}{Number of gates involved in the QFT}
        $\# gates (sw(+_{mod n})) = n -1$
        $\# gates (QFT_n) = n + n-1 + \# gates (QFT_{n-1})$

        Thus
        \[
                 \# gates (QFT_n) =  \sum_{i = 1}^{n} i + \sum_{i = 1}^{n} i - 1 =
                 \sum_{i = 1}^{n} i + \sum_{i = 1}^{n-1} i \approx
                 \frac{n^2}{2} + \frac{(n-1)^2}{2} 
        \]
        \[
                 \leq
                 \frac{n^2}{2} + \frac{n^2}{2}  = n^2
        \]
\end{frame}

\end{document}
