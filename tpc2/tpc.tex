\documentclass[a4paper, 11pt]{article}

%% packages
\usepackage{fullpage} % changes the margin
\usepackage{hyperref} % Links
\usepackage[utf8]{inputenc}
\usepackage{lmodern}
\usepackage{amsfonts}
\usepackage{amsthm}
\usepackage{amsmath}
\usepackage{braket}
\usepackage{graphicx}
\usepackage{tikz}
\usepackage{tikz-cd}
\usepackage{quantikz}
\usepackage[linewidth=1pt]{mdframed}
%%

%% environments
\theoremstyle{definition}
\newtheorem{definition}{Definition}
\newtheorem{examples}{Example}
\newtheorem{exercises}{Exercises}
\newtheorem{exercise}{Exercise}
\newtheorem{postulate}{Postulate}
%%

%% config
\date{}
\linespread{1.15}
%%

\begin{document}

\allowdisplaybreaks[2]
\title{Quantum Computation 2022/23\\ \small{TPC-2}}
\author{Renato Neves \\ \scriptsize
  \href{mailto:nevrenato@di.uminho.pt}{nevrenato@di.uminho.pt}}
\maketitle

\noindent
The goal of this assignment is to write a short essay (around 2 pages) that
lists and discusses applications of Grover's algorithm~\cite{nielsen16} to
different domains. Among other things, we will value essays whose claims and
ideas are tested and illustrated via implementations in \texttt{Qiskit}. For
example, a possibility is to illustrate via one such implementation the claim
that Grover's algorithm can be applied to a \textsc{SAT} problem with more than
one solution.

\begin{mdframed}
  What to submit: The report in \texttt{PDF}. Please send by email
  (\texttt{nevrenato@di.uminho.pt}) a unique zip file with the name
  ``\texttt{qc2122-N.zip}'', where ``\texttt{N}'' is your student number.  The
  subject of the email should be ``\texttt{qc2122 N TPC-2}''.
\end{mdframed}



%% Bibliography
\bibliographystyle{alpha}
\bibliography{biblio}

\end{document}
