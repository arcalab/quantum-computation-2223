\documentclass{beamer}
\usepackage{etex} % fixes new-dimension error
\usepackage{lmodern}
\usepackage[T1]{fontenc}
\usepackage{mathtools}
%%%%%%%%%%%%% Macros
%%%% Categories
\newcommand{\catfont}[1]{\mathsf{#1}}
\newcommand{\cop}{\catfont{op}}
\newcommand{\Law}{\catfont{Law}}
\newcommand{\catV}{\catfont{V}}
\newcommand{\catX}{\catfont{X}}
\newcommand{\catC}{\catfont{C}}
\newcommand{\catD}{\catfont{D}}
\newcommand{\catA}{\catfont{A}}
\newcommand{\catB}{\catfont{B}}
\newcommand{\catI}{\catfont{I}}
\newcommand{\Set}{\catfont{Set}}
\newcommand{\Top}{\catfont{Top}}
\newcommand{\Pos}{\catfont{Pos}}
\newcommand{\Inj}{\catfont{Inj}}
\newcommand{\Det}{\catfont{RMhat}}
\newcommand{\CoAlg}[1]{\catfont{CoAlg}\left (#1 \right )}
\newcommand{\Mon}{\catfont{Mon}}
\newcommand{\Mnd}{\catfont{Mnd}(\catC)}
\newcommand{\SMnd}{\catfont{Mnd}(\Set)}
\newcommand{\CLat}{\catfont{CLat}}
\newcommand{\Stone}{\catfont{Stone}}
\newcommand{\Spectral}{\catfont{Spectral}}
\newcommand{\CompHaus}{\catfont{CompHaus}}
\newcommand{\Subs}[2]{\catfont{Sub}_{}}
\newcommand{\Cone}{\catfont{Cone}}
\newcommand{\StComp}{\catfont{StablyComp}}
\newcommand{\PosC}{\catfont{PosComp}}
\newcommand{\Haus}{\catfont{Haus}}
\newcommand{\Meas}{\catfont{Meas}}
\newcommand{\Ord}{\catfont{Ord}}
\newcommand{\EndoC}{[\catC,\catC]}
%% General functors
\newcommand{\funfont}[1]{#1}
\newcommand{\funF}{\funfont{F}}
\newcommand{\funU}{\funfont{U}}
\newcommand{\funG}{\funfont{G}}
\newcommand{\funT}{\funfont{T}}
\newcommand{\funI}{\funfont{I}}
%% Particular kinds of functors
\newcommand{\sfunfont}[1]{\mathrm{#1}}
\newcommand{\Pow}{\sfunfont{P}}
\newcommand{\Dist}{\sfunfont{D}}
\newcommand{\Maybe}{\sfunfont{M}}
\newcommand{\List}{\sfunfont{L}}
\newcommand{\UForg}{\sfunfont{U}}
\newcommand{\Forg}[1]{\sfunfont{U}_{#1}}
\newcommand{\Id}{\sfunfont{Id}}
\newcommand{\Vie}{\sfunfont{V}}
\newcommand{\Disc}{\funfont{D}}
\newcommand{\Weight}{\sfunfont{W}}
\newcommand{\homf}{\sfunfont{hom}}
\newcommand{\Yoneda}{\sfunfont{Y}}
%% Diagram functors
\newcommand{\Diag}{\mathscr{D}}
\newcommand{\KDiag}{\mathscr{K}}
\newcommand{\LDiag}{\mathscr{L}}
%% Monads
\newcommand{\monadfont}[1]{\mathbb{#1}}
\newcommand{\monadT}{\monadfont{T}}
\newcommand{\monadS}{\monadfont{S}}
\newcommand{\monadU}{\monadfont{U}}
\newcommand{\monadH}{\monadfont{H}}
\newcommand{\str}{\mathrm{str}}
%% Adjunctions
\newcommand\adjunct[2]{\xymatrix@=8ex{\ar@{}[r]|{\top}\ar@<1mm>@/^2mm/[r]^{{#2}}
& \ar@<1mm>@/^2mm/[l]^{{#1}}}}
\newcommand\adjunctop[2]{\xymatrix@=8ex{\ar@{}[r]|{\bot}\ar@<1mm>@/^2mm/[r]^{{#2}}
& \ar@<1mm>@/^2mm/[l]^{{#1}}}}
%% Retractions
\newcommand\retract[2]{\xymatrix@=8ex{\ar@{}[r]|{}\ar@<1mm>@/^2mm/@{^{(}->}[r]^{{#2}}
& \ar@<1mm>@/^2mm/@{->>}[l]^{{#1}}}}
%% Limits
\newcommand{\pv}[2]{\langle #1, #2 \rangle}
\newcommand{\limt}{\mathrm{lim}}
\newcommand{\pullbackcorner}[1][dr]{\save*!/#1+1.2pc/#1:(1,-1)@^{|-}\restore}
\newcommand{\pushoutcorner}[1][dr]{\save*!/#1-1.2pc/#1:(-1,1)@^{|-}\restore}
%% Colimits
\newcommand{\colim}{\mathrm{colim}}
\newcommand{\inl}{\mathrm{inl}}
\newcommand{\inr}{\mathrm{inr}}
%% Distributive categories
\newcommand{\distr}{\mathrm{dist}}
\newcommand{\undistr}{\mathrm{undist}}
%% Closedness
\newcommand{\curry}[1]{\mathrm{curry}{#1}}
\newcommand{\app}{\mathrm{app}}
%% Misc. operations
\newcommand{\const}[1]{\underline{#1}}
\newcommand{\comp}{\cdot}
\newcommand{\id}{\mathrm{id}}
%% Factorisations
\newcommand{\EClass}{E}
\newcommand{\MClass}{M}
\newcommand{\MConeClass}{\mathcal{M}}
%%%%%%%%%%%%%%%% End of Categorical Stuff

%%%% Misc
%% Operations
\newcommand{\blank}{\, - \,}
\newcommand{\sem}[1]{\llbracket #1 \rrbracket}
\newcommand{\closure}[1]{\overline{#1}}
\DeclareMathOperator{\img}{\mathrm{im}}
\DeclareMathOperator{\dom}{\mathrm{dom}}
\DeclareMathOperator{\codom}{\mathrm{codom}}
%% Sets of numbers
\newcommand{\Nats}{\mathbb{N}}
\newcommand{\Reals}{\mathbb{R}}
\newcommand{\Rz}{\Reals_{\geq 0}}
%% Writing
\newcommand{\cf}{\emph{cf.}}
\newcommand{\ie}{\emph{i.e.}}
\newcommand{\eg}{\emph{e.g.}}
\newcommand{\df}[1]{\emph{\textbf{#1}}}
%%%%%%%%%%%%%%%% End of Misc

%%%% Programming Stuff
%% Types
\newcommand{\typefont}[1]{\mathbb{#1}}
\newcommand{\typeOne}{1}
\newcommand{\typeTwo}{2}
\newcommand{\typeA}{\typefont{A}}
\newcommand{\typeB}{\typefont{B}}
\newcommand{\typeC}{\typefont{C}}
\newcommand{\typeV}{\typefont{V}}
\newcommand{\typeD}{\typefont{D}}
%% RuleName
\newcommand{\rulename}[1]{(\mathrm{#1})}
%% Sequents
\newcommand{\jud}{\vdash}
\newcommand{\vljud}{\vdash}
\newcommand{\cojud}{\vdash_{\co}}
\newcommand{\vl}{\mathtt{v}}
\newcommand{\co}{\mathtt{c}}
% Program font
\newcommand{\prog}[1]{\mathtt{#1}}
\newcommand{\pseq}[3]{#1 \leftarrow #2; #3}
\newcommand{\ppm}[4]{(#1,#2) \leftarrow #3; #4}
\newcommand{\pinl}[1]{\prog{inl}(#1)}
\newcommand{\pinr}[1]{\prog{inr}(#1)}
\newcommand{\pcase}[4]{\prog{ case } #1 \prog{ of } \pinl{#2} \Rightarrow #3 ; \pinr{#2} \Rightarrow #4}
%% Sets of terms
\newcommand{\ValuesBP}[2]{\mathsf{Values}(#1, #2)}
\newcommand{\TermsBP}[2]{\mathsf{Terms}(#1, #2)}
\newcommand{\closValP}[1]{\ValuesBP{\emptyset}{#1}}
\newcommand{\closTermP}[1]{\TermsBP{\emptyset}{#1}}
\newcommand{\closVal}{\closValP{\typeA}}
\newcommand{\closTerm}{\closTermP{\typeA}}
%% Contextual equivalence
\newcommand{\ctxeq}{\equiv_{\prog{ctx}}}
%%%% End of Programming Stuff

%-------------- template --------------------------------------------------
\usetheme{metropolis}
\metroset{block=fill}
%\usetheme{Boadilla}
% Configuring the foot line
\setbeamertemplate{footline}
{
  \leavevmode%
  \hbox{%
  \begin{beamercolorbox}[wd=.4\paperwidth,ht=2.25ex,dp=1ex,center]{author in head/foot}%
    \usebeamerfont{author in head/foot}\insertshortauthor
  \end{beamercolorbox}%
  \begin{beamercolorbox}[wd=.5\paperwidth,ht=2.25ex,dp=1ex,center]{title in head/foot}%
    \usebeamerfont{title in head/foot}\insertsection
  \end{beamercolorbox}%
  \begin{beamercolorbox}[wd=.1\paperwidth,ht=2.25ex,dp=1ex,right]{date in head/foot}%
    \insertframenumber{} / \inserttotalframenumber\hspace*{2ex} 
  \end{beamercolorbox}}%
  \vskip0pt%
}
% No configuration symbols
\setbeamertemplate{navigation symbols}{}
%--------------- packages -------------------------------------------------
\usepackage{graphicx,amsmath}
\usepackage{stmaryrd} % cf. interleave
\usepackage{booktabs}
\usepackage{amscd}
\usepackage{multicol}
\usepackage[absolute,overlay]{textpos}
\usepackage{alltt}
\usepackage{proof}
\usepackage{subcaption}
\usepackage{tikz}
\usepackage{tikz-cd}
\usepackage{quantikz}
\usepackage[new]{old-arrows}
\usepackage[all]{xy}
\usepackage{pgfplots}
\usepackage{textcomp}
\usetikzlibrary{arrows.meta, calc, fit, tikzmark}
\usepackage{pstricks,pst-node,pst-text,pst-3d}
% context
\AtBeginSection[]
{
    \begin{frame}
        \frametitle{Table of Contents}
        \tableofcontents[currentsection]
    \end{frame}
}

\author[Renato Neves]{Renato Neves}

% logos of institutions
\titlegraphic{
  \begin{textblock*}{5cm}(6.7cm,7.45cm)
     \includegraphics[scale=0.06]{logos/uminho.png}\hspace*{.85cm}~%
  \end{textblock*}
  \begin{textblock*}{5cm}(9.4cm,7.42cm)
    \includegraphics[scale=0.50]{logos/haslab.pdf}
  \end{textblock*}
}

% No date
\date{}

\begin{document}

\title{Setting an Exponential Separation between Quantum and Classical Computation}

\frame[plain]{\titlepage}

\section{Overview}

\begin{frame}{Previously\dots}

        \begin{block}{The Problem}
        Take a function $f : \{0,1\} \to \{0,1\}$

        Either $f(0) = f(1)$ or $f(0) \not = f(1)$

        Tell us whether the first or second case hold
        \end{block}

        Classically, need to run $f$ \alert{twice}.  Quantumly, 
        \alert{once} is enough 

        \pause
        Can we have more impressive differences in complexity?

\end{frame}

\section{Global and local phases}

\begin{frame}{Global Phase Factor}

        \begin{definition}
        Let $v,u \in \Complex^{2^n}$ be vectors.  If $u = e^{i\theta}v$ we say
        that it is equal to $v$ up to \alert{global phase factor} $e^{i
        \theta}$
        \end{definition}

        \vfill
        \begin{theorem}
         $e^{i\theta}v$ and $v$ are indistinguishable
         in the world of quantum mechanics
        \end{theorem}

        \begin{block}{Proof sketch}
                Show that equality up to global phase is preserved by operators
                and normalisation + show that probability outcomes associated
                with $v$ and $e^{i\theta}v$ are the same
        \end{block}
\end{frame}

\begin{frame}{Relative Phase Factor}
        
        \begin{definition}
        We say that vectors $\sum_{x \in 2^n} \alpha_x \ket{x}$ and $\sum_{x
        \in 2^n} \beta_x \ket{x}$ differ by a \alert{relative phase factor} if
        for all $x \in 2^n$
        \[
                \alpha_x = e^{i \theta_x} \beta_x
                \qquad \text{(for some angle $\theta_x$)}
        \]
        \end{definition}       

        \begin{example}
               Vectors $\ket{0} + \ket{1}$ and $\ket{0} - \ket{1}$
                differ by a relative phase factor
        \end{example}

        \pause
        Vectors that differ by a relative phase factor are \alert{distinguishable}
\end{frame}

\section{Phase Kickback}

\begin{frame}{The Phase Kickback Effect pt. I}
        Recall that every quantum operation
        \begin{quantikz}
               &[4mm] \gate{U} \qwbundle{n} &[4mm] \qw \qwbundle{n}
        \end{quantikz}$\,$
        gives rise to a controlled quantum operation, which is depicted below
        \[ 
                \begin{quantikz}
                &[4mm] \ctrl{1} &[4mm] \qw  \\
                &[4mm] \gate{U} \qwbundle{n} &[4mm] \qw \qwbundle{n}
                \end{quantikz}
        \]

        Let $v$ be an \alert{\underline{eigenvector}} of $U$ (\ie\ $U v = e^{i
        \theta} v$) and calculate
        \begin{flalign*}
              &  \, \, cU \Big ( (\alpha \ket{0} + \beta \ket{1}) \otimes v \Big ) & \\
              & = cU (\alpha \ket{0} \otimes v + \beta \ket{1} \otimes v) & \\
              & = \alpha \ket{0} \otimes v + \beta \ket{1} \otimes \alert{e^{i\theta}} v & \\
              & = (\alpha \ket{0} + \alert{e^{i\theta}} \beta \ket{1}) \otimes v 
        \end{flalign*}
\end{frame}
\begin{frame}{The Phase Kickback Effect pt. II}
        What just happened?
        \pause
        \begin{itemize}
                \item Global phase $e^{i \theta}$ (introduced to $v$)
                        was 'kickedback' as a relative phase in the control
                        qubit
                        \pause
                \item Some information of $U$ is now encoded in the control
                        qubit
         \end{itemize}
         In general kickingback such phases causes
         \alert{interference patterns} that give away information 
         about $U$

\end{frame}

\begin{frame}{The Phase Kickback Effect pt. III}
        Consider the controlled-not operation 
        \[
               \begin{quantikz}
                & \ctrl{1} & \qw  \\
                & \targ{} & \qw 
                \end{quantikz}
        \]

        $X$ has $\ket{-}$ as eigenvector with associated
        eigenstate $-1$. It thus yields the equation
        \[
                cX \ket{b} \ket{-} = (-1)^{b} \ket{b} \ket{-}
        \]
        with $\ket{b}$ an element of the computational basis
\end{frame}

\begin{frame}{Back to Deutsch's Problem}

        \begin{center}
                \hspace{-1.5cm}
                        \begin{quantikz}[transparent]
                                \lstick{\ket{0}} &\gate{H} 
                                \gategroup[wires=1,steps=1,style={dashed,
                                rounded corners,fill=blue!10, inner xsep=0.2pt},background]
                                {\tiny{parallelism}}
                                & \gate[wires=2]{U_f} \gategroup[wires=2,steps=1,style={dashed,
                                rounded corners,fill=blue!10, inner xsep=0.2pt}, 
                                label style={label position = 
                                below, yshift=-0.4cm},background]
                                {\tiny{interference pattern}}
                                & \gate{H} 
                                \gategroup[wires=1,steps=1,style={dashed,
                                rounded corners,fill=blue!10, inner xsep=0.2pt},background]{
                                \tiny{wave collapse}}
                                & \qw \\
                                \lstick{$\ket{-}$}
                                                 & \qw & & \qw & \qw 
                        \end{quantikz}
        \end{center} 

        \vfill
        \pause
        $U_f$ can be seen as a \alert{generalised} controlled not-operation
        \[
                \left \sem{ 
               \begin{quantikz}
                & \ctrl{1} & \qw  \\
                & \gate{f} & \qw 
                \end{quantikz}
                \right } = 
                \ket{x}\ket{y} \mapsto 
                \begin{cases}
                        \ket{x} \ket{y} & \text{ if } f(x) = 0\\
                        \ket{x} \neg \ket{y} & \text{ if } f(x) = 1
                \end{cases}
        \]

\end{frame}

\begin{frame}{Back to Deutsch's Problem pt. II}

        $U_f$ can be seen as a \alert{generalised} controlled not-operation
        \[
                \left \sem{ 
               \begin{quantikz}
                & \ctrl{1} & \qw  \\
                & \gate{f} & \qw 
                \end{quantikz}
                \right } = 
                \ket{x}\ket{y} \mapsto 
                \begin{cases}
                        \ket{x} \ket{y} & \text{ if } f(x) = 0\\
                        \ket{y} \neg \ket{y} & \text{ if } f(x) = 1
                \end{cases}
        \]

        Recall that $\ket{-}$ is an eigenvector of $X$ with eigenstate $-1$.
        Thus analogously to before we deduce 
        \[
                \alert{ U_f \ket{x} \ket{-} = (-1)^{f(x)} \ket{x} \ket{-} }
        \]
\end{frame}

\begin{frame}{Back to Deutsch's Problem pt. III}

        \begin{center}
                \hspace{-1.5cm}
                        \begin{quantikz}[transparent]
                                \lstick{\ket{0}} &\gate{H} 
                                \gategroup[wires=1,steps=1,style={dashed,
                                rounded corners,fill=blue!10, inner xsep=0.2pt},background]
                                {\tiny{parallelism}}
                                & \ctrl{1} \gategroup[wires=2,steps=1,style={dashed,
                                rounded corners,fill=blue!10, inner xsep=0.2pt}, 
                                label style={label position = 
                                below, yshift=-0.4cm},background]
                                {\tiny{interference pattern (created by phase kickback)}}
                                & \gate{H} 
                                \gategroup[wires=1,steps=1,style={dashed,
                                rounded corners,fill=blue!10, inner xsep=0.2pt},background]{
                                \tiny{wave collapse}}
                                & \qw \\
                                \lstick{$\ket{-}$}
                                & \qw & \gate{\, f \,} & \qw & \qw 
                        \end{quantikz}
        \end{center} 

\end{frame}

\section{Bernstein-Vazirani's problem}

\begin{frame}{Going Beyond the Current Separation}

        Albeit looking almost magical how we handled Deutsch's problem, the
        corresponding complexity difference between quantum and classical is
        \alert{unimpressive}
        
        Can we come up with a more impressive separation?
\end{frame}

\begin{frame}{Setting the Stage}
        \begin{lemma}
                For $a,b \in \{0,1\}$ the equation $(-1)^a(-1)^b = (-1)^{a \oplus b}$
                holds
        \end{lemma}
        \begin{block}{Prook sketch}
                Build a truth table for each case and compare the corresponding
                contents
        \end{block}

        \vfill
        \begin{definition}
                Given two bit-strings $x,y \in \{0,1\}^n$ we define
                their product $x \cdot y \in \{0,1\}$ as
                        $x \cdot y = (x_1 \wedge y_1) \oplus \dots
                        \oplus (x_n \wedge y_n)$
        \end{definition}
\end{frame}

\begin{frame}{Setting the Stage}

        \begin{lemma}
                For any three binary strings $x,a,b \in \{0,1\}^n$ the equation
                $(x \cdot a) \oplus (x \cdot b) = x \cdot (a \oplus b)$ holds
        \end{lemma}

        \begin{block}{Proof sketch} 
                Follows from the fact that for any three bits $a,b,c \in
                \{0,1\}$ the equation $(a \wedge b) \oplus (a \wedge c) = a
                \wedge (b \oplus c)$ holds
        \end{block}
\end{frame}

\begin{frame}{Setting the Stage}
        \begin{lemma}
                For any element $\ket{b}$ in the computational basis of
                $\Complex^2$ we have $H \ket{b} =
                \textstyle{\frac{1}{\sqrt{2}}} \sum_{z \in 2} (-1)^{b \wedge z}
                \ket{z}$
        \end{lemma}
        \begin{block}{Proof sketch}
                Build a truth table and compare the corresponding contents
        \end{block}
        \vfill

        \begin{theorem}
                For any element $\ket{b}$ in the computational basis
                of $\Complex^{2^n}$ we have
                $H^{\otimes n} \ket{b} = 
                \textstyle{\frac{1}{\sqrt{2^n}}} \sum_{z \in 2^n} (-1)^{b \cdot z}
                \ket{z}$
        \end{theorem}
        \begin{block}{Proof sketch}
                Follows from induction on the size of $n$
        \end{block}
\end{frame}

\begin{frame}{Bernstein-Vazirani}

        \begin{block}{The Problem}
                Take a function $f : \{0,1\}^{\alert{n}} \to \{0,1\}$

                You are promised that $f(x) = s \cdot x$ for some fixed bit-string $s$

                Find $s$
        \end{block}

        Classically, we run $f$ $n$-times by computing
        \begin{flalign*}
                f(1 \dots 0) & = (\alert{s_1} \wedge 1) \oplus \dots \oplus (s_n \wedge 0) 
                = \alert{s_1} \\
                & \> \vdots \\
                f(0 \dots 1) & = (s_1 \wedge 0) \oplus \dots \oplus 
                (\alert{s_n} \wedge 1) = \alert{s_n}
        \end{flalign*}

        Quantumly, we discover $s$ by running $f$ only \alert{once}
\end{frame}

\begin{frame}{The Circuit}
        \begin{center}
                \hspace{-1.5cm}
                        \begin{quantikz}[transparent]
                                \lstick{\ket{0}} &[4mm] \gate{H^{\otimes n}} 
                                \gategroup[wires=1,steps=1,style={dashed,
                                rounded corners,fill=blue!10, inner xsep=0.2pt},background]
                                {\tiny{parallelism}} \qwbundle{n}
                                & \ctrl{1} \gategroup[wires=2,steps=1,style={dashed,
                                rounded corners,fill=blue!10, inner xsep=0.2pt}, 
                                label style={label position = 
                                below, yshift=-0.4cm},background]
                                {\tiny{interference pattern (created by phase kickback)}}
                                & \gate{H^{\otimes n}} 
                                \gategroup[wires=1,steps=1,style={dashed,
                                rounded corners,fill=blue!10, inner xsep=0.2pt},background]{
                                \tiny{wave collapse}} 
                                &[4mm] \qw \qwbundle{n} \\
                                \lstick{$\ket{-}$}
                                & \qw & \gate{\, f \,} & \qw & \qw 
                        \end{quantikz}
        \end{center} 
\end{frame}

\begin{frame}{The Computation}

       {\scriptsize
       \textbf{N.B.} In order to not overburden notation we omit $\ket{-}$ }
       \begin{flalign*}
       & \, H^{\otimes n} \ket{0} & \\
       & = \textstyle{\frac{1}{\sqrt{2^n}}} \sum_{z \in 2^n}  \ket{z} & 
       \text{\{Theorem slide 18\}} \\
       & \stackrel{U_f}{\mapsto} \textstyle{\frac{1}{\sqrt{2^n}}} 
               \sum_{z \in 2^n} (-1)^{f(z)} \ket{z} & \text{\{Definition slide 12\}} \\
       & \stackrel{H^{\otimes n}}{\mapsto} \textstyle{\frac{1}{2^n}} 
        \sum_{z \in 2^n} (-1)^{f(z)} \Big ( \sum_{z' \in 2^n} (-1)^{z \cdot z'} \ket{z'} 
       \Big ) &  
       \text{\{Theorem slide 18\}} \\
       & = \textstyle{\frac{1}{2^n}} 
       \sum_{z \in 2^n} \sum_{z' \in 2^n} (-1)^{(z \cdot s) \oplus (z \cdot z')}  \ket{z'}
       & \text{\{Lemma slide 16\}} \\
       & = \textstyle{\frac{1}{2^n}} 
       \sum_{z \in 2^n} \sum_{z' \in 2^n} (-1)^{z \cdot (s \oplus z')} \ket{z'} 
       & \text{\{Lemma slide 17\}} 
       \end{flalign*}
\end{frame}

\begin{frame}{The Computation pt. II}
        Probability of measuring $s$ at the end given by
        \begin{flalign*}
                & \, \big \lvert \textstyle{\frac{1}{2^n}} 
                \sum_{z \in 2^n} (-1)^{z \cdot (s \oplus s)} \ket{s} \big \rvert^2
                & \\
                & = \big \lvert \textstyle{\frac{1}{2^n}} 
                \sum_{z \in 2^n} (-1)^{z \cdot 0} \ket{s} \big \rvert^2
                & \\
                & = \big \lvert \textstyle{\frac{1}{2^n}} 
                \sum_{z \in 2^n} 1 \ket{s} \big \rvert^2
                &  \\
                & = \big \lvert \textstyle{\frac{2^n}{2^n}} 
                \big \rvert^2
                &  \\
                & = 1
        \end{flalign*}

        This means that somehow all values yielding wrong answers
        were completely cancelled

        {\scriptsize \textbf{T.P.C.} Show exactly how all the wrong answers
        were cancelled}
\end{frame}

\begin{frame}{Going Even Further Beyond}

        We went from running $f$ $n$ times
        to running just once

        \pause
        Still not very impressive (at least for the Computer Scientist :-)) 

        \pause
        Can we do even better?
\end{frame}

\section{Deutsch-Josza's problem}

\begin{frame}{Deutsch-Josza}
        \begin{block}{The Problem}
                Take a function $f : \{0,1\}^n \to \{0,1\}$

                You are promised that $f$ is either constant or balanced

                Find out which case holds
        \end{block}
        
        Classically, we evaluate half of the inputs ($\frac{2^{n}}{2} = 2^{n-1}$),
        evaluate one more and run the decision procedure,
        \begin{itemize}
                \item output always the same $\Longrightarrow$ constant
                \item otherwise $\Longrightarrow$ balanced
        \end{itemize}
        which requires running $f$  \alert{$2^{n-1} + 1$} times

        Quantumly, we know the answer by running $f$ only \alert{once}
\end{frame}

\begin{frame}{The Circuit}
        \begin{center}
                \hspace{-1.5cm}
                        \begin{quantikz}[transparent]
                                \lstick{\ket{0}} &[4mm] \gate{H^{\otimes n}} 
                                \gategroup[wires=1,steps=1,style={dashed,
                                rounded corners,fill=blue!10, inner xsep=0.2pt},background]
                                {\tiny{parallelism}} \qwbundle{n}
                                & \ctrl{1} \gategroup[wires=2,steps=1,style={dashed,
                                rounded corners,fill=blue!10, inner xsep=0.2pt}, 
                                label style={label position = 
                                below, yshift=-0.4cm},background]
                                {\tiny{interference pattern (created by phase kickback)}}
                                & \gate{H^{\otimes n}} 
                                \gategroup[wires=1,steps=1,style={dashed,
                                rounded corners,fill=blue!10, inner xsep=0.2pt},background]{
                                \tiny{wave collapse}} 
                                &[4mm] \qw \qwbundle{n} \\
                                \lstick{$\ket{-}$}
                                & \qw & \gate{\, f \,} & \qw & \qw 
                        \end{quantikz}
        \end{center} 
\end{frame}

\begin{frame}{The Computation}

       {\scriptsize
       \textbf{N.B.} In order to not overburden notation we omit $\ket{-}$ }
       \begin{flalign*}
       & \, H^{\otimes n} \ket{0} & \\
       & = \textstyle{\frac{1}{\sqrt{2^n}}} \sum_{z \in 2^n}  \ket{z} & 
       \text{\{Theorem slide 18\}} \\
       & \stackrel{U_f}{\mapsto} \textstyle{\frac{1}{\sqrt{2^n}}} 
               \sum_{z \in 2^n} (-1)^{f(z)} \ket{z} & \text{\{Definition slide 12\}} \\
       & \stackrel{H^{\otimes n}}{\mapsto} \textstyle{\frac{1}{2^n}} 
        \sum_{z \in 2^n} (-1)^{f(z)} \Big ( \sum_{z' \in 2^n} (-1)^{z \cdot z'} \ket{z'} 
       \Big ) &  
       \text{\{Theorem slide 18\}} 
       \end{flalign*}
       We then proceed by case distinction. Assume that $f$ is constant
       \begin{flalign*}
               & \, \textstyle{\frac{1}{2^n}} 
               \sum_{z \in 2^n} (-1)^{f(z)} 
               \Big ( \sum_{z' \in 2^n} (-1)^{z \cdot z'} \ket{z'} \Big ) & \\ 
               & =  
               \textstyle{\frac{1}{2^n}} \alert{(\pm 1)}
               \sum_{z \in 2^n}  
               \Big ( \sum_{z' \in 2^n} (-1)^{z \cdot z'} \ket{z'} \Big ) &
       \end{flalign*}
\end{frame}

\begin{frame}{The Computation pt. II}
        Probability of measuring $\ket{0}$ at the end given by
        \begin{flalign*}
                & \, \big \lvert \textstyle{\frac{1}{2^n}} (\pm 1)
               \sum_{z \in 2^n}  
               (-1)^{z \cdot 0} \ket{0} \big \rvert^2
                & \\
                & = \big \lvert \textstyle{\frac{1}{2^n}} (\pm 1)
                \sum_{z \in 2^n}  
                1 \ket{0} \big \rvert^2
                & \\
                & = \big \lvert \textstyle{\frac{2^n}{2^n}} 
                \big \rvert^2
                &  \\
                & = 1
        \end{flalign*}

        So if $f$ is constant we measure $\ket{0}$ with probability $1$. Now if
        $f$ is balanced\dots
\end{frame}

\begin{frame}{The Computation pt. III}
        \begin{flalign*}
               &\, \textstyle{\frac{1}{2^n}} \sum_{z \in 2^n} (-1)^{f(z)} \Big ( \sum_{z'
                \in 2^n} (-1)^{z \cdot z'} \ket{z'} \Big ) &  \\
               & = \textstyle{\frac{1}{2^n}} \bigg (\sum_{z \in 2^n, \alert{f(z) = 0}} 
               (-1)^{f(z)} \Big ( \sum_{z' \in 2^n} (-1)^{z \cdot z'} \ket{z'} \Big ) & \\
               & \qquad +
               \textstyle{\sum_{z \in 2^n, \alert{f(z) = 1}}}
               (-1)^{f(z)} \Big ( \sum_{z'
               \in 2^n} (-1)^{z \cdot z'} \ket{z'} \Big )  \bigg ) \\
               & = \textstyle{\frac{1}{2^n}} \bigg (\sum_{z \in 2^n, f(z) = 0} 
               \Big ( \sum_{z' \in 2^n} (-1)^{z \cdot z'} \ket{z'} \Big ) & \\
               & \qquad +
               \textstyle{\sum_{z \in 2^n, f(z) = 1}}
               (-1) \Big ( \sum_{z'
               \in 2^n} (-1)^{z \cdot z'} \ket{z'} \Big )  \bigg ) 
        \end{flalign*}
\end{frame}
\begin{frame}{The Computation pt. IV}
        Probability of measuring $\ket{0}$ at the end given by
        \begin{flalign*}
                & \, \Big \lvert \textstyle{\frac{1}{2^n}} \Big (\sum_{z \in 2^n, f(z) = 0} 
               (-1)^{z \cdot 0} \ket{0} +
               \textstyle{\sum_{z \in 2^n, f(z) = 1}}
               (-1)(-1)^{z \cdot 0} \ket{0} \Big ) \Big \rvert^2
               & \\
               & = \Big \lvert \textstyle{\frac{1}{2^n}} \Big (\sum_{z \in 2^n, f(z) = 0} 
               \ket{0} +
               \textstyle{\sum_{z \in 2^n, f(z) = 1}}
               (-1)\ket{0} \Big ) \Big \rvert^2 & \\
               & = \Big \lvert \textstyle{\frac{1}{2^n}} \Big (\sum_{z \in 2^n, f(z) = 0} 
               \ket{0} -
               \textstyle{\sum_{z \in 2^n, f(z) = 1}}
               \ket{0} \Big ) \Big \rvert^2 & \\
                & = 0
        \end{flalign*}
        So if $f$ is balanced we measure $\ket{0}$ with probability $0$
\end{frame}

\section{Simon's problem}

\begin{frame}{Revisiting Deutsch-Josza}
        \begin{block}{The Problem}
                Take a function $f : \{0,1\}^{\alert{n}} \to \{0,1\}$.
                The latter  either constant or balanced

                Find out which case holds
        \end{block}
        
        Classically, evaluate half of the inputs ($\frac{2^{n}}{2} = \alert{2^{n-1}}$),
        evaluate one more and run the decision procedure,
        \begin{itemize}
                \item output always the same $\Longrightarrow$ constant
                \item otherwise $\Longrightarrow$ balanced
        \end{itemize}

        Quantumly, we know the answer by running $f$ only \alert{once}

        \pause
        However \dots
\end{frame}

\begin{frame}{Tackling Deutsch-Josza with Probabilities}

        To solve the problem with \alert{some margin of error} evaluate
        two \alert{arbitrary} inputs $x$ and $y$,
        \begin{itemize}
                \item $f(x) = f(y) \Longrightarrow$ constant
                \item $f(x) \not = f(y) \Longrightarrow$ balanced
        \end{itemize}
        \pause
        Probability of giving the right answer?

        \pause
        \begin{itemize}
                \item $f$ is constant $\Longrightarrow$ right answer
                        with probability $1$ 
                \item $f$ is balanced $\Longrightarrow$ right answer with probability
                        $\frac{2^{n-1}}{2^n} = \alert{\frac{1}{2}}$
        \end{itemize}

        \pause
        Can we do better?
\end{frame}

\begin{frame}{Tackling Deutsch-Josza with Probabilities pt. II}

        To solve the problem with \alert{some margin of error} evaluate
        $k$ arbitrary inputs $x_1,\dots,x_k$,
        \begin{itemize}
                \item output always the same  $\Longrightarrow$ constant
                \item otherwise $\Longrightarrow$ balanced
        \end{itemize}
        \pause
        Probability of giving the right answer?

        \pause
        \begin{itemize}
                \item $f$ is constant $\Longrightarrow$ right answer
                        with probability $1$ 
                \item $f$ is balanced $\Longrightarrow$ right answer
                        with probability \dots
                        \[
                                \textstyle{ 1-  
                                \tikzmark{r1} \Big (\frac{2^{n-1}}{2^n} \Big)^k \tikzmark{r2}
                                = \alert{1 - \frac{1}{2^k}}
                                }
                        \]
        \end{itemize}


        \begin{tikzpicture}[overlay,remember picture,
        box/.style = {rounded corners},
        pin edge={-Stealth,thick, red}]
        \coordinate (r1) at ($({pic cs:r1})+(+0.5ex, 1.5ex)$);
        \coordinate (r2) at ($({pic cs:r2})+(-0.5ex,-0.5ex)$);
        \node[semitransparent, 
             fit=(r1) (r2),
             pin=below:\tiny{Probability of observing the same output in $k$ tries}]  {};
       \end{tikzpicture}
\end{frame}

\begin{frame}{Simon}
        \begin{block}{The Problem}
                Take a \alert{2-1} function $f : \{0,1\}^{\alert{n}} \to \{0,1\}^{\alert{n}}$

                There exists a string $s \in \{0,1\}^n$
                s.t.  $f(x) = f(y) \Rightarrow y = x \oplus s$

                Find out $s$
        \end{block}

        Classically, evaluate inputs until collision is detected, \ie\
        $f(x) = f(y)$ for some $x,y$. Then compute $x \oplus y =
        x \oplus (x \oplus s) = s$

        Since $f$ is 2-1, after collecting $2^{n-1}$ evaluations with no
        collisions, next evaluation must cause a collision

        So in the worst case we need $\alert{2^{n-1} + 1}$ evaluations
\end{frame}

\begin{frame}{Tackling Simon with Probabilities}

        How many evaluations do we need to have a collision
        with probability $p$?

        \pause
        To have a collision with probability $p =
        \frac{1}{k} \leq \frac{1}{2}$ we need  
        \[
                \approx \tikzmark{z1} \sqrt{(2 \cdot 2^n) \cdot p} \tikzmark{z2} = 
                \textstyle{\sqrt{\frac{2}{k} \cdot 2^n} }
                = \sqrt{\frac{2}{k}} \cdot \alert{2^{\frac{n}{2}}}
                \quad \text{evaluations}
        \]

        \begin{tikzpicture}[overlay,remember picture,
        box/.style = {rounded corners},
        pin edge={-Stealth,thick, red}]
        \coordinate (z1) at ($({pic cs:z1})+(+0.5ex, 1.5ex)$);
        \coordinate (z2) at ($({pic cs:z2})+(-0.5ex,-0.5ex)$);
        \node[semitransparent, 
              fit=(z1) (z2),
              pin=below:\tiny{See the Birthday's problem}]  {};
        \end{tikzpicture}

        \pause
        But quantumly, we solve the problem in \alert{\underline{polynomial time}} with
        probability $\approx \textstyle{\frac{1}{4}}$
\end{frame}

\begin{frame}{Simon's Algorithm: The Key Steps}
       \begin{enumerate}
                \item Prepare superposition $\frac{1}{\sqrt{2}}(\ket{x} +
                        \ket{x \oplus s})$ for some string $x$
                \item Use \alert{\underline{interference}} to extract a string $y$ s.t.
                        $y \cdot s = 0$
                \item Repeat previous steps $\alert{n-1}$ times to obtain system of 
                        equations s.t. $y_k \cdot s = 0$
                \item Solve the system for $s$ using 
                        \tikzmark{a1}Gaussian elimination\tikzmark{a2}
       \end{enumerate}
       \begin{tikzpicture}[overlay,remember picture,
       box/.style = {rounded corners},
       pin edge={-Stealth,thick, red}]
       \coordinate (a1) at ($({pic cs:a1})+(+0.5ex, 1.5ex)$);
       \coordinate (a2) at ($({pic cs:a2})+(-0.5ex,-0.5ex)$);
       \node[semitransparent, 
             fit=(a1) (a2),
             pin=below:\tiny{Complexity $n^3$}]  {};
       \end{tikzpicture}
\end{frame}

\begin{frame}{Simon's Algorithm: Preparing the Superposition}
        \begin{center}
                        \begin{quantikz}[transparent]
                                \lstick{\ket{0}} &[4mm] \gate{H^{\otimes n}} \qwbundle{n}
                                & \gate[wires=2]{U_f}  
                                & \qw 
                                & \qw \\
                                \lstick{$\mathbf{\ket{0}}$}
                                & \qwbundle{n} \qw & & \meter{} & \qw 
                        \end{quantikz}

                        {\tiny
                        \textbf{N.B.} $U_f \ket{x} \ket{y} = \ket{x} \ket{ y \oplus f(x) }$}
        \end{center} 
        \begin{flalign*}
        & \, U_f (H^{\otimes n} \otimes I) \ket{0}\mathbf{\ket{0}} & \\
        & = U_f \Big ( 
                \textstyle{\frac{1}{\sqrt{2^n}}} \textstyle{\sum_{x \in 2^n}} \ket{x} 
                \mathbf{\ket{0}}
        \Big ) & \\
        & = \textstyle{\frac{1}{\sqrt{2^n}}} \textstyle{\sum_{x \in 2^n}} \ket{x} \ket{f(x)} 
        &
        \end{flalign*}
        We then measure the $n$-bottom qubits to obtain a superposition
        \[
               \textstyle{\frac{1}{\sqrt{2}}} (\ket{x} + \ket{x \oplus s})
        \]
\end{frame}

\begin{frame}{Simon's Algorithm: Extracting the String}
        \begin{center}
                        \begin{quantikz}[transparent]
                                \lstick{\ket{\psi}} &[4mm] \gate{H^{\otimes n}} \qwbundle{n}
                                & \qw 
                                & \qw \qwbundle{n}
                        \end{quantikz}
        \end{center} 
        \pause
        \begin{flalign*}
        & \, H^{\otimes n} \textstyle{\frac{1}{\sqrt{2}}} (\ket{x} + \ket{x \oplus s}) & \\
        & = \textstyle{\frac{1}{\sqrt{2^{n+1}}}} 
        \sum_{y \in 2^n} (-1)^{x \cdot y} \ket{y} + (-1)^{(x \oplus s) \cdot y} \ket{y} 
        &\text{\{Theorem slide 18\}}\\
        & =  \textstyle{\frac{1}{\sqrt{2^{n+1}}}} 
        \sum_{y \in 2^n} (-1)^{x \cdot y} \ket{y} + (-1)^{x \cdot y \oplus  s \cdot y} \ket{y} 
        &\text{\{Lemma slide 17\}} \\
        & =  \textstyle{\frac{1}{\sqrt{2^{n+1}}}} 
        \sum_{y \in 2^n} (-1)^{x \cdot y} \ket{y} + (-1)^{x \cdot y} 
        (-1)^{s \cdot y} \ket{y} 
        &\text{\{Lemma slide 16\}} \\
        & =  \textstyle{\frac{1}{\sqrt{2^{n+1}}}} 
        \sum_{y \in 2^n} (-1)^{x \cdot y} (\alert{1 +(-1)^{s \cdot y}}) \ket{y} 
        &
        \end{flalign*}

        \pause
        Destructive interference when $s \cdot y = 1$. We only observe
        $\ket{y}$ s.t. $s \cdot y = 0$
\end{frame}

\begin{frame}{The Circuit}
        \begin{center}
                        \begin{quantikz}[transparent]
                                \lstick{\ket{0}} &[4mm] \gate{H^{\otimes n}} \qwbundle{n}                                            \gategroup[wires=2,steps=3,style={dashed,
                                rounded corners,fill=blue!10, inner xsep=0.2pt},background]
                                {\tiny{state preparation}} 
                                &[4mm] \gate[wires=2]{U_f}  
                                &[4mm] \qw 
                                &[4mm] \gate{H^{\otimes n}} \qw 
                                \gategroup[wires=1,steps=1,style={dashed,
                                rounded corners,fill=blue!10, inner xsep=0.2pt},background]
                                {\tiny{string extraction by intro. of intrf.}} 
                                & \qw \\
                                \lstick{$\mathbf{\ket{0}}$}
                                & \qwbundle{n} \qw & & \meter{} & \qw & \qw
                        \end{quantikz}
        \end{center} 
\end{frame}

\begin{frame}{Simon's Algorithm: Solving the System to Extract $s$}

        A system of $n-1$ \alert{linearly independent} equations,
        \[
                \begin{cases}
                        y_1 \cdot s = 0 \\
                        \dots \\
                        y_{n-1} \cdot s = 0
                \end{cases}
        \]
        has two solutions.  One is $s = 0$ but it violates the 2-1
        promise. So only the other solution is of interest

        \pause
        Probability of obtaining such a system of equations
        by running the circuit $n-1$ times?
\end{frame}

\begin{frame}{Simon's Algorithm: Probability of Success}

        \begin{block}{Homework}
                If $s \not = 0$ then for half of the inputs $y$ we
                have $y \cdot s = 0$ and for the other half $y \cdot s = 1$ 
        \end{block}

        \vfill
        \begin{tabular}{|c | c | c| }
                \hline 
                \# & Possibilities of failure at each step &  Probability of failure 
                \\
                \hline
                $1$ & $\{0\}$ & $\frac{2^0}{2^{n-1}}$ \\
                \hline 
                $2$ & $\{0,y_1\}$ & $\frac{2^1}{2^{n-1}}$ \\
                \hline 
                $3$ & $\{0,y_1,y_2,y_1 \oplus y_2\}$ & $\frac{2^2}{2^{n-1}}$  \\
                \hline
                \dots & \dots & \dots \\
                \hline
                $n-1$ &  $\{0,y_1,y_2, y_3 \dots \}$ &  $\frac{2^{n-2}}{2^{n-1}}$ \\
                \hline
        \end{tabular}
\end{frame}

\begin{frame}{Simon's Algorithm: Probability of Success}
        \begin{center}
        \scalebox{0.7}{
        \begin{tabular}{|c | c | c| }
                \hline 
                \# & Possibilities of failure at each step &  Probability of failure 
                \\
                \hline
                $1$ & $\{0\}$ & $\frac{2^0}{2^{n-1}}$ \\
                \hline 
                $2$ & $\{0,y_1\}$ & $\frac{2^1}{2^{n-1}}$ \\
                \hline 
                $3$ & $\{0,y_1,y_2,y_1 \oplus y_2\}$ & $\frac{2^2}{2^{n-1}}$  \\
                \hline
                \dots & \dots & \dots \\
                \hline
                $n-1$ &  $\{0,y_1,y_2, y_3 \dots \}$ &  $\frac{2^{n-2}}{2^{n-1}}$ \\
                \hline
        \end{tabular}
        }
        \end{center}

        Table yields the sequence of probabilities of failure,
        \[
                \textstyle{
                        \frac{1}{2}, \frac{1}{4}, \frac{1}{8}, \dots, \frac{1}{2^{n-1}}
                }
                \qquad
                \text{ (from bottom to top) }
        \]
        Probability of failing in the first $n-2$ steps is thus 
        \[
                \textstyle{
                \frac{1}{4} + \frac{1}{8} + \dots = 
                \frac{1}{4} \Big (1 + \frac{1}{2} + \dots \Big ) \alert{\leq}
                \frac{1}{4} \cdot \left ( \tikzmark{s1} \sum_{i \in \mathbb{N}} \frac{1}{2^i}
                \tikzmark{s2} \right )
                = \frac{1}{2}
                }
        \]
        \begin{tikzpicture}[overlay,remember picture,
        box/.style = {rounded corners},
        pin edge={-Stealth,thick, red}]
        \coordinate (s1) at ($({pic cs:s1})+(+0.5ex, 1.5ex)$);
        \coordinate (s2) at ($({pic cs:s2})+(-0.5ex,-0.5ex)$);
        \node[semitransparent, 
             fit=(s1) (s2),
             pin=below:\tiny{Geometric series whose sum is equal to two}]  {};
       \end{tikzpicture}

\end{frame}

\begin{frame}{Simon's Algorithm: Probability of Success}

       Probability of succeeding in the first $n-2$ steps at least $\frac{1}{2}$

       Probability of succeeding in the $(n-1)$-th step is $\frac{1}{2}$

       Thus probability of succeeding in all $n-1$ steps at least $\alert{\frac{1}{4}}$

       \vspace{0.5cm}
       \pause
       More advanced maths tell that the probability is slightly higher
       (around $0.28878\dots$)
\end{frame}

\section{Conclusions}

\begin{frame}{What Have We Learned?}

        \alert{Exponential separation} between classical and
        quantum\dots even if probabilities are involved

        \pause
        Always looking for a \alert{global} property of $f$; not a
        local one

        \pause
        Superposition and interference were instrumental 

        \vspace{0.5cm}
        \pause
        Problems solved were somewhat contrived. In the next lectures we
        will analyse problems with broader applications
\end{frame}


\end{document}
