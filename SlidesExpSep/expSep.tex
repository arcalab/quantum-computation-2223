\documentclass{beamer}
\usepackage{etex} % fixes new-dimension error
\usepackage{lmodern}
\usepackage[T1]{fontenc}
\usepackage{mathtools}
%%%%%%%%%%%%% Macros
%%%% Categories
\newcommand{\catfont}[1]{\mathsf{#1}}
\newcommand{\cop}{\catfont{op}}
\newcommand{\Law}{\catfont{Law}}
\newcommand{\catV}{\catfont{V}}
\newcommand{\catX}{\catfont{X}}
\newcommand{\catC}{\catfont{C}}
\newcommand{\catD}{\catfont{D}}
\newcommand{\catA}{\catfont{A}}
\newcommand{\catB}{\catfont{B}}
\newcommand{\catI}{\catfont{I}}
\newcommand{\Set}{\catfont{Set}}
\newcommand{\Top}{\catfont{Top}}
\newcommand{\Pos}{\catfont{Pos}}
\newcommand{\Inj}{\catfont{Inj}}
\newcommand{\Det}{\catfont{RMhat}}
\newcommand{\CoAlg}[1]{\catfont{CoAlg}\left (#1 \right )}
\newcommand{\Mon}{\catfont{Mon}}
\newcommand{\Mnd}{\catfont{Mnd}(\catC)}
\newcommand{\SMnd}{\catfont{Mnd}(\Set)}
\newcommand{\CLat}{\catfont{CLat}}
\newcommand{\Stone}{\catfont{Stone}}
\newcommand{\Spectral}{\catfont{Spectral}}
\newcommand{\CompHaus}{\catfont{CompHaus}}
\newcommand{\Subs}[2]{\catfont{Sub}_{}}
\newcommand{\Cone}{\catfont{Cone}}
\newcommand{\StComp}{\catfont{StablyComp}}
\newcommand{\PosC}{\catfont{PosComp}}
\newcommand{\Haus}{\catfont{Haus}}
\newcommand{\Meas}{\catfont{Meas}}
\newcommand{\Ord}{\catfont{Ord}}
\newcommand{\EndoC}{[\catC,\catC]}
%% General functors
\newcommand{\funfont}[1]{#1}
\newcommand{\funF}{\funfont{F}}
\newcommand{\funU}{\funfont{U}}
\newcommand{\funG}{\funfont{G}}
\newcommand{\funT}{\funfont{T}}
\newcommand{\funI}{\funfont{I}}
%% Particular kinds of functors
\newcommand{\sfunfont}[1]{\mathrm{#1}}
\newcommand{\Pow}{\sfunfont{P}}
\newcommand{\Dist}{\sfunfont{D}}
\newcommand{\Maybe}{\sfunfont{M}}
\newcommand{\List}{\sfunfont{L}}
\newcommand{\UForg}{\sfunfont{U}}
\newcommand{\Forg}[1]{\sfunfont{U}_{#1}}
\newcommand{\Id}{\sfunfont{Id}}
\newcommand{\Vie}{\sfunfont{V}}
\newcommand{\Disc}{\funfont{D}}
\newcommand{\Weight}{\sfunfont{W}}
\newcommand{\homf}{\sfunfont{hom}}
\newcommand{\Yoneda}{\sfunfont{Y}}
%% Diagram functors
\newcommand{\Diag}{\mathscr{D}}
\newcommand{\KDiag}{\mathscr{K}}
\newcommand{\LDiag}{\mathscr{L}}
%% Monads
\newcommand{\monadfont}[1]{\mathbb{#1}}
\newcommand{\monadT}{\monadfont{T}}
\newcommand{\monadS}{\monadfont{S}}
\newcommand{\monadU}{\monadfont{U}}
\newcommand{\monadH}{\monadfont{H}}
\newcommand{\str}{\mathrm{str}}
%% Adjunctions
\newcommand\adjunct[2]{\xymatrix@=8ex{\ar@{}[r]|{\top}\ar@<1mm>@/^2mm/[r]^{{#2}}
& \ar@<1mm>@/^2mm/[l]^{{#1}}}}
\newcommand\adjunctop[2]{\xymatrix@=8ex{\ar@{}[r]|{\bot}\ar@<1mm>@/^2mm/[r]^{{#2}}
& \ar@<1mm>@/^2mm/[l]^{{#1}}}}
%% Retractions
\newcommand\retract[2]{\xymatrix@=8ex{\ar@{}[r]|{}\ar@<1mm>@/^2mm/@{^{(}->}[r]^{{#2}}
& \ar@<1mm>@/^2mm/@{->>}[l]^{{#1}}}}
%% Limits
\newcommand{\pv}[2]{\langle #1, #2 \rangle}
\newcommand{\limt}{\mathrm{lim}}
\newcommand{\pullbackcorner}[1][dr]{\save*!/#1+1.2pc/#1:(1,-1)@^{|-}\restore}
\newcommand{\pushoutcorner}[1][dr]{\save*!/#1-1.2pc/#1:(-1,1)@^{|-}\restore}
%% Colimits
\newcommand{\colim}{\mathrm{colim}}
\newcommand{\inl}{\mathrm{inl}}
\newcommand{\inr}{\mathrm{inr}}
%% Distributive categories
\newcommand{\distr}{\mathrm{dist}}
\newcommand{\undistr}{\mathrm{undist}}
%% Closedness
\newcommand{\curry}[1]{\mathrm{curry}{#1}}
\newcommand{\app}{\mathrm{app}}
%% Misc. operations
\newcommand{\const}[1]{\underline{#1}}
\newcommand{\comp}{\cdot}
\newcommand{\id}{\mathrm{id}}
%% Factorisations
\newcommand{\EClass}{E}
\newcommand{\MClass}{M}
\newcommand{\MConeClass}{\mathcal{M}}
%%%%%%%%%%%%%%%% End of Categorical Stuff

%%%% Misc
%% Operations
\newcommand{\blank}{\, - \,}
\newcommand{\sem}[1]{\llbracket #1 \rrbracket}
\newcommand{\closure}[1]{\overline{#1}}
\DeclareMathOperator{\img}{\mathrm{im}}
\DeclareMathOperator{\dom}{\mathrm{dom}}
\DeclareMathOperator{\codom}{\mathrm{codom}}
%% Sets of numbers
\newcommand{\Nats}{\mathbb{N}}
\newcommand{\Reals}{\mathbb{R}}
\newcommand{\Rz}{\Reals_{\geq 0}}
%% Writing
\newcommand{\cf}{\emph{cf.}}
\newcommand{\ie}{\emph{i.e.}}
\newcommand{\eg}{\emph{e.g.}}
\newcommand{\df}[1]{\emph{\textbf{#1}}}
%%%%%%%%%%%%%%%% End of Misc

%%%% Programming Stuff
%% Types
\newcommand{\typefont}[1]{\mathbb{#1}}
\newcommand{\typeOne}{1}
\newcommand{\typeTwo}{2}
\newcommand{\typeA}{\typefont{A}}
\newcommand{\typeB}{\typefont{B}}
\newcommand{\typeC}{\typefont{C}}
\newcommand{\typeV}{\typefont{V}}
\newcommand{\typeD}{\typefont{D}}
%% RuleName
\newcommand{\rulename}[1]{(\mathrm{#1})}
%% Sequents
\newcommand{\jud}{\vdash}
\newcommand{\vljud}{\vdash}
\newcommand{\cojud}{\vdash_{\co}}
\newcommand{\vl}{\mathtt{v}}
\newcommand{\co}{\mathtt{c}}
% Program font
\newcommand{\prog}[1]{\mathtt{#1}}
\newcommand{\pseq}[3]{#1 \leftarrow #2; #3}
\newcommand{\ppm}[4]{(#1,#2) \leftarrow #3; #4}
\newcommand{\pinl}[1]{\prog{inl}(#1)}
\newcommand{\pinr}[1]{\prog{inr}(#1)}
\newcommand{\pcase}[4]{\prog{ case } #1 \prog{ of } \pinl{#2} \Rightarrow #3 ; \pinr{#2} \Rightarrow #4}
%% Sets of terms
\newcommand{\ValuesBP}[2]{\mathsf{Values}(#1, #2)}
\newcommand{\TermsBP}[2]{\mathsf{Terms}(#1, #2)}
\newcommand{\closValP}[1]{\ValuesBP{\emptyset}{#1}}
\newcommand{\closTermP}[1]{\TermsBP{\emptyset}{#1}}
\newcommand{\closVal}{\closValP{\typeA}}
\newcommand{\closTerm}{\closTermP{\typeA}}
%% Contextual equivalence
\newcommand{\ctxeq}{\equiv_{\prog{ctx}}}
%%%% End of Programming Stuff

%-------------- template --------------------------------------------------
\usetheme{metropolis}
\metroset{block=fill}
%\usetheme{Boadilla}
% Configuring the foot line
\setbeamertemplate{footline}
{
  \leavevmode%
  \hbox{%
  \begin{beamercolorbox}[wd=.4\paperwidth,ht=2.25ex,dp=1ex,center]{author in head/foot}%
    \usebeamerfont{author in head/foot}\insertshortauthor
  \end{beamercolorbox}%
  \begin{beamercolorbox}[wd=.5\paperwidth,ht=2.25ex,dp=1ex,center]{title in head/foot}%
    \usebeamerfont{title in head/foot}\insertsection
  \end{beamercolorbox}%
  \begin{beamercolorbox}[wd=.1\paperwidth,ht=2.25ex,dp=1ex,right]{date in head/foot}%
    \insertframenumber{} / \inserttotalframenumber\hspace*{2ex} 
  \end{beamercolorbox}}%
  \vskip0pt%
}
% No configuration symbols
\setbeamertemplate{navigation symbols}{}
%--------------- packages -------------------------------------------------
\usepackage{graphicx,amsmath}
\usepackage{stmaryrd} % cf. interleave
\usepackage{booktabs}
\usepackage{amscd}
\usepackage{multicol}
\usepackage[absolute,overlay]{textpos}
\usepackage{alltt}
\usepackage{proof}
\usepackage{subcaption}
\usepackage{tikz}
\usepackage{tikz-cd}
\usepackage{quantikz}
\usepackage[new]{old-arrows}
\usepackage[all]{xy}
\usepackage{pgfplots}
\usepackage{textcomp}
\usetikzlibrary{arrows.meta, calc, fit, tikzmark}
\usepackage{pstricks,pst-node,pst-text,pst-3d}
% context
\AtBeginSection[]
{
    \begin{frame}
        \frametitle{Table of Contents}
        \tableofcontents[currentsection]
    \end{frame}
}

\author[Renato Neves]{Renato Neves}

% logos of institutions
\titlegraphic{
  \begin{textblock*}{5cm}(6.7cm,7.45cm)
     \includegraphics[scale=0.06]{logos/uminho.png}\hspace*{.85cm}~%
  \end{textblock*}
  \begin{textblock*}{5cm}(9.4cm,7.42cm)
    \includegraphics[scale=0.50]{logos/haslab.pdf}
  \end{textblock*}
}

% No date
\date{}

\begin{document}

\title{Exponential Separation between Quantum and Classical Computation}

\frame[plain]{\titlepage}

\section{Overview}

\begin{frame}{Previously\dots}

        \begin{block}{The Problem}
        Take a function $f : \{0,1\} \to \{0,1\}$

        Either $f(0) = f(1)$ or $f(0) \not = f(1)$

        Tell us whether the first or second case hold
        \end{block}

        Classically, need to run $f$ \alert{twice}.  Quantumly, it suffices to
        run \alert{once}

        \pause
        Can we have a more impressive difference in complexities?

\end{frame}

\section{Global and local phases}

\begin{frame}{Global Phase Factor}

        We say that vector $e^{i\theta}v$ is equal to $v$
        up to \alert{global phase factor} $e^{i \theta}$

        \vfill
        \begin{theorem}
         $e^{i\theta}v$ and $v$ are indistinguishable
         in the world of quantum mechanics
        \end{theorem}

        \begin{block}{Proof sketch}
                Show that equality up to global phase is preserved by operators
                and normalisation + show that probability outcomes associated
                with $v$ and $e^{i\theta}v$ are the same
        \end{block}
\end{frame}

\begin{frame}{Relative Phase Factor}
        
        We say that vectors $\sum_{x \in 2^n} \alpha_x \ket{x}$ and $\sum_{x
        \in 2^n} \beta_x \ket{x}$ differ by a \alert{relative phase factor} if
        for all $x \in 2^n$
        \[
                \alpha_x = e^{i \theta_x} \beta_x
                \qquad \text{(for some angle $\theta_x$)}
        \]

        \begin{example}
                Vectors $\ket{0} + \ket{1}$ and $\ket{0} - \ket{1}$
                differ by a relative phase factor
        \end{example}

        \pause
        \vfill
        Vectors that differ by a relative phase factor are distinguishable
\end{frame}

\section{Phase Kickback}

\begin{frame}{The Phase Kickback Effect pt. I}
        Recall that every quantum operation
        \begin{quantikz}
               &[4mm] \gate{U} \qwbundle{n} &[4mm] \qw \qwbundle{n}
        \end{quantikz}$\,$
        gives rise to a `controlled' quantum operation
        \[ 
                \begin{quantikz}
                &[4mm] \ctrl{1} &[4mm] \qw  \\
                &[4mm] \gate{U} \qwbundle{n} &[4mm] \qw \qwbundle{n}
                \end{quantikz}
        \]

        Then let $v$ be an \alert{eigenvector} of $U$, \ie\ $U v = e^{i \theta} v$,
        and calculate
        \begin{flalign*}
              &  \, \, cU ( (\alpha \ket{0} + \beta \ket{1}) \otimes v) & \\
              & = cU (\alpha \ket{0} \otimes v + \beta \ket{1} \otimes v) & \\
              & = \alpha \ket{0} \otimes v + \beta \ket{1} \otimes e^{i\theta} v & \\
              & = (\alpha \ket{0} + \alert{e^{i\theta}} \beta \ket{1}) \otimes v 
        \end{flalign*}
\end{frame}
\begin{frame}{The Phase Kickback Effect pt. II}
        What just happened?
        \pause
        \begin{itemize}
                \item Global phase $e^{i \theta}$ introduced to $v$
                        was 'kickedback' as a relative phase in the control
                        qubit
                        \pause
                \item Some information of $U$ is now encoded in the control
                        qubit
                        \pause
                \item In general, the introduction of such phases causes
                        interference patterns that give away information 
                        about $U$
        \end{itemize}
\end{frame}

\begin{frame}{The Phase Kickback Effect pt. III}
        Consider the circuit:

        \[
               \begin{quantikz}
                & \ctrl{1} & \qw  \\
                & \targ{} & \qw 
                \end{quantikz}
        \]

        What is the unit vector $v$ that satisfies
        the equation below?
        \[
                cX \ket{b} v = (-1)^{b} \ket{b} v
        \]
\end{frame}

\begin{frame}{Back to Deutsch's Problem}

        \begin{center}
                \hspace{-1.5cm}
                        \begin{quantikz}[transparent]
                                \lstick{\ket{0}} &\gate{H} 
                                \gategroup[wires=1,steps=1,style={dashed,
                                rounded corners,fill=blue!10, inner xsep=0.2pt},background]
                                {\tiny{parallelism}}
                                & \gate[wires=2]{U_f} \gategroup[wires=2,steps=1,style={dashed,
                                rounded corners,fill=blue!10, inner xsep=0.2pt}, 
                                label style={label position = 
                                below, yshift=-0.4cm},background]
                                {\tiny{interference pattern}}
                                & \gate{H} 
                                \gategroup[wires=1,steps=1,style={dashed,
                                rounded corners,fill=blue!10, inner xsep=0.2pt},background]{
                                \tiny{wave collapse}}
                                & \qw \\
                                \lstick{$\ket{-}$}
                                                 & \qw & & \qw & \qw 
                        \end{quantikz}
        \end{center} 

        \vfill
        \pause
        $U_f$ can be seen as a \alert{generalised} controlled not-operation
        \[
                \left \sem{ 
               \begin{quantikz}
                & \ctrl{1} & \qw  \\
                & \gate{f} & \qw 
                \end{quantikz}
                \right } = 
                \ket{x}\ket{y} \mapsto 
                \begin{cases}
                        \ket{x} \ket{y} & \text{ if } f(x) = 0\\
                        \ket{y} \neg \ket{y} & \text{ if } f(x) = 1
                \end{cases}
        \]

\end{frame}

\begin{frame}{Back to Deutsch's Problem pt. II}

        $U_f$ can be seen as a \alert{generalised} controlled \alert{not}-operation
        \[
                \left \sem{ 
               \begin{quantikz}
                & \ctrl{1} & \qw  \\
                & \gate{f} & \qw 
                \end{quantikz}
                \right } = 
                \ket{x}\ket{y} \mapsto 
                \begin{cases}
                        \ket{x} \ket{y} & \text{ if } f(x) = 0\\
                        \ket{y} \neg \ket{y} & \text{ if } f(x) = 1
                \end{cases}
        \]

        $\ket{-}$ is an eigenvector of $X$ with associated eigenstate $-1$

        So analogously to $cX \ket{x} \ket{-} = (-1)^{x} \ket{x} \ket{-}$ we
        deduce that
        \[
                U_f \ket{x} \ket{-} = (-1)^{f(x)} \ket{x} \ket{-}
        \]
\end{frame}

\begin{frame}{Back to Deutsch's Problem pt. III}

        \begin{center}
                \hspace{-1.5cm}
                        \begin{quantikz}[transparent]
                                \lstick{\ket{0}} &\gate{H} 
                                \gategroup[wires=1,steps=1,style={dashed,
                                rounded corners,fill=blue!10, inner xsep=0.2pt},background]
                                {\tiny{parallelism}}
                                & \ctrl{1} \gategroup[wires=2,steps=1,style={dashed,
                                rounded corners,fill=blue!10, inner xsep=0.2pt}, 
                                label style={label position = 
                                below, yshift=-0.4cm},background]
                                {\tiny{interference pattern (created by phase kickback)}}
                                & \gate{H} 
                                \gategroup[wires=1,steps=1,style={dashed,
                                rounded corners,fill=blue!10, inner xsep=0.2pt},background]{
                                \tiny{wave collapse}}
                                & \qw \\
                                \lstick{$\ket{-}$}
                                & \qw & \gate{\, f \,} & \qw & \qw 
                        \end{quantikz}
        \end{center} 

\end{frame}

\section{Bernstein-Vazirani's problem}

\begin{frame}{Going further beyond\dots}

        Despite looking almost magical how we handled Deutsch's problem\dots

        corresponding complexity differences between quantum and classical
        cases is \alert{unimpressive}
        
        Can we come up with a more impressive separation?
\end{frame}

\begin{frame}{Setting the Stage}
        \begin{lemma}
                For $a,b \in \{0,1\}$ the equation $(-1)^a(-1)^b = (-1)^{a \oplus b}$
                holds
        \end{lemma}
        \begin{block}{Prook sketch}
                Build a truth table for each case and compare the corresponding
                contents
        \end{block}

        \vfill
        \begin{definition}
                Given two bit-strings $x,y \in \{0,1\}^n$ we define
                their product $x \cdot y \in \{0,1\}$ as
                        $x \cdot y = (x_1 \wedge y_1) \oplus \dots
                        \oplus (x_n \wedge y_n)$
        \end{definition}
\end{frame}

\begin{frame}{Setting the Stage}

        \begin{lemma}
                For any three binary strings $x,a,b \in \{0,1\}^n$ the equation
                $(x \cdot a) \oplus (x \cdot b) = x \cdot (a \oplus b)$ holds
        \end{lemma}

        \begin{block}{Proof sketch} 
                Follows from the fact that for any three bits $a,b,c \in
                \{0,1\}$ the equation $(a \wedge b) \oplus (a \wedge c) = a
                \wedge (b \oplus c)$ holds
        \end{block}
\end{frame}

\begin{frame}{Setting the Stage}
        \begin{lemma}
                For any element $\ket{b}$ in the computational basis of
                $\Complex^2$ we have $H \ket{b} =
                \textstyle{\frac{1}{\sqrt{2}}} \sum_{z \in 2} (-1)^{b \wedge z}
                \ket{z}$
        \end{lemma}
        \begin{block}{Proof sketch}
                Build a truth table and compare the corresponding contents
        \end{block}
        \vfill

        \begin{theorem}
                For any element $\ket{b}$ in the computational basis
                of $\Complex^{2^n}$ we have
                $H^{\otimes n} \ket{b} = 
                \textstyle{\frac{1}{\sqrt{2^n}}} \sum_{z \in 2^n} (-1)^{b \cdot z}
                \ket{z}$
        \end{theorem}
        \begin{block}{Proof sketch}
                Follows from induction on the size of $n$
        \end{block}
\end{frame}

\begin{frame}{Bernstein-Vazirani}

        \begin{block}{The Problem}
                Take a function $f : \{0,1\}^{\alert{n}} \to \{0,1\}$

                You are promised that $f(x) = s \cdot x$ for some fixed bit-string $s$

                Find $s$
        \end{block}

        Classically, we compute
        \begin{flalign*}
                f(10 \dots 0) & = (s_1 \wedge 1) \oplus \dots \oplus (s_n \wedge 0) 
                = \alert{s_1} \\
                f(01 \dots 0) & = (s_1 \wedge 0) \oplus (s_2 \wedge 1) \oplus \dots \oplus
                (s_n \wedge 0) = \alert{s_2} \\
                \dots & \\
                f(00 \dots 1) & = (s_1 \wedge 0) \oplus \dots \oplus 
                (s_n \wedge 1) = \alert{s_n}
        \end{flalign*}

        Quantumly, we can discover $s$ by running $f$ only \alert{once}
\end{frame}

\begin{frame}{The Circuit}
        \begin{center}
                \hspace{-1.5cm}
                        \begin{quantikz}[transparent]
                                \lstick{\ket{0}} &[4mm] \gate{H^{\otimes n}} 
                                \gategroup[wires=1,steps=1,style={dashed,
                                rounded corners,fill=blue!10, inner xsep=0.2pt},background]
                                {\tiny{parallelism}} \qwbundle{n}
                                & \ctrl{1} \gategroup[wires=2,steps=1,style={dashed,
                                rounded corners,fill=blue!10, inner xsep=0.2pt}, 
                                label style={label position = 
                                below, yshift=-0.4cm},background]
                                {\tiny{interference pattern (created by phase kickback)}}
                                & \gate{H^{\otimes n}} 
                                \gategroup[wires=1,steps=1,style={dashed,
                                rounded corners,fill=blue!10, inner xsep=0.2pt},background]{
                                \tiny{wave collapse}} 
                                &[4mm] \qw \qwbundle{n} \\
                                \lstick{$\ket{-}$}
                                & \qw & \gate{\, f \,} & \qw & \qw 
                        \end{quantikz}
        \end{center} 
\end{frame}

\begin{frame}{The Computation}
       In order to make computations simpler we will omit $\ket{-}$ 
       \begin{flalign*}
       & \, H^{\otimes n} \ket{0} & \\
       & = \textstyle{\frac{1}{\sqrt{2^n}}} \sum_{z \in 2^n}  \ket{z} & 
       \text{\{Theorem slide 18\}} \\
       & \stackrel{U_f}{\mapsto} \textstyle{\frac{1}{\sqrt{2^n}}} 
               \sum_{z \in 2^n} (-1)^{f(z)} \ket{z} & \text{\{Definition slide 12\}} \\
       & \stackrel{H^{\otimes n}}{\mapsto} \textstyle{\frac{1}{2^n}} 
        \sum_{z \in 2^n} (-1)^{f(z)} \Big ( \sum_{z' \in 2^n} (-1)^{z \cdot z'} \ket{z'} 
       \Big ) &  
       \text{\{Theorem slide 18\}} \\
       & = \textstyle{\frac{1}{2^n}} 
       \sum_{z \in 2^n} \sum_{z' \in 2^n} (-1)^{(z \cdot s) \oplus (z \cdot z')}  \ket{z'}
       & \text{\{Lemma slide 16\}} \\
       & = \textstyle{\frac{1}{2^n}} 
       \sum_{z \in 2^n} \sum_{z' \in 2^n} (-1)^{z \cdot (s \oplus z')} \ket{z'} 
       & \text{\{Lemma slide 17\}} 
       \end{flalign*}
\end{frame}

\begin{frame}{The computation pt. II}
        Probability of measuring $s$ at the end given by
        \begin{flalign*}
                & \, \big \lvert \textstyle{\frac{1}{2^n}} 
                \sum_{z \in 2^n} (-1)^{z \cdot (s \oplus s)} \ket{s} \big \rvert^2
                & \\
                & = \big \lvert \textstyle{\frac{1}{2^n}} 
                \sum_{z \in 2^n} (-1)^{z \cdot 0} \ket{s} \big \rvert^2
                & \\
                & = \big \lvert \textstyle{\frac{1}{2^n}} 
                \sum_{z \in 2^n} 1 \ket{s} \big \rvert^2
                &  \\
                & = \big \lvert \textstyle{\frac{2^n}{2^n}} 
                \big \rvert^2
                &  \\
                & = 1
        \end{flalign*}

        This entails that somehow all values yielding wrong answers
        were cancelled
\end{frame}

\begin{frame}{Going Further Beyond}

        We went from running $f$ $n$ times
        to running just once

        \pause
        Still not very impressive (at least for the Computer Scientist :-)) 

        \pause
        Can we do even better?
\end{frame}

\section{Deutsch-Josza's problem}

\begin{frame}{Deutsch-Josza}
        \begin{block}{The problem}
                Take a function $f : \{0,1\}^n \to \{0,1\}$

                You are promised that $f$ is either constant or balanced

                Find out which case holds
        \end{block}
        
        Classically, we evaluate half of the inputs ($\frac{2^{n}}{2} = 2^{n-1}$),
        evaluate one more and run the decision procedure,
        \begin{itemize}
                \item output always the same $\Longrightarrow$ constant
                \item otherwise $\Longrightarrow$ balanced
        \end{itemize}
        which requires running $f$  $2^{n-1} + 1$ times

        Quantumly, we know the answer by runing $f$ only once
\end{frame}

\begin{frame}{The Circuit}
        \begin{center}
                \hspace{-1.5cm}
                        \begin{quantikz}[transparent]
                                \lstick{\ket{0}} &[4mm] \gate{H^{\otimes n}} 
                                \gategroup[wires=1,steps=1,style={dashed,
                                rounded corners,fill=blue!10, inner xsep=0.2pt},background]
                                {\tiny{parallelism}} \qwbundle{n}
                                & \ctrl{1} \gategroup[wires=2,steps=1,style={dashed,
                                rounded corners,fill=blue!10, inner xsep=0.2pt}, 
                                label style={label position = 
                                below, yshift=-0.4cm},background]
                                {\tiny{interference pattern (created by phase kickback)}}
                                & \gate{H^{\otimes n}} 
                                \gategroup[wires=1,steps=1,style={dashed,
                                rounded corners,fill=blue!10, inner xsep=0.2pt},background]{
                                \tiny{wave collapse}} 
                                &[4mm] \qw \qwbundle{n} \\
                                \lstick{$\ket{-}$}
                                & \qw & \gate{\, f \,} & \qw & \qw 
                        \end{quantikz}
        \end{center} 
\end{frame}

\begin{frame}{The Computation}
       In order to make computations simpler we will omit $\ket{-}$ 
       \begin{flalign*}
       & \, H^{\otimes n} \ket{0} & \\
       & = \textstyle{\frac{1}{\sqrt{2^n}}} \sum_{z \in 2^n}  \ket{z} & 
       \text{\{Theorem slide 18\}} \\
       & \stackrel{U_f}{\mapsto} \textstyle{\frac{1}{\sqrt{2^n}}} 
               \sum_{z \in 2^n} (-1)^{f(z)} \ket{z} & \text{\{Definition slide 12\}} \\
       & \stackrel{H^{\otimes n}}{\mapsto} \textstyle{\frac{1}{2^n}} 
        \sum_{z \in 2^n} (-1)^{f(z)} \Big ( \sum_{z' \in 2^n} (-1)^{z \cdot z'} \ket{z'} 
       \Big ) &  
       \text{\{Theorem slide 18\}} 
       \end{flalign*}
       We then proceed by case distinction. Assume that $f$ is constant:
       \begin{flalign*}
               & \, \textstyle{\frac{1}{2^n}} 
               \sum_{z \in 2^n} (-1)^{f(z)} 
               \Big ( \sum_{z' \in 2^n} (-1)^{z \cdot z'} \ket{z'} \Big ) & \\ 
               & =  
               \textstyle{\frac{1}{2^n}} (\pm 1)
               \sum_{z \in 2^n}  
               \Big ( \sum_{z' \in 2^n} (-1)^{z \cdot z'} \ket{z'} \Big ) &
       \end{flalign*}
\end{frame}

\begin{frame}{The computation pt. II}
        Probability of measuring $\ket{0}$ at the end given by
        \begin{flalign*}
                & \, \big \lvert \textstyle{\frac{1}{2^n}} (\pm 1)
               \sum_{z \in 2^n}  
               (-1)^{z \cdot 0} \ket{0} \big \rvert^2
                & \\
                & = \big \lvert \textstyle{\frac{1}{2^n}} (\pm 1)
                \sum_{z \in 2^n}  
                1 \ket{0} \big \rvert^2
                & \\
                & = \big \lvert \textstyle{\frac{2^n}{2^n}} 
                \big \rvert^2
                &  \\
                & = 1
        \end{flalign*}

        So if $f$ is constant we measure $\ket{0}$ with probability $1$. Now if
        $f$ is balanced \dots
\end{frame}

\begin{frame}{The computation pt. III}
        \vspace{-0.8cm}
        \small{
        \begin{flalign*}
               &\, \textstyle{\frac{1}{2^n}} \sum_{z \in 2^n} (-1)^{f(z)} \Big ( \sum_{z'
                \in 2^n} (-1)^{z \cdot z'} \ket{z'} \Big ) &  \\
               & = \textstyle{\frac{1}{2^n}} \Big (\sum_{z \in 2^n, f(z) = 0} 
               (-1)^{f(z)} \Big ( \sum_{z' \in 2^n} (-1)^{z \cdot z'} \ket{z'} \Big ) & \\
               & \qquad +
               \textstyle{\sum_{z \in 2^n, f(z) = 1}}
               (-1)^{f(z)} \Big ( \sum_{z'
               \in 2^n} (-1)^{z \cdot z'} \ket{z'} \Big )  \Big ) \\
               & = \textstyle{\frac{1}{2^n}} \Big (\sum_{z \in 2^n, f(z) = 0} 
               \Big ( \sum_{z' \in 2^n} (-1)^{z \cdot z'} \ket{z'} \Big ) & \\
               & \qquad +
               \textstyle{\sum_{z \in 2^n, f(z) = 1}}
               (-1) \Big ( \sum_{z'
               \in 2^n} (-1)^{z \cdot z'} \ket{z'} \Big )  \Big ) 
        \end{flalign*}
        }
        Probability of measuring $\ket{0}$ at the end given by
        \small{
        \begin{flalign*}
                & \, \Big \lvert \textstyle{\frac{1}{2^n}} \Big (\sum_{z \in 2^n, f(z) = 0} 
               (-1)^{z \cdot 0} \ket{0} +
               \textstyle{\sum_{z \in 2^n, f(z) = 1}}
               (-1)(-1)^{z \cdot 0} \ket{0} \Big ) \big \rvert^2
               & \\
               & = \Big \lvert \textstyle{\frac{1}{2^n}} \Big (\sum_{z \in 2^n, f(z) = 0} 
               \ket{0} +
               \textstyle{\sum_{z \in 2^n, f(z) = 1}}
               (-1)\ket{0} \Big ) \big \rvert^2 & \\
               & = \Big \lvert \textstyle{\frac{1}{2^n}} \Big (\sum_{z \in 2^n, f(z) = 0} 
               \ket{0} -
               \textstyle{\sum_{z \in 2^n, f(z) = 1}}
               \ket{0} \Big ) \big \rvert^2 & \\
                & = 0
        \end{flalign*}
        }
        So if $f$ is balanced we measure $\ket{0}$ with probability $0$
\end{frame}

\end{document}
