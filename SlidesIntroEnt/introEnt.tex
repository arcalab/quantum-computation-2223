\documentclass{beamer}
\usepackage{etex} % fixes new-dimension error
\usepackage{lmodern}
\usepackage[T1]{fontenc}
\usepackage{mathtools}
%%%%%%%%%%%%% Macros
%%%% Categories
\newcommand{\catfont}[1]{\mathsf{#1}}
\newcommand{\cop}{\catfont{op}}
\newcommand{\Law}{\catfont{Law}}
\newcommand{\catV}{\catfont{V}}
\newcommand{\catX}{\catfont{X}}
\newcommand{\catC}{\catfont{C}}
\newcommand{\catD}{\catfont{D}}
\newcommand{\catA}{\catfont{A}}
\newcommand{\catB}{\catfont{B}}
\newcommand{\catI}{\catfont{I}}
\newcommand{\Set}{\catfont{Set}}
\newcommand{\Top}{\catfont{Top}}
\newcommand{\Pos}{\catfont{Pos}}
\newcommand{\Inj}{\catfont{Inj}}
\newcommand{\Det}{\catfont{RMhat}}
\newcommand{\CoAlg}[1]{\catfont{CoAlg}\left (#1 \right )}
\newcommand{\Mon}{\catfont{Mon}}
\newcommand{\Mnd}{\catfont{Mnd}(\catC)}
\newcommand{\SMnd}{\catfont{Mnd}(\Set)}
\newcommand{\CLat}{\catfont{CLat}}
\newcommand{\Stone}{\catfont{Stone}}
\newcommand{\Spectral}{\catfont{Spectral}}
\newcommand{\CompHaus}{\catfont{CompHaus}}
\newcommand{\Subs}[2]{\catfont{Sub}_{}}
\newcommand{\Cone}{\catfont{Cone}}
\newcommand{\StComp}{\catfont{StablyComp}}
\newcommand{\PosC}{\catfont{PosComp}}
\newcommand{\Haus}{\catfont{Haus}}
\newcommand{\Meas}{\catfont{Meas}}
\newcommand{\Ord}{\catfont{Ord}}
\newcommand{\EndoC}{[\catC,\catC]}
%% General functors
\newcommand{\funfont}[1]{#1}
\newcommand{\funF}{\funfont{F}}
\newcommand{\funU}{\funfont{U}}
\newcommand{\funG}{\funfont{G}}
\newcommand{\funT}{\funfont{T}}
\newcommand{\funI}{\funfont{I}}
%% Particular kinds of functors
\newcommand{\sfunfont}[1]{\mathrm{#1}}
\newcommand{\Pow}{\sfunfont{P}}
\newcommand{\Dist}{\sfunfont{D}}
\newcommand{\Maybe}{\sfunfont{M}}
\newcommand{\List}{\sfunfont{L}}
\newcommand{\UForg}{\sfunfont{U}}
\newcommand{\Forg}[1]{\sfunfont{U}_{#1}}
\newcommand{\Id}{\sfunfont{Id}}
\newcommand{\Vie}{\sfunfont{V}}
\newcommand{\Disc}{\funfont{D}}
\newcommand{\Weight}{\sfunfont{W}}
\newcommand{\homf}{\sfunfont{hom}}
\newcommand{\Yoneda}{\sfunfont{Y}}
%% Diagram functors
\newcommand{\Diag}{\mathscr{D}}
\newcommand{\KDiag}{\mathscr{K}}
\newcommand{\LDiag}{\mathscr{L}}
%% Monads
\newcommand{\monadfont}[1]{\mathbb{#1}}
\newcommand{\monadT}{\monadfont{T}}
\newcommand{\monadS}{\monadfont{S}}
\newcommand{\monadU}{\monadfont{U}}
\newcommand{\monadH}{\monadfont{H}}
\newcommand{\str}{\mathrm{str}}
%% Adjunctions
\newcommand\adjunct[2]{\xymatrix@=8ex{\ar@{}[r]|{\top}\ar@<1mm>@/^2mm/[r]^{{#2}}
& \ar@<1mm>@/^2mm/[l]^{{#1}}}}
\newcommand\adjunctop[2]{\xymatrix@=8ex{\ar@{}[r]|{\bot}\ar@<1mm>@/^2mm/[r]^{{#2}}
& \ar@<1mm>@/^2mm/[l]^{{#1}}}}
%% Retractions
\newcommand\retract[2]{\xymatrix@=8ex{\ar@{}[r]|{}\ar@<1mm>@/^2mm/@{^{(}->}[r]^{{#2}}
& \ar@<1mm>@/^2mm/@{->>}[l]^{{#1}}}}
%% Limits
\newcommand{\pv}[2]{\langle #1, #2 \rangle}
\newcommand{\limt}{\mathrm{lim}}
\newcommand{\pullbackcorner}[1][dr]{\save*!/#1+1.2pc/#1:(1,-1)@^{|-}\restore}
\newcommand{\pushoutcorner}[1][dr]{\save*!/#1-1.2pc/#1:(-1,1)@^{|-}\restore}
%% Colimits
\newcommand{\colim}{\mathrm{colim}}
\newcommand{\inl}{\mathrm{inl}}
\newcommand{\inr}{\mathrm{inr}}
%% Distributive categories
\newcommand{\distr}{\mathrm{dist}}
\newcommand{\undistr}{\mathrm{undist}}
%% Closedness
\newcommand{\curry}[1]{\mathrm{curry}{#1}}
\newcommand{\app}{\mathrm{app}}
%% Misc. operations
\newcommand{\const}[1]{\underline{#1}}
\newcommand{\comp}{\cdot}
\newcommand{\id}{\mathrm{id}}
%% Factorisations
\newcommand{\EClass}{E}
\newcommand{\MClass}{M}
\newcommand{\MConeClass}{\mathcal{M}}
%%%%%%%%%%%%%%%% End of Categorical Stuff

%%%% Misc
%% Operations
\newcommand{\blank}{\, - \,}
\newcommand{\sem}[1]{\llbracket #1 \rrbracket}
\newcommand{\closure}[1]{\overline{#1}}
\DeclareMathOperator{\img}{\mathrm{im}}
\DeclareMathOperator{\dom}{\mathrm{dom}}
\DeclareMathOperator{\codom}{\mathrm{codom}}
%% Sets of numbers
\newcommand{\Nats}{\mathbb{N}}
\newcommand{\Reals}{\mathbb{R}}
\newcommand{\Rz}{\Reals_{\geq 0}}
%% Writing
\newcommand{\cf}{\emph{cf.}}
\newcommand{\ie}{\emph{i.e.}}
\newcommand{\eg}{\emph{e.g.}}
\newcommand{\df}[1]{\emph{\textbf{#1}}}
%%%%%%%%%%%%%%%% End of Misc

%%%% Programming Stuff
%% Types
\newcommand{\typefont}[1]{\mathbb{#1}}
\newcommand{\typeOne}{1}
\newcommand{\typeTwo}{2}
\newcommand{\typeA}{\typefont{A}}
\newcommand{\typeB}{\typefont{B}}
\newcommand{\typeC}{\typefont{C}}
\newcommand{\typeV}{\typefont{V}}
\newcommand{\typeD}{\typefont{D}}
%% RuleName
\newcommand{\rulename}[1]{(\mathrm{#1})}
%% Sequents
\newcommand{\jud}{\vdash}
\newcommand{\vljud}{\vdash}
\newcommand{\cojud}{\vdash_{\co}}
\newcommand{\vl}{\mathtt{v}}
\newcommand{\co}{\mathtt{c}}
% Program font
\newcommand{\prog}[1]{\mathtt{#1}}
\newcommand{\pseq}[3]{#1 \leftarrow #2; #3}
\newcommand{\ppm}[4]{(#1,#2) \leftarrow #3; #4}
\newcommand{\pinl}[1]{\prog{inl}(#1)}
\newcommand{\pinr}[1]{\prog{inr}(#1)}
\newcommand{\pcase}[4]{\prog{ case } #1 \prog{ of } \pinl{#2} \Rightarrow #3 ; \pinr{#2} \Rightarrow #4}
%% Sets of terms
\newcommand{\ValuesBP}[2]{\mathsf{Values}(#1, #2)}
\newcommand{\TermsBP}[2]{\mathsf{Terms}(#1, #2)}
\newcommand{\closValP}[1]{\ValuesBP{\emptyset}{#1}}
\newcommand{\closTermP}[1]{\TermsBP{\emptyset}{#1}}
\newcommand{\closVal}{\closValP{\typeA}}
\newcommand{\closTerm}{\closTermP{\typeA}}
%% Contextual equivalence
\newcommand{\ctxeq}{\equiv_{\prog{ctx}}}
%%%% End of Programming Stuff

%-------------- template --------------------------------------------------
\usetheme{metropolis}
\metroset{block=fill}
%\usetheme{Boadilla}
% Configuring the foot line
\setbeamertemplate{footline}
{
  \leavevmode%
  \hbox{%
  \begin{beamercolorbox}[wd=.4\paperwidth,ht=2.25ex,dp=1ex,center]{author in head/foot}%
    \usebeamerfont{author in head/foot}\insertshortauthor
  \end{beamercolorbox}%
  \begin{beamercolorbox}[wd=.5\paperwidth,ht=2.25ex,dp=1ex,center]{title in head/foot}%
    \usebeamerfont{title in head/foot}\insertsection
  \end{beamercolorbox}%
  \begin{beamercolorbox}[wd=.1\paperwidth,ht=2.25ex,dp=1ex,right]{date in head/foot}%
    \insertframenumber{} / \inserttotalframenumber\hspace*{2ex} 
  \end{beamercolorbox}}%
  \vskip0pt%
}
% No configuration symbols
\setbeamertemplate{navigation symbols}{}
%--------------- packages -------------------------------------------------
\usepackage{graphicx,amsmath}
\usepackage{stmaryrd} % cf. interleave
\usepackage{booktabs}
\usepackage{amscd}
\usepackage{multicol}
\usepackage[absolute,overlay]{textpos}
\usepackage{alltt}
\usepackage{proof}
\usepackage{subcaption}
\usepackage{tikz}
\usepackage{tikz-cd}
\usepackage{quantikz}
\usepackage[new]{old-arrows}
\usepackage[all]{xy}
\usepackage{pgfplots}
\usepackage{textcomp}
\usetikzlibrary{arrows.meta, calc, fit, tikzmark}
\usepackage{pstricks,pst-node,pst-text,pst-3d}
% context
\AtBeginSection[]
{
    \begin{frame}
        \frametitle{Table of Contents}
        \tableofcontents[currentsection]
    \end{frame}
}

\author[Renato Neves]{Renato Neves}

% logos of institutions
\titlegraphic{
  \begin{textblock*}{5cm}(6.7cm,7.45cm)
     \includegraphics[scale=0.06]{logos/uminho.png}\hspace*{.85cm}~%
  \end{textblock*}
  \begin{textblock*}{5cm}(9.4cm,7.42cm)
    \includegraphics[scale=0.50]{logos/haslab.pdf}
  \end{textblock*}
}

% No date
\date{}

\begin{document}

\title{Entanglement and Teleportation}

\frame[plain]{\titlepage}

\section{The Problem}

\begin{frame}{Mission (apparently) Impossible}

        \begin{block}{The Problem}
                Two secure labs and in one of these a qubit 

                Terrain between the two labs full of entities
                that wish to access the qubit's state

                How to transfer the quantum state from one lab to the other?
        \end{block}

        \pause
        Classically, the complete data would need to be \alert{\underline{moved}}
        from point A to point B

        \pause
        Quantumly, we can do better via entanglement
\end{frame}

\section{Recap}

\begin{frame}{Entanglement}

        \begin{definition}
                A vector $u \in V \otimes W$ is \alert{entangled} if it cannot
                be written as a tensor $v \otimes w$ with $v \in V$ and $w \in
                W$
        \end{definition}

        \pause
        \begin{example}
                All four states below are entangled
                \begin{align*}
                        \textstyle{\frac{1}{\sqrt{2}}}(\ket{00} + \ket{11})
                        \qquad \qquad &
                        \textstyle{\frac{1}{\sqrt{2}}}(\ket{00} - \ket{11})
                        \\
                        \textstyle{\frac{1}{\sqrt{2}}}(\ket{01} + \ket{10})
                        \qquad \qquad &
                        \textstyle{\frac{1}{\sqrt{2}}}(\ket{01} - \ket{10})
                \end{align*}
                They form a basis of $\Complex^4$,
                which is often called the \alert{Bell basis}
        \end{example}
\end{frame}

\begin{frame}{Building Bell States -- an important ingredient}

        Every quantum operation
        \begin{quantikz}
               &[4mm] \gate{U} \qwbundle{n} &[4mm] \qw \qwbundle{n}
        \end{quantikz}$\,$
        gives rise to a \alert{`controlled'} quantum operation

        \[ 
        \left \sem{ 
                \begin{quantikz}
                &[4mm] \ctrl{1} &[4mm] \qw  \\
                &[4mm] \gate{U} \qwbundle{n} &[4mm] \qw \qwbundle{n}
                \end{quantikz}
        \right } = cU : \Complex^2 \otimes \Complex^{2^n} 
        \to \Complex^2 \otimes \Complex^{2^n}
        \]

        defined as $cU(\ket{0} \ket{b}) = \ket{0}  \ket{b}$, 
        $cU(\ket{1}  \ket{b}) = \ket{1} U\ket{b}$ 

        \pause
        \vfill
        \small{
        \textbf{N.B.} 
        The circuit
        \begin{quantikz}
                &\ctrl{1} & \qw  \\
                &\gate{X} & \qw 
        \end{quantikz}\
        is also denoted as
        \begin{quantikz}\
                & \ctrl{1} & \qw  \\
                & \targ{} & \qw 
        \end{quantikz}\
        }
\end{frame}

\begin{frame}{Building Bell States -- the circuit }

        \[
                \begin{quantikz}
                & \gate{H}  & \ctrl{1} & \qw  \\
                & \qw       & \targ{} & \qw 
                \end{quantikz}
        \]

        Every vector in the computational basis of $\Complex^4$ when fed to the
        circuit above yields a Bell state

\end{frame}

\begin{frame}{Measurement}

        Two maps $M_0$ and $M_1$ of type $\Complex^2 \to \Complex^2$ for
        measuring a qubit
        \[
                M_0 = \begin{pmatrix}
                        1 & 0 \\
                        0 & 0 
                \end{pmatrix}
                \qquad
                M_1 = \begin{pmatrix}
                        0 & 0 \\
                        0 & 1
                \end{pmatrix}
        \]

        A map $M_k : \Complex^{2} \to \Complex^{2}$, $k \in \{0,1\}$ possibly
        tensored with identities $\id : \Complex^2 \to \Complex^2$ is called a
        \alert{measurement}


        \vfill
        \begin{block}{Postulate}
           Take a state $v \in \Complex^{2^n}$ and measurement $M :
           \Complex^{2^n} \to \Complex^{2^n}$
           \begin{itemize}
                \item Probability of the outcome represented by $M$ is $\langle
                        M v, M v \rangle$
                \item State after the observed outcome is $\frac{1}{\norm{M
                        v}}M v$

           \end{itemize}         
        \end{block}
\end{frame}

\section{A First Approach to Quantum Teleportation}

\begin{frame}{Quantum Teleportation Intra-Gate pt. I}
        We transfer the top wire qubit's state to the bottom wire
        \[
                \begin{quantikz}
                \ket{\psi} & \ctrl{1} & \gate{H} & \qw  \\
                \ket{0} & \targ{} & \qw & \qw 
                \end{quantikz}
        \]
        \begin{flalign*}
              & \, (H \otimes I) cX (\alpha \ket{0} + \beta \ket{1}) \ket{0} & \\
              & = (H \otimes I) cX (\alpha \ket{00} + \beta \ket{11}) & \\
              & = H \otimes I (\alpha\ket{00} + \ket{11}) & \\
              & = \ket{+}\alpha \ket{0} + \ket{-} \beta \ket{1} \\
              & = \textstyle{\frac{1}{\sqrt{2}}} (
              \ket{0} \alpha \ket{0} + \ket{1} \alpha \ket{0} +
              \ket{0} \beta \ket{1} - \ket{1} \beta \ket{1}) \\
              & = \textstyle{\frac{1}{\sqrt{2}}} \left (
              \ket{0} \alert{(\alpha \ket{0} + \beta \ket{1})} + 
              \ket{1} \alert{(\alpha \ket{0} - \beta \ket{1})}  \right )
        \end{flalign*}
\end{frame}

\begin{frame}{Quantum Teleportation Intra-Gate pt. II}
        We transfer the top wire qubit's state to the bottom wire
        \[
                \begin{quantikz}
                \ket{\psi} & \ctrl{1} & \gate{H} & \qw  \\
                \ket{1} & \targ{} & \qw & \qw 
                \end{quantikz}
        \]

        \begin{flalign*}
              & \, (H \otimes I) cX (\alpha \ket{0} + \beta \ket{1}) \ket{1} & \\
              & = \dots & \\
              & = \textstyle{\frac{1}{\sqrt{2}}} \left (
              \ket{0} \alert{(\alpha \ket{1} + \beta \ket{0})} + 
              \ket{1} \alert{(\alpha \ket{1} - \beta \ket{0})}  \right )
        \end{flalign*}
\end{frame}

\begin{frame}{Quantum Teleportation Intra-Gate pt. III}
     Problem?
     \pause

     We can indeed transfer a qubit's state, but both qubits
     need to be connected by a logical gate

     \pause
     Fortunately we can do better \dots

     \pause
     \dots\ we will use \alert{entanglement} to connect the 
     two qubits (this presents no distance limitations)
\end{frame}

\section{Quantum Teleportation}

\begin{frame}{Quantum Teleportation -- the circuit}

        \begin{center}
                \begin{quantikz}[transparent]
                        \lstick{\ket{\psi}} & \qw & \qw & \ctrl{1} 
                        \gategroup[wires=2,steps=2,style={dashed,
                        rounded corners,fill=blue!10, inner xsep=0.2pt},background]
                        {\tiny{Teleportation intra-gate}}
                        & \gate{H} & \qw \\
                        \lstick{\ket{0}} & \gate{H}
                        \gategroup[wires=2,steps=2,style={dashed,
                        rounded corners,fill=blue!10, inner xsep=0.2pt},
                        label style={label position = below, yshift=-0.4cm}, background]
                        {\tiny{Creation of Bell state}}
                        & \ctrl{1} & \targ{} & \qw & \qw \\
                        \lstick{\ket{0}} & \qw & \targ{} & \qw & \qw & \qw
                \end{quantikz}
        \end{center} 

        Bottom qubits become entangled and thus connected, even if far away
        from each other later on 
\end{frame}

\begin{frame}{Quantum Teleportation -- the computation}

        \begin{flalign*}
           & \, ((H \otimes I) \otimes I) (cX \otimes I)  \left (
           (\alpha \ket{0} + \beta \ket{1}) \otimes \textstyle{\frac{1}{\sqrt{2}}}
           (\ket{00} + \ket{11}) \right ) 
           &  \\
           & = \textstyle{\frac{1}{\sqrt{2}}}
           ((H \otimes I) \otimes I) (cX \otimes I)  \Big (
           \alpha \ket{000} + \alpha \ket{011} + \beta \ket{100} + \beta \ket{111}
           \Big ) 
           & \\
           & = \textstyle{\frac{1}{\sqrt{2}}}
           ((H \otimes I) \otimes I)  \Big (
                   \alpha \ket{000} + \alpha \ket{011} + 
                   \beta \ket{110} + \beta \ket{101}
           \Big ) 
           & \\
           & = \textstyle{\frac{1}{\sqrt{2}}}
           ((H \otimes I) \otimes I)  \Big (
           \ket{0} (\alpha \ket{00} + \alpha \ket{11}) + 
           \ket{1} (\beta \ket{10} + \beta \ket{01})
           \Big ) 
           &  \\
           & = \textstyle{\frac{1}{2}}
           \Big (
           (\ket{0} + \ket{1}) (\alpha \ket{00} + \alpha \ket{11}) + 
           (\ket{0} - \ket{1}) (\beta \ket{10} + \beta \ket{01})
           \Big ) 
           & \\
           & = \Big  (
                   \ket{00}  \alert{(\alpha \ket{0} + \beta \ket{1})} +
                   \ket{01}  \alert{(\alpha \ket{1} + \beta \ket{0})} +
                   \ket{10}  \alert{(\alpha \ket{0} - \beta \ket{1})}\ \dots  
           & 
           \\
           & \dots +  \ket{11}  \alert{(\alpha \ket{1} - \beta \ket{0})} \Big ) &
        \end{flalign*}
\end{frame}

\begin{frame}{Quantum Teleportation -- full circuit}

        \begin{center}
                \begin{quantikz}[transparent]
                        \lstick{\ket{\psi}} & \qw & \qw & \ctrl{1} 
                        \gategroup[wires=2,steps=2,style={dashed,
                        rounded corners,fill=blue!10, inner xsep=0.2pt},background]
                        {\tiny{Teleportation intra-gate}}
                        & \gate{H} & \meter{} & \ctrl{2} 
                        \gategroup[wires=3,steps=2,style={dashed,
                        rounded corners,fill=blue!10, inner xsep=0.2pt},background]
                        {\tiny{Correction}}
                        & \qw & \qw \\
                        \lstick{\ket{0}} & \gate{H}
                        \gategroup[wires=2,steps=2,style={dashed,
                        rounded corners,fill=blue!10, inner xsep=0.2pt},
                        label style={label position = below, yshift=-0.4cm}, background]
                        {\tiny{Creation of Bell state}}
                        & \ctrl{1} & \targ{} & \qw & \meter{} & \qw & \ctrl{1} & \qw \\
                        \lstick{\ket{0}} & \qw & \targ{} & \qw & \qw & \qw & \gate{Z}
                                         & \gate{X} & \qw
                \end{quantikz}
        \end{center} 
\end{frame}

\section{Afterthoughts}

\begin{frame}{Did we just break physics?}

        \begin{block}{No-cloning}
                Did not end up with two copies of $\ket{\psi}$, because
                the state of the top qubit was destroyed by the measurement
        \end{block}

        \pause
        \begin{block}{FTL communication}
                Did not communicate faster than light, because the communication
                required sending two classical bits
        \end{block}
\end{frame}

\begin{frame}{What's Next?}

        First glimpse of applications of quantum phenomena to algorithmics
        and communication
        \begin{itemize}
                \item superposition \& interference
                \item entanglement
        \end{itemize}

        We will next overview more sophisticated applications of these
        phenomena
\end{frame}
\end{document}
