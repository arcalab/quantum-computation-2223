\documentclass{beamer}
\usepackage{etex} % fixes new-dimension error
\usepackage{lmodern}
\usepackage[T1]{fontenc}
\usepackage{mathtools}
%%%%%%%%%%%%% Macros
%%%% Categories
\newcommand{\catfont}[1]{\mathsf{#1}}
\newcommand{\cop}{\catfont{op}}
\newcommand{\Law}{\catfont{Law}}
\newcommand{\catV}{\catfont{V}}
\newcommand{\catX}{\catfont{X}}
\newcommand{\catC}{\catfont{C}}
\newcommand{\catD}{\catfont{D}}
\newcommand{\catA}{\catfont{A}}
\newcommand{\catB}{\catfont{B}}
\newcommand{\catI}{\catfont{I}}
\newcommand{\Set}{\catfont{Set}}
\newcommand{\Top}{\catfont{Top}}
\newcommand{\Pos}{\catfont{Pos}}
\newcommand{\Inj}{\catfont{Inj}}
\newcommand{\Det}{\catfont{RMhat}}
\newcommand{\CoAlg}[1]{\catfont{CoAlg}\left (#1 \right )}
\newcommand{\Mon}{\catfont{Mon}}
\newcommand{\Mnd}{\catfont{Mnd}(\catC)}
\newcommand{\SMnd}{\catfont{Mnd}(\Set)}
\newcommand{\CLat}{\catfont{CLat}}
\newcommand{\Stone}{\catfont{Stone}}
\newcommand{\Spectral}{\catfont{Spectral}}
\newcommand{\CompHaus}{\catfont{CompHaus}}
\newcommand{\Subs}[2]{\catfont{Sub}_{}}
\newcommand{\Cone}{\catfont{Cone}}
\newcommand{\StComp}{\catfont{StablyComp}}
\newcommand{\PosC}{\catfont{PosComp}}
\newcommand{\Haus}{\catfont{Haus}}
\newcommand{\Meas}{\catfont{Meas}}
\newcommand{\Ord}{\catfont{Ord}}
\newcommand{\EndoC}{[\catC,\catC]}
%% General functors
\newcommand{\funfont}[1]{#1}
\newcommand{\funF}{\funfont{F}}
\newcommand{\funU}{\funfont{U}}
\newcommand{\funG}{\funfont{G}}
\newcommand{\funT}{\funfont{T}}
\newcommand{\funI}{\funfont{I}}
%% Particular kinds of functors
\newcommand{\sfunfont}[1]{\mathrm{#1}}
\newcommand{\Pow}{\sfunfont{P}}
\newcommand{\Dist}{\sfunfont{D}}
\newcommand{\Maybe}{\sfunfont{M}}
\newcommand{\List}{\sfunfont{L}}
\newcommand{\UForg}{\sfunfont{U}}
\newcommand{\Forg}[1]{\sfunfont{U}_{#1}}
\newcommand{\Id}{\sfunfont{Id}}
\newcommand{\Vie}{\sfunfont{V}}
\newcommand{\Disc}{\funfont{D}}
\newcommand{\Weight}{\sfunfont{W}}
\newcommand{\homf}{\sfunfont{hom}}
\newcommand{\Yoneda}{\sfunfont{Y}}
%% Diagram functors
\newcommand{\Diag}{\mathscr{D}}
\newcommand{\KDiag}{\mathscr{K}}
\newcommand{\LDiag}{\mathscr{L}}
%% Monads
\newcommand{\monadfont}[1]{\mathbb{#1}}
\newcommand{\monadT}{\monadfont{T}}
\newcommand{\monadS}{\monadfont{S}}
\newcommand{\monadU}{\monadfont{U}}
\newcommand{\monadH}{\monadfont{H}}
\newcommand{\str}{\mathrm{str}}
%% Adjunctions
\newcommand\adjunct[2]{\xymatrix@=8ex{\ar@{}[r]|{\top}\ar@<1mm>@/^2mm/[r]^{{#2}}
& \ar@<1mm>@/^2mm/[l]^{{#1}}}}
\newcommand\adjunctop[2]{\xymatrix@=8ex{\ar@{}[r]|{\bot}\ar@<1mm>@/^2mm/[r]^{{#2}}
& \ar@<1mm>@/^2mm/[l]^{{#1}}}}
%% Retractions
\newcommand\retract[2]{\xymatrix@=8ex{\ar@{}[r]|{}\ar@<1mm>@/^2mm/@{^{(}->}[r]^{{#2}}
& \ar@<1mm>@/^2mm/@{->>}[l]^{{#1}}}}
%% Limits
\newcommand{\pv}[2]{\langle #1, #2 \rangle}
\newcommand{\limt}{\mathrm{lim}}
\newcommand{\pullbackcorner}[1][dr]{\save*!/#1+1.2pc/#1:(1,-1)@^{|-}\restore}
\newcommand{\pushoutcorner}[1][dr]{\save*!/#1-1.2pc/#1:(-1,1)@^{|-}\restore}
%% Colimits
\newcommand{\colim}{\mathrm{colim}}
\newcommand{\inl}{\mathrm{inl}}
\newcommand{\inr}{\mathrm{inr}}
%% Distributive categories
\newcommand{\distr}{\mathrm{dist}}
\newcommand{\undistr}{\mathrm{undist}}
%% Closedness
\newcommand{\curry}[1]{\mathrm{curry}{#1}}
\newcommand{\app}{\mathrm{app}}
%% Misc. operations
\newcommand{\const}[1]{\underline{#1}}
\newcommand{\comp}{\cdot}
\newcommand{\id}{\mathrm{id}}
%% Factorisations
\newcommand{\EClass}{E}
\newcommand{\MClass}{M}
\newcommand{\MConeClass}{\mathcal{M}}
%%%%%%%%%%%%%%%% End of Categorical Stuff

%%%% Misc
%% Operations
\newcommand{\blank}{\, - \,}
\newcommand{\sem}[1]{\llbracket #1 \rrbracket}
\newcommand{\closure}[1]{\overline{#1}}
\DeclareMathOperator{\img}{\mathrm{im}}
\DeclareMathOperator{\dom}{\mathrm{dom}}
\DeclareMathOperator{\codom}{\mathrm{codom}}
%% Sets of numbers
\newcommand{\Nats}{\mathbb{N}}
\newcommand{\Reals}{\mathbb{R}}
\newcommand{\Rz}{\Reals_{\geq 0}}
%% Writing
\newcommand{\cf}{\emph{cf.}}
\newcommand{\ie}{\emph{i.e.}}
\newcommand{\eg}{\emph{e.g.}}
\newcommand{\df}[1]{\emph{\textbf{#1}}}
%%%%%%%%%%%%%%%% End of Misc

%%%% Programming Stuff
%% Types
\newcommand{\typefont}[1]{\mathbb{#1}}
\newcommand{\typeOne}{1}
\newcommand{\typeTwo}{2}
\newcommand{\typeA}{\typefont{A}}
\newcommand{\typeB}{\typefont{B}}
\newcommand{\typeC}{\typefont{C}}
\newcommand{\typeV}{\typefont{V}}
\newcommand{\typeD}{\typefont{D}}
%% RuleName
\newcommand{\rulename}[1]{(\mathrm{#1})}
%% Sequents
\newcommand{\jud}{\vdash}
\newcommand{\vljud}{\vdash}
\newcommand{\cojud}{\vdash_{\co}}
\newcommand{\vl}{\mathtt{v}}
\newcommand{\co}{\mathtt{c}}
% Program font
\newcommand{\prog}[1]{\mathtt{#1}}
\newcommand{\pseq}[3]{#1 \leftarrow #2; #3}
\newcommand{\ppm}[4]{(#1,#2) \leftarrow #3; #4}
\newcommand{\pinl}[1]{\prog{inl}(#1)}
\newcommand{\pinr}[1]{\prog{inr}(#1)}
\newcommand{\pcase}[4]{\prog{ case } #1 \prog{ of } \pinl{#2} \Rightarrow #3 ; \pinr{#2} \Rightarrow #4}
%% Sets of terms
\newcommand{\ValuesBP}[2]{\mathsf{Values}(#1, #2)}
\newcommand{\TermsBP}[2]{\mathsf{Terms}(#1, #2)}
\newcommand{\closValP}[1]{\ValuesBP{\emptyset}{#1}}
\newcommand{\closTermP}[1]{\TermsBP{\emptyset}{#1}}
\newcommand{\closVal}{\closValP{\typeA}}
\newcommand{\closTerm}{\closTermP{\typeA}}
%% Contextual equivalence
\newcommand{\ctxeq}{\equiv_{\prog{ctx}}}
%%%% End of Programming Stuff

%-------------- template --------------------------------------------------
\usetheme{metropolis}
\metroset{block=fill}
%\usetheme{Boadilla}
% Configuring the foot line
\setbeamertemplate{footline}
{
  \leavevmode%
  \hbox{%
  \begin{beamercolorbox}[wd=.4\paperwidth,ht=2.25ex,dp=1ex,center]{author in head/foot}%
    \usebeamerfont{author in head/foot}\insertshortauthor
  \end{beamercolorbox}%
  \begin{beamercolorbox}[wd=.5\paperwidth,ht=2.25ex,dp=1ex,center]{title in head/foot}%
    \usebeamerfont{title in head/foot}\insertsection
  \end{beamercolorbox}%
  \begin{beamercolorbox}[wd=.1\paperwidth,ht=2.25ex,dp=1ex,right]{date in head/foot}%
    \insertframenumber{} / \inserttotalframenumber\hspace*{2ex} 
  \end{beamercolorbox}}%
  \vskip0pt%
}
% No configuration symbols
\setbeamertemplate{navigation symbols}{}
%--------------- packages -------------------------------------------------
\usepackage{graphicx,amsmath}
\usepackage{stmaryrd} % cf. interleave
\usepackage{booktabs}
\usepackage{amscd}
\usepackage{multicol}
\usepackage[absolute,overlay]{textpos}
\usepackage{alltt}
\usepackage{proof}
\usepackage{subcaption}
\usepackage{tikz}
\usepackage{tikz-cd}
\usepackage{quantikz}
\usepackage[new]{old-arrows}
\usepackage[all]{xy}
\usepackage{pgfplots}
\usepackage{textcomp}
\usetikzlibrary{arrows.meta, calc, fit, tikzmark, fillbetween}
\usepackage{pstricks,pst-node,pst-text,pst-3d}
% context
\AtBeginSection[]
{
    \begin{frame}
        \frametitle{Table of Contents}
        \tableofcontents[currentsection]
    \end{frame}
}

\author[Renato Neves]{Renato Neves}

% logos of institutions
\titlegraphic{
  \begin{textblock*}{5cm}(6.7cm,7.45cm)
     \includegraphics[scale=0.06]{logos/uminho.png}\hspace*{.85cm}~%
  \end{textblock*}
  \begin{textblock*}{5cm}(9.4cm,7.42cm)
    \includegraphics[scale=0.50]{logos/haslab.pdf}
  \end{textblock*}
}

% No date
\date{}

\begin{document}

\title{An Application of QPE: Order-Finding}

\frame[plain]{\titlepage}

\section{Introduction}

\begin{frame}{Period-Finding}

        \begin{block}{The Problem}
                A \alert{periodic} function $f$. Find its period.
        \end{block}

        \pause
        Problem can be difficult (particularly if $f$ has no obvious
        structure, such as being trigonometric)

        We will see how quantum computation tackles it
\end{frame}

\begin{frame}{Order-Finding}
        Actually we tackle only a specific
        case $\Rightarrow$  \alert{order-finding}

        The latter is handled efficiently via QPE 

        \alert{Integer factorisation} reduces to it

        The only quantum component in Shor's algorithm
\end{frame}

\section{A sprinkle of number theory}

\begin{frame}{A Handful of Definitions}

        \begin{definition}
                We call the integer $x$ a \alert{divisor} of the 
                integer $y$ if $k \cdot x = y$ for some integer $k$
        \end{definition}

        \begin{block}{Examples}
                $2$ is a divisor of $10$ and $5$ is a divisor of $15$. What are
                the divisors of a prime number?
        \end{block}

        \begin{definition}
                For two integers $x$ and $y$, $\alert{gcd(x,y)}$ is the
                greatest divisor common to $x$ and $y$
        \end{definition}

        \begin{block}{Examples}
                $gcd(8,12) = 4$ and $gcd(10,15) = 5$
        \end{block}
\end{frame}

\begin{frame}{A Handful of Definitions pt. II}

        \begin{definition}
                Two integers $x$ and $y$ are called
                \alert{co-prime} if $gcd(x,y) = 1$
        \end{definition}

        \begin{block}{Examples}
                $8$ and $9$ are co-prime and $13$ and $15$
                are co-prime as well. The integers $12$ and
                $15$ are not co-prime.
        \end{block}
\end{frame}

\begin{frame}{Modular Arithmetic}
        \begin{definition}
                Given an integer $N$ the set of \alert{integers mod $N$} is $\{
                0, 1, \dots, N - 1 \}$
        \end{definition}

        We can think of this set as a \alert{circular} circuit with different
        positions and where the position after $N-1$ is $0$

        \begin{definition}
                For two integers $x$ and $y$ we write $\alert{x \equiv y \,
                (\mathrm{mod} \, N)}$ if $x \, \mathrm{mod} \, N = y$ 
        \end{definition}

        \begin{block}{Examples}
                $5 \equiv 0 \, (\mathrm{mod} \, 5)$ and
                $6 \equiv 1 \, (\mathrm{mod} \, 5)$
        \end{block}
\end{frame}

\begin{frame}{Order-Finding}

        \begin{definition}
                For co-prime integers $a < N$ the \alert{order of $a \,
                (\mathrm{mod} \, N)$} is the smallest integer $r > 0$ s.t.
                $a^r \equiv 1 \, (\mathrm{mod} \, N)$
        \end{definition}

        \begin{block}{Example}
                If $N = 5$ the sequence $3^0, 3^1, 3^2, 3^3, 3^4, 3^5, 3^6,
                \dots$ leads to the sequence $1, 3, 4, 2, \alert{1}, 3, 4,
                \dots$ 

                Order of $3 \, (\mathrm{mod} \, 5)$ is thus $4$
        \end{block}

        \begin{block}{Exercise}
                What is the order of $2 \, (\mathrm{mod} \, 11)$?
        \end{block}
\end{frame}

\section{The problem of order-finding}

\begin{frame}{Order-Finding}
        \begin{block}{The Problem}
                Co-prime integers $a < N$ 

                What is the order of $a \, (\mathrm{mod} \, N)$?
        \end{block}

        \pause
        Classically, problem can be difficult for large integers

        Quantumly, it can be solved efficiently via QPE
\end{frame}

\begin{frame}{QPE Revisited}

        Recall the QPE circuit
        \begin{center}
                \begin{quantikz}
                        \lstick{\ket{0}} & \qwbundle{n} 
                                         & \gate{QFT_n}
                                         \gategroup[wires=1,steps=1,style={dashed,
                                         rounded corners,fill=blue!10, inner xsep=0.2pt}, 
                                         label style={label position = 
                                         above, yshift=0cm},background]
                                         {\tiny{converts $\ket{0}$ to Fourier basis}} 
                                         & \ctrl{1}\gategroup[wires=2,steps=1,style={dashed,
                                         rounded corners,fill=blue!10, inner xsep=0.2pt}, 
                                         label style={label position = 
                                         below, yshift=-0.4cm,xshift=-0.2cm},background]
                                         {\tiny{encodes $k$ in local phases (in the form of
                                         rotations)}}
                                         & \gate{QFT^{-1}_n}
                                         \gategroup[wires=1,steps=1,style={dashed,
                                         rounded corners,fill=blue!10, inner xsep=0.2pt}, 
                                         label style={label position = 
                                         above, yshift=0.0cm, xshift=0.6cm},background]
                                         {\tiny{converts encoded info. to 
                                         comput. basis}}
                                         & \qwbundle{n}
                        \\
                        \lstick{\ket{\psi}} & \qwbundle{m}
                                            & \qw
                                            & \gate{U}
                                            & \qw
                                            & \qwbundle{m}
                \end{quantikz}
        \end{center}

        Need to choose suitable $U$ and $\ket{\psi}$ to disclose the order
\end{frame}

\section{Choosing suitale input parameters in QPE}

\begin{frame}{Choosing the Right Unitary}

        Take co-prime integers $a < N$

        Let $m = \lceil \log_2 N \rceil$ and define
        $U : \Complex^{2^m} \to \Complex^{2^m}$
        \[
                U \ket{x} = \begin{cases}
                        \ket{ x a \, (\mathrm{mod} \, N)} & \text{if }
                        0 \leq x \leq N -1 \\
                        \ket{x} & \text{otherwise}
                \end{cases}
        \]

        \begin{block}{Exercise}
                Show that $U \ket{a^n \, (\mathrm{mod} \, N)} = 
                \ket{a^{n+1} \, (\mathrm{mod} \, N)}$
        \end{block}

        \pause
        Next step is to identify \alert{\underline{suitable eigenvectors}}
\end{frame}

\begin{frame}{Starting with an Example}
                Recall: if $N = 5$ sequence $3^0, 3^1, 3^2, 3^3, 3^4, 3^5, 3^6,
                \dots$ leads to $\underline{1, 3, 4, 2}, \alert{1}, 3, 4, \dots$ 

                Order $r$ of $3 \, (\mathrm{mod} \, 5)$ is $4$. We
                then calculate,
                \begin{flalign*}
                        & \, \textstyle{
                                U \Big (\frac{1}{\sqrt{r}} (\ket{1} + \ket{3} 
                        + \ket{4} + \ket{2} \Big )} &
                        \\
                        & \,
                        \textstyle{
                                = U \Big (\frac{1}{\sqrt{r}} 
                                \sum_{i = 0}^{r-1} \ket{3^i \, (\mathrm{mod} \, 5)} \Big ) 
                        } &
                        \\
                        & \,
                        \textstyle{
                                = \frac{1}{\sqrt{r}} 
                                \sum_{i = 0}^{r-1} \ket{3^{i+1} \, (\mathrm{mod} \, 5)}
                        } &
                        \\
                        & \,
                        \textstyle{
                                = \frac{1}{\sqrt{r}} \Big (
                                        \ket{3} + \ket{4} + \ket{2} + \ket{1}
                                \Big )
                        } &
                        \\
                        & \,
                        \textstyle{
                                = \frac{1}{\sqrt{r}} \Big (
                                        \ket{1} + \ket{3} + \ket{4} + \ket{2}
                                \Big )
                        } &
                \end{flalign*}
                The latter state is therefore an eigenvector of $U$
\end{frame}

\begin{frame}{A First Approach} 
        Previous example alludes to the equation
        \[
                \textstyle{
                                U \Big (\frac{1}{\sqrt{r}} 
                                \sum_{i = 0}^{r-1} \ket{a^i \, (\mathrm{mod} \, N)} \Big ) 
                } 
                =
                \textstyle{
                                \frac{1}{\sqrt{r}} 
                                \sum_{i = 0}^{r-1} \ket{a^i \, (\mathrm{mod} \, N)} 
                } 
        \]
        Unfortunately, corresponding eigenvalue is $1 = e^{i 2 \pi  \alert{0} 
        \cdot \frac{1}{2^n}}$

        It does not disclose any information about the period $r$ :(

        \pause
        Need to find eigenvectors with \alert{more informative eigenvalues}
\end{frame}

\begin{frame}{A Second Approach}

        Let $\omega = e^{i 2 \pi \cdot \frac{1}{r}}$ 
        $\underbrace{\text{(division of the \alert{\underline{unit circle}} in
        $r$ slices)}}_{\text{a.k.a. the r roots of unity}}$
        \begin{flalign*}
                &
                \textstyle{
                \, U \Big ( \frac{1}{\sqrt{r}} \sum_{i = 0}^{r-1} \omega^{-i} 
                \ket{a^i \, (\mathrm{mod} \, N) }\Big )
                } &
                \\
                & = \textstyle{
                \frac{1}{\sqrt{r}} \sum_{i = 0}^{r-1} \omega^{-i} 
                \ket{a^{i +1} \, (\mathrm{mod} \, N) }
                } &
                \\
                & = \textstyle{
                \frac{1}{\sqrt{r}} \sum_{i = 0}^{r-1} \omega \omega^{-(i+1)} 
                \ket{a^{i +1} \, (\mathrm{mod} \, N) }
                } &
                \\& = \textstyle{ \omega \left (
                \frac{1}{\sqrt{r}} \sum_{i = 0}^{r-1} \omega^{-(i+1)} 
                \ket{a^{i +1} \, (\mathrm{mod} \, N) }\right )
                } &
                \\ & = \textstyle{ \omega \left (
                \frac{1}{\sqrt{r}} \sum_{i = 0}^{r-1} \omega^{-i} 
                \ket{a^{i} \, (\mathrm{mod} \, N) }\right )
                } &
        \end{flalign*}

        \begin{block}{Exercise}
                Formally justify all the steps in the calculation
                above
        \end{block}
\end{frame}

\begin{frame}{A Second Approach}
        Let $\omega = e^{i 2 \pi \cdot \frac{1}{r}}$ and $\ket{\psi_1} =
        \frac{1}{\sqrt{r}} \sum_{i = 0}^{r-1} \omega^{-i} \ket{a^{i} \,
        (\mathrm{mod} \, N) }$

        Previous slide says $U\ket{\psi_1} = \omega \ket{\psi_1}$

        So if we feed QPE with  $\alert{U}$ and  $\alert{\ket{\psi_1}}$ we
        obtain an approximation of $\alert{\frac{1}{r}}$ with good success
        probability ($\geq \frac{4}{\pi^2} \approx 0.4$)

        \pause
        However $\ket{\psi_1}$ is difficult to construct. Can you see why?

\end{frame}

\newcommand{\iprod}[2]{ \langle #1 | #2 \rangle}

\begin{frame}{A Third Approach}
        We define a \alert{superposition of eigenvectors}
        that is equal to $\ket{1}$:

        set $\ket{\psi_k} = \textstyle{ \frac{1}{\sqrt{r}} \sum_{i = 0}^{r - 1} \omega^{-ik}
        \ket{a^i \, (\mathrm{mod} \, N)} }$ and $\ket{\psi} = \textstyle{
        \frac{1}{\sqrt{r}} \sum_{k = 0}^{r-1} \ket{\psi_k} }$

        \begin{block}{Exercise}
                Then show $U\ket{\psi_k} = \omega^k \ket{\psi_k}$
        \end{block}

        \begin{block}{Exercise} Finally show $\ket{\psi} = \ket{1}$ (hint:
        show $\iprod{1}{\psi} = 1$ or alternatively use the closed-form
        formula of geometric series) 
        \end{block}
\end{frame}

\begin{frame}{A Third Approach}
        $U\ket{\psi_k} = \omega^{k} \ket{\psi_k} = e^{i 2 \pi
        \alert{\frac{k}{r}}} \ket{\psi_k}$ and $\alert{\ket{\psi}} = \frac{1}{\sqrt{r}}
        \sum_{i = 0}^{r - 1} \ket{\psi_k}$. Therefore
        \begin{center}
                \begin{quantikz}
                        \lstick{\ket{0}} & \qwbundle{n} 
                                         & \gate{QFT_n}
                                         \gategroup[wires=1,steps=1,style={dashed,
                                         rounded corners,fill=blue!10, inner xsep=0.2pt}, 
                                         label style={label position = 
                                         above, yshift=0cm},background]
                                         {\tiny{converts $\ket{0}$ to Fourier basis}} 
                                         & \ctrl{1}\gategroup[wires=2,steps=1,style={dashed,
                                         rounded corners,fill=blue!10, inner xsep=0.2pt}, 
                                         label style={label position = 
                                         below, yshift=-0.4cm,xshift=-0.2cm},background]
                                         {\tiny{encodes info. in local phases (in the form of
                                         rotations)}}
                                         & \gate{QFT^{-1}_n}
                                         \gategroup[wires=1,steps=1,style={dashed,
                                         rounded corners,fill=blue!10, inner xsep=0.2pt}, 
                                         label style={label position = 
                                         above, yshift=0.0cm, xshift=0.6cm},background]
                                         {\tiny{converts encoded info. to 
                                         comput. basis}}
                                         & \qwbundle{n}
                        \\
                        \lstick{\ket{\psi}} & \qwbundle{m}
                                            & \qw
                                            & \gate{U}
                                            & \qw
                                            & \qwbundle{m}
                \end{quantikz}
        \end{center}
        returns $\frac{1}{\sqrt{r}} \sum_{k = 0}^{r -1} \left (\ket{\tilde
        \phi_k} \ket{\psi_k} \right )$ where each $\ket{\tilde \phi_k}$ is the
        best $\alert{n}$-bit approximation of $\alert{\frac{k}{r}}$ with
        probability $\geq \frac{4}{\pi^2}$

        \pause
        But how to extract $r$ from $\ket{\tilde \phi_k}$?

\end{frame}

\begin{frame}{Extracting the Period}
        Let $\varphi$ be the best $n$-bit approximation of
        some $\frac{k}{r}$

        \begin{theorem}
                If $\left | \frac{k}{r} - \varphi \right | \leq \frac{1}{2r^2}$
                then we can extract $\frac{k}{r}$ in \underline{reduced form},
                and with complexity $O(m^3)$
        \end{theorem}

        \begin{proof}

                Uses the continued fractions alg. (see Appendix 4, Nielsen
                and Chuang, \emph{Quantum Computation and Quantum Information})
        \end{proof}
        Previous theorem tells we need to use a \alert{\underline{minimum
        number}} $n$ of qubits to represent $\varphi$. Particularly,
\end{frame}

\begin{frame}{Extracting the Period}
        recall: $m = \lceil log_2 N \rceil$
        \begin{flalign*}
                & \, 2^{n+1} \geq 2 r^2 & \\
                & \Leftarrow 2^{n+1} \geq 2(2^m)^2  & \left \{ r \leq N \leq 2^m \right \} \\
                & \Leftarrow 2 2^n \geq 2(2^m)^2  &  \\
                & \Leftarrow 2^n \geq 2^{2m} & \\
                & \Leftarrow n \geq 2m
        \end{flalign*}

        Thus the number of qubits to use in the approximation $\varphi$ should
        be at least $2m$
\end{frame}

\begin{frame}{Finally\dots}

        In order to obtain the order $r$, proceed with the following steps
        \begin{enumerate}
                \item run QPE + continued fractions alg. twice to obtain two
                        reduced fractions $\frac{k_1}{r_1}$ and
                        $\frac{k_2}{r_2}$
                \item if $gcd(k_1,k_2) = 1$ repeat previous step else set
                $r:=$ least common multiple of $r_1$ and $r_2$
                \item if $a^r \, (\mathrm{mod} \, N) \equiv 1$ 
                         output $r$ else go back to step $1$
        \end{enumerate}
       
        \pause
        In step 2, probability of $gcd(k_1,k_2) = 1$ is $\geq \frac{1}{4}$.
        Hence whole algorithm has \alert{\underline{constant probability}} of
        success
        
        In step 2, computation of $gcd$ and least common multiple has
        complexity $O(m^2)$. Hence the whole algorithm must be efficient
\end{frame}
\end{document}
