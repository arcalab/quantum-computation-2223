\documentclass[a4paper, 11pt]{article}

%% packages
\usepackage{fullpage} % changes the margin
\usepackage{hyperref} % Links
\usepackage[utf8]{inputenc}
\usepackage{lmodern}
\usepackage{amsfonts}
\usepackage{amsthm}
\usepackage{amsmath}
\usepackage{braket}
%%

%% macros
\newcommand{\complex}{\mathbb{C}}
\newcommand{\vecs}{\mathcal{V}}
\newcommand{\id}{\mathrm{id}}
%% environments
\theoremstyle{definition}
\newtheorem{definition}{Definition}
\newtheorem{examples}{Example}
\newtheorem{exercises}{Exercises}
\newtheorem{exercise}{Exercise}
\newtheorem{postulate}{Postulate}
%% config
\date{}
\linespread{1.15}
%%

\begin{document}

\title{Quantum Computing @ MEF \\ \large Background}
\author{Ana Neri \\ \scriptsize
\href{mailto:ana.i.neri@inesctec.pt}{ana.i.neri@inesctec.pt}}
\maketitle

\section{Quantum Measurement}

In order to render notation more convenient, we will often omit the
parentheses in function application and start to denote linear maps by
capital letters. Also, we will now use $\ket{0}$ and $\ket{1}$ to
denote the elements $(1,0)$ and $(0,1)$ in $\complex^2$,
respectively. We extend this notation to any space $\complex^{2^n}$ by
observing that,
\begin{align*}
  \complex^{2^n} \simeq\ \underbrace{\complex^2 \otimes \dots \otimes \complex^2}_
  {n \text{ times} }
\end{align*}
and representing
$\ket{b_1} \otimes \dots \otimes \ket{b_n} \in \complex^{2^n}$ simply
as $\ket{b_1, \dots, b_n}$. Thus, a vector $v \in \complex^2$ is a
linear combination $\alpha \ket{0} + \beta \ket{1}$ and $\| v \| = 1$
entails that the equation $| \alpha |^2 + | \beta |^2 = 1$
holds. Later on we will see that $| \alpha |^2$ is the probability of
observing $\ket{0}$ when measuring a qubit in state $v$ and
analogously for $| \beta |^2$. Similarly, a vector $v \in \complex^4$
is a linear combination
$\alpha \ket{00} + \beta \ket{01} + \gamma \ket{10} + \delta \ket{11}$
and $\| v \| = 1$ entails that the equation
$| \alpha |^2 + | \beta |^2 + | \gamma |^2 + | \delta |^2 = 1$ holds. The
component $| \alpha |^2$ is the probability of observing $\ket{00}$
when measuring two qubits at state $v$, and analogously for the three
other components.

In this course, we will heavily use two maps $M_0$ and $M_1$ of type
$\complex^2 \to \complex^2$ for measuring qubits. The map $M_0$ is
defined by the equations,
\begin{align*}
  M_0 \ket{0} = \ket{0} \hspace{2cm} M_0 \ket{1} = 0
\end{align*}
and represents the outcome of the qubit measured being at state
$\ket{0}$; the map $M_1$ arises from an analogous reasoning. For the
space $\complex^{2^n}$ we represent the outcome of the $i$-th
qubit being at state $\ket{k}$ by the map,
\begin{align*}
  \underbrace{\id \otimes \dots \otimes \id}_{i-1 \text{ times}} \otimes\ M_k \otimes
  \underbrace{\id \otimes \dots \otimes \id}_{n - i  \text{ times}}
  : \complex^{2^n} \to \complex^{2^n}
\end{align*}
We call `measurement maps' those maps that are built in this way and
that arise by composing measurement maps with one another.

\begin{postulate}[Quantum measurement]
  Let $v \in \complex^{2^n}$ be a quantum state of $n$ qubits and let
  us consider a measurement map
  $M : \complex^{2^n} \to \complex^{2^n}$. Then the probability of the
  outcome represented by $M$ is $\langle M \, v, M \, v \rangle$ and the
  quantum state of the $n$ qubits after the observed outcome is
  defined by,
  \begin{align*}
    \frac{M \, v}{\| M \, v \|}
  \end{align*}
  (note that we perform a normalisation, which is necessary because
  measurement maps are not unitary).
\end{postulate}

\begin{exercise}
  Let $H : \complex^2 \to \complex^2$ be the unitary map defined by
  the matrix,
  \begin{align*} \frac{1}{\sqrt{2}} \cdot
     \begin{bmatrix}
       1 & 1 \\
       1 & -1
  \end{bmatrix}
  \end{align*}
  What is the probability of the outcome $\ket{0}$ when measuring
  $H \ket{0}$?
\end{exercise}

\begin{exercise}
  Consider the quantum state,
  \begin{align*}
    \frac{1}{2} \ket{00} + \frac{1}{2} \ket{01} + \frac{1}{2} \ket{10} +
    \frac{1}{2} \ket{11}
  \end{align*}
  What is the probability of the outcome $\ket{0}$ when measuring the
  leftmost qubit? Let us assume that we indeed observed that the
  leftmost qubit is at state $\ket{0}$. What is the probability of
  the outcome $\ket{1}$ when measuring the rightmost qubit?
\end{exercise}

\begin{exercise}
  Consider the quantum state,
  \begin{align*}
    \frac{1}{\sqrt{2}} \ket{00} + \frac{1}{\sqrt{2}} \ket{11}
  \end{align*}
  What is the probability of the outcome $\ket{0}$ when measuring the
  leftmost qubit? What is the probability of the outcome $\ket{1}$
  when measuring the rightmost qubit? Assume that we indeed observed
  that the leftmost qubit is at state $\ket{0}$. Then what is the
  probability of the outcome $\ket{1}$ when measuring the rightmost
  qubit?~\footnote{The quantum state briefly studied in this exercise
    is one of those that gave rise to the famous phrase 
    `spooky action at a distance' by A. Einstein.}
\end{exercise}

\section{Entanglement}

Consider two vector spaces $V$ and $W$. We say that a vector
$u \in V \otimes W$ is entangled if it cannot be written as
$v \otimes w$ for some $v \in V$ and $w \in W$. In words, the state
$u$ (of a composite system) is entangled if it cannot be seen as a
mere aggregation $v, w$ of states (of the constituent systems). If the
state $u$ is not entangled then we say that is separable.

\begin{exercise}
  Show that the quantum state,
  \begin{align*}
    \frac{1}{\sqrt{2}} \ket{00} + \frac{1}{\sqrt{2}} \ket{11}
  \end{align*}
  is entangled.
\end{exercise}


The quantum state
$\frac{1}{\sqrt{2}} \ket{00} + \frac{1}{\sqrt{2}} \ket{11}$ (mentioned
in the previous exercise) can be obtained from the unitary map
$CX \cdot (H \otimes \id)$ and the initial state
$\ket{0} \otimes \ket{0}$, where
$CX : \complex^2 \otimes \complex^2 \to \complex^2 \otimes \complex^2$
reads as ``controlled not'' and is defined as,
\begin{align*}
  CX \ket{00} = \ket{00},
  \hspace{0.7cm}
  CX \ket{01} = \ket{01},
  \hspace{0.7cm}
  CX \ket{10} = \ket{11},
  \hspace{0.7cm}
  CX \ket{11} = \ket{10}.
\end{align*}
In a nutshell $CX$ flips the state of the second qubit depending on
the state of the first qubit being $\ket{0}$ or $\ket{1}$ -- such a
behaviour extends to all elements of $\complex^2 \otimes \complex^2$
by linearity. Actually, any initial state $\ket{i} \otimes \ket{j}$ in
the usual basis of $\complex^2 \otimes \complex^2$ and the operator
$CX \cdot (H \otimes \id)$ yield an entangled quantum state. The four
states obtained in this way are usually called Bell states, and are
defined as follows:
\begin{align*}
  \frac{1}{\sqrt{2}} \ket{00} + \frac{1}{\sqrt{2}} \ket{11}
  \hspace{0.6cm}
  \frac{1}{\sqrt{2}} \ket{00} - \frac{1}{\sqrt{2}} \ket{11}
  \hspace{0.6cm}
  \frac{1}{\sqrt{2}} \ket{01} + \frac{1}{\sqrt{2}} \ket{10}
  \hspace{0.6cm}
  \frac{1}{\sqrt{2}} \ket{01} - \frac{1}{\sqrt{2}} \ket{10}
\end{align*}


%% Bibliography
\bibliographystyle{alpha}
\bibliography{biblioTeaching}

\end{document}