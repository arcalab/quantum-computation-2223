\documentclass[a4paper, 11pt]{article}

%% packages
\usepackage{fullpage} % changes the margin
\usepackage{hyperref} % Links
\usepackage[utf8]{inputenc}
\usepackage{lmodern}
\usepackage{amsfonts}
\usepackage{amsthm}
\usepackage{amsmath}
\usepackage{braket}
%%

%% macros
\newcommand{\complex}{\mathbb{C}}
\newcommand{\vecs}{\mathcal{V}}
\newcommand{\id}{\mathrm{id}}
%% environments
\theoremstyle{definition}
\newtheorem{definition}{Definition}
\newtheorem{examples}{Example}
\newtheorem{exercises}{Exercises}
\newtheorem{exercise}{Exercise}
\newtheorem{postulate}{Postulate}
%% config
\date{}
\linespread{1.15}
%%

\begin{document}

\title{Quantum Computing @ MEF \\ \large Background}
\author{Ana Neri \\ \scriptsize
\href{mailto:ana.i.neri@inesctec.pt}{ana.i.neri@inesctec.pt}}
\maketitle

\section{Quantum States}

Models of computation often put at center stage a notion of state and
a corresponding notion of state transition~\cite{bruni17}. In the
quantum world, states usually involve superpositions, angles, and
lengths; or in other words, they involve aspects related to
geometry. This suggests us to familiarise with both the notion of a
\emph{vector space} and (the more refined) notion of an \emph{inner
  product space}. It also suggests us to delve deep into the inner
workings of maps between vector spaces and maps between inner product
spaces, both intuitively yielding a notion of a quantum state
transition (i.e. a quantum operation).

\subsection{Vector spaces}

Let $\complex$ denote the set of complex numbers.

\begin{definition}[Vector Space]
  A vector space (over the complex numbers)~\footnote{In this course
    we will only consider vector spaces over the complex numbers. Note
    however that many of the mentioned results hold for a general
    field.} is a set $V$ together with an `addition' operation
  $+ : V + V \to V$, a `multiplication' operation
  $\cdot : \complex \times V \to V$, a `zero' element $0 \in V$, and
  an `inverse' operation $- : V \to V$ such that the following
  equations hold for arbitrary $v,u,w \in V$, $s,r \in \complex$:
  \begin{align*}
    v + (u + w) & = (v + u) + w & v + u & = u + v  \\
    v + 0 & = v & v + (-v) & = 0 \\
    (s r) \cdot v & = s \cdot (r \cdot v) & 1 \cdot v & = v  \\
    s \cdot (v + u) & = s \cdot v + s \cdot u & (s + r) \cdot v & = s \cdot v + r \cdot u
  \end{align*}
\end{definition}

To keep notation simple we will often omit the dot of the scalar
multiplication, i.e. we will write expressions $s \cdot v$ simply as
$s v$.

\begin{examples}
  The complex numbers themselves form a vector space and the set
  $\complex^2$ of pairs of complex numbers also forms a vector
  space. This last space underlies the mathematical representation of
  the state of a qubit. Recall that a qubit is the unit in quantum
  information. Later on we will see that our notion of state
  corresponds exactly to the state of a sequence of qubits.
\end{examples}


\begin{exercise}
  Show that for any finite set $n$ we can build a vector space
  $[n,\complex]$ over the complex numbers. Show also that the set
  $\mathsf{Mat}_\complex(n,m)$ of matrices with $n$ lines and $m$
  columns and whose values are complex numbers also forms a vector
  space (hint: observe that matrices can be given a functional
  representation).
\end{exercise}

\begin{definition}[Linear maps a.k.a. linear operators or simply operators]
  Consider two vector spaces $V$ and $W$. A linear map $f : V \to W$
  is a function that satisfies the equations,
  \begin{align*}
    f(v_1 + v_2) = f (v_1) + f(v_2) \hspace{2cm}
    f (s v) = s  f(v)
  \end{align*}
  We call $f$ a \emph{linear isomorphism} or simply isomorphism if it
  is bijective.
  %% check the correspondence in cats
  When such is the case, we say that $V$ and $W$ are isomorphic to each
  other (i.e. essentially the same), in symbols $V \simeq W$.
\end{definition}

\begin{exercise}
  Show that the identity map $\id : V \to V$ is linear. Additionally show
  that if $f : V \to W$ and $g : W \to U$ are linear maps then their
  composition $g \cdot f : V \to U$ is also a linear map.
\end{exercise}

\begin{exercise}
  \label{ex:states_as_maps}
  Consider a vector space $V$. Show that linear maps
  $f : \complex \to V$ are in one-to-one correspondence with the
  elements of $V$.
\end{exercise}


A crucial concept for our notion of state and state transition is that
of a tensor. In essence, it allows to mathematically represent the
state of a \emph{sequence} of qubits (instead of working with just one
qubit).

\begin{definition}[Tensor]
  Let $V$ and $W$ be two vector spaces. Their tensor, denoted by
  $V \otimes W$, is the vector space consisting of all linear
  combinations $\sum_{i \leq n} s_i  (v_i \otimes w_i)$ with
  $s_i \in \complex$, $v_i \in V$, $w_i \in W$, that satisfies the equations,
  \begin{align*}
   v \otimes w + u \otimes w & = (v + u) \otimes w &
   v \otimes w + v \otimes u & = v \otimes (w + u) \\
   s  (v \otimes w) & = (s v) \otimes w &
   s  (v \otimes w) & = v \otimes (s w)
  \end{align*}
\end{definition}

\begin{exercise}
  Show that from linear maps $f : V \to V'$ and $g : W \to W'$ we can
  define a new linear map
  $f \otimes g : V \otimes W \to V' \otimes W'$. Show that
  $(f' \otimes g') \cdot (f \otimes g)  = (f' \cdot f ) \otimes (g'
  \cdot g)$. Prove the existence of linear isomorphisms
  $V \otimes W \simeq W \otimes V$ and $V \otimes \complex \simeq V$.
\end{exercise}

\begin{exercise}
  Show that the map
  $\Delta : \complex^2 \to \complex^2 \otimes \complex^2$ defined by
  $\Delta(v) = v \otimes v$ is \emph{non-linear}. What is the relation
  between this map and the no-cloning theorem?
\end{exercise}

Another concept that will be imensely useful in the course is that of
a basis.

\begin{definition}[Basis]
  A basis for a vector space $V$ is a set $B \subseteq V$ of vectors that
  respects the following conditions:
  \begin{itemize}
  \item for every $v \in V$, we can find $v_1,\dots,v_n \in B$ and
    $s_1,\dots,s_n \in \complex$ such that
    $\sum_{i \leq n} s_i v_i = v$
  \item for every sequence of vectors $v_1,\dots,v_n \in B$ and
    sequence of complex numbers $s_1,\dots,s_n \in \complex$ if
    $\sum_{i \leq n} s_i  v_i = 0$ then $s_i = 0$ for all
    $i \leq n$.
  \end{itemize}
\end{definition}

\begin{examples}
  The set $\{ 1 \}$ is a basis for $\complex$  and the set
  $\{(1,0),(0,1)\}$ is a basis for $\complex^2$.
\end{examples}

Let $B$ be a basis for a vector space $V$. If $B$ has $n$ elements we
say that $V$ is $n$-dimensional. If $B$ is finite we say that $V$ is
\emph{finite-dimensional}.

In this course we are primarily interested in finite-dimensional
vector spaces. Intuitively, this is justified by the fact we will only
need to work with a finite number of qubits at a time. Thus from now
on all vector spaces that we consider are finite-dimensional.


\begin{exercise}
  Let $n$ be a natural number and $\complex^n$ be the vector space of
  $n$-tuples of complex numbers. Present a basis for $\complex^n$ and
  subsequently indicate its dimension. Next let
  $\mathsf{Mat}_\complex(n,m)$ be the vector space of matrices with
  $n$ lines and $m$ columns and whose values are complex
  numbers. Present a basis for this space and subsequently indicate
  its dimension.
\end{exercise}


\begin{exercise}
  Consider a linear map $f : V \to W$ and let $B$ be a basis for
  $V$. Show that this map is \emph{uniquely determined} by the way it
  maps the elements in $B$. Moreover, show that a function $B \to W$
  mapping elements in the basis of $V$ to $W$ induces a linear map of
  type $V \to W$.
\end{exercise}


\begin{exercise}
  \label{ex:skele}
  Show that any vector space $V$ with dimension $n$ is isomorphic to
  the vector space $\complex^n$.
\end{exercise}

Matrices provide a very convenient way of representing states and also
of representing state transitions. Let us analyse how such a
representation works.  Let $V$ and $W$ be vector spaces,
$\{b_1,\dots,b_n\}$ a basis for $V$ and $\{c_1,\dots,c_m\}$ a basis
for $W$. Consider then a linear map $f : V \to W$ and observe that for
every $i \leq n$ we have $f(b_i) = \sum_{j \leq m} s_{ij} c_j$ for
some $s_{i1}, \dots, s_{im} \in \complex$. We obtain a matrix
representation $M \in \mathsf{Mat}_\complex(m,n)$ of $f$ by setting
$M_{ji} = s_{ij}$.  Conversely, consider a matrix
$M \in \mathsf{Mat}_\complex(m,n)$. It induces a linear map
$f : V \to W$ by setting $f(b_i) = \sum_{j \leq m} M_{ji} c_j$.

\begin{exercise}
  Show that the two operations described above (for switching between
  linear maps and their matrix representation) are inverse of each
  other.
\end{exercise}


\begin{exercise}
  What is the matrix representation of the linear map
  $f : \complex^2 \to \complex^2$ defined by $f(1,0) = (0,1)$ and
  $f(0,1) = (1,0)$? What is the matrix representation of the linear
  map $f : \complex^2 \to \complex^2$ defined by
  $f(1,0) = \frac{1}{\sqrt{2}}(1,0) + \frac{1}{\sqrt{2}}(0,1)$ and
  $f(0,1) = \frac{1}{\sqrt{2}}(1,0) - \frac{1}{\sqrt{2}}(0,1)$?
\end{exercise}

Before moving forward in the course, we need to fix extra
notation. Specifically, we will use $M : n \to m$ to denote a matrix
$M$ with $n$ lines, $m$ columns, and whose values are complex
numbers. Also for two matrices $M : n \to m$ and $N : m \to o$, we
will use $M N : n \to o$ to denote the matrix multiplication of $M$
with $N$. Finally, given a linear map $f : V \to W$ such that $V$ and
$W$ have dimension $n$ and $m$, respectively, we will use
$M_f : m \to n$ to denote the corresponding matrix.
  
\begin{exercise}[H]
  Show that elements of $ V$ are in one-to-one correspondence with
  elements of $\mathsf{Mat}_\complex(n,1)$. For all linear maps
  $f : V \to W$ and $g : W \to U$ show that
  $M_g M_f = M_{g \cdot f}$.
\end{exercise}


\begin{exercise}[H]
  Let $B \subseteq V$, $C \subseteq W$ be bases for vector spaces $V$
  and $W$, respectively.  Show that the set
  $\{ b \otimes c \mid b \in B, c \in C \}$ is a basis for
  $V \otimes W$. Then show that
  $\complex^n \otimes \complex^m \simeq \complex^{n m}$
  (hint: recall Exercise~\ref{ex:skele}).
\end{exercise}

Consider matrices $M : n \to m$ and $N : o \to p$. Their tensor
$M \otimes N : n \cdot o \to m \cdot p$ (also called Kronecker
product) is defined by,
\begin{align*}
  M \otimes N =
  \begin{bmatrix}
    M_{1,1} \cdot N, & \dots , &  M_{1,m} \cdot N \\
    \vdots & \vdots & \vdots \\
    M_{n,1} \cdot N, & \dots , &  M_{n,m} \cdot N
  \end{bmatrix}
\end{align*}

\begin{exercise}[H]
  For all linear maps $f : V \to V'$ and $g : W \to W'$ show that
  $M_{f \otimes g} = M_f \otimes M_g$.
\end{exercise}


\begin{exercise}
  For a given matrix $M : n \to m$, we will use $M^\ast : n \to m$ to
  denote the matrix such that $M_{ij}^\ast = (M_{ij})^\ast$,
  $M^T : m \to n$ to denote the transpose of $M$, and
  $M^\dagger : m \to n$ to denote the matrix $(M^T)^\ast$, i.e. the
  conjugate transpose (a.k.a. adjoint) of $M$. Show that the following
  equations hold.
  \begin{align*}
    (M \otimes N)^\ast = M^\ast \otimes N^\ast \hspace{1cm}
    (M \otimes N)^T = M^T \otimes N^T \hspace{1cm}
    (M \otimes N)^\dagger = M^\dagger \otimes N^\dagger
  \end{align*}
\end{exercise}


\subsection{Inner product spaces}

Recall that for some complex number $c$ the expression $c^\ast$
denotes the \emph{complex conjugate} of $c$.

\begin{definition}[Inner product space] An inner product space is a
  vector space $V$ equipped with a function
  $\langle \cdot,\cdot \rangle : V \times V \to \complex$ (the inner
  product) that satisfies the conditions,
  \begin{align*}
    \left \langle v, \sum_{i \leq n} s_i v_i \right \rangle
    & = \sum_{i \leq n} s_i \cdot \langle v,v_i \rangle
    & \langle v,w \rangle  & = \langle w,v \rangle^\ast \\
    \langle v,v \rangle & \geq 0 & 
    \langle v,v \rangle & = 0 \text{ entails } v = 0
  \end{align*}
  for all $v, v_i, w \in V$ and $s_i \in \complex$.  \footnote{Since
    we assume that all vector spaces at hand are finite-dimensional we
    can see inner product spaces as Hilbert spaces.}
\end{definition}

\begin{exercise}
  Let $n$ be a natural number.  Show that the vector space
  $\complex^n$ becomes an inner product space when equipped with the
  function $\langle \cdot, \cdot \rangle
  : \complex^n \times \complex^n \to \complex$
  defined by,
  \begin{align*}
   \langle (a_1,\dots,a_n),(b_1,\dots,b_n) \rangle = \sum_{i \leq n} a_i^\ast b_i 
  \end{align*}
\end{exercise}

Recall that a norm over a vector space $V$ provides a notion of length
to the vector space and is formally defined as a function
$\| \cdot \| : V \to [0,\infty)$ such that the following conditions
are satisfied,
\begin{flalign*}
  \| v \| = 0 \text{ iff } v = 0 \hspace{1.5cm}
  \| s \cdot v \| = | s | \cdot \| v \| \hspace{1.5cm}
  \| v + w \| \leq \| v \| + \| w \|
\end{flalign*}
for all $v,w \in V$, $s \in \complex$.  Moreover, every inner product
space $V$ induces a norm $\| \cdot \| : V \to [0,\infty)$ defined by
$\| v \| = \sqrt{\langle v,v \rangle}$.

As we will see, the mathematical representation of the state of
$n$-qubits is a vector $v \in \complex^{2^n}$ with norm $\| v \| = 1$.

\begin{exercise}[Vector normalisation]
  Let $v \in V$ be a vector. Show that,
  \begin{align*}
   \left \| \frac{v}{\| v \|} \right \| = 1 
  \end{align*}

\end{exercise}

\begin{definition}[Orthonormal basis]
  Two vectors $v, w \in V$ are said to be orthogonal to each other if
  $ \langle v,w \rangle = 0$.  A basis $B$ for an inner product space
  $V$ is called orthonormal if all elements of $B$ have norm $1$ and
  all elements $v \not = w \in B$ are orthogonal to each other.
\end{definition}

\begin{exercise}
  Show that the basis $\{ (1,0), (0,1) \}$ for $\complex^2$ is
  orthonormal.
\end{exercise}

\begin{definition}[Tensor]
  Let $V$ and $W$ be two inner spaces. Their tensor, denoted by
  $V \otimes W$, is the tensor of $V$ and $W$ as vector spaces
  equipped with the function,
  \begin{align*}
    \left \langle  \sum_{i \leq n} s_i (v_i \otimes w_i), \sum_{j \leq m} r_j (v_j \otimes w_j)
    \right \rangle = \sum_{i \leq n, j \leq m} s_i^\ast r_j \cdot \langle v_i,v_j \rangle
    \cdot \langle w_i,w_j \rangle
  \end{align*}
\end{definition}

\begin{exercise}
  Let $B \subseteq V$, $C \subseteq W$ be orthonormal bases for inner
  product spaces $V$ and $W$, respectively.  Show that the set
  $\{ b \otimes c \mid b \in B, c \in C \}$ is an orthonormal basis
  for $V \otimes W$.
\end{exercise}

When working with linear maps $f : V \to W$ between inner product
spaces $V$ and $W$ we are often interested in those maps that are
isometric.
\begin{definition}[Isometry]
  Consider inner product spaces $V$ and $W$ and a linear map
  $f : V \to W$ between them. We call  $f$ an isometry if
  the equation,
  \begin{flalign*}
    \langle v_1, v_2 \rangle = \langle f(v_1), f(v_2) \rangle
  \end{flalign*}
  holds for all $v_1,v_2 \in V$. Equivalently, $f$ is an isometry
  iff $\| v \| = \| f (v) \|$ for all $v \in V$.
\end{definition}
A key property of isometries is they always send unit vectors to unit
vectors (because isometries preserve norms). In the particular case of
$V = W = \complex^{2^n}$, this means that quantum states are always
mapped to quantum states (and not to something else).

Additionally, quantum physics postulates that quantum operations on an
isolated system must be \emph{reversible}. In other words, maps
$f : V \to W$ representing pure quantum operations must have an
\emph{inverse} $f^{-1} : W \to V$ which satisfies
$f^{-1} \cdot f = f \cdot f^{-1} = \id$. Together with the notion of
an isometry, this condition gives rise to the notion of a unitary map.

\begin{definition}[Unitary maps]
  Let $V$ and $W$ be inner product spaces. A linear map $f : V \to W$
  is called unitary if $f$ is an isometry and surjective\footnote{Both
  conditions entail that $f$ has an inverse $f^{-1}$.}.
\end{definition}

\begin{postulate}[Quantum state and state transition]
  The state of an \emph{isolated} quantum computer is given by a unit
  vector in the space $\complex^{2^n}$ for some finite number $n$ --
  the number $n$ corresponds to the number of available qubits. State
  transitions arise via unitary maps, more concretely the state of an
  isolated quantum computer changes by an application of a unitary
  map.~\footnote{See a more general version of this postulate in
    Section 2.2 of~\cite{nielsen02}}
\end{postulate}

The notion of a unitary map can also be formulated via matrices, and
often this alternative formulation is easier to work with: let us
consider a linear map $f : V \to V$ and its matrix representation
$M_f : n \to n$. Then $f$ is unitary iff
$M_f^\dagger M_f = M_f M_f^\dagger = I$.
\begin{exercise}
  Show that the following two maps are unitary:
  \begin{itemize}
  \item $f : \complex^2 \to \complex^2$ defined by $f(1,0) = (0,1)$ and
    $f(0,1) = (1,0)$.
  \item $g : \complex^2 \to \complex^2$ defined by
  $g(1,0) = \frac{1}{\sqrt{2}}(1,0) + \frac{1}{\sqrt{2}}(0,1)$ and
  $g(0,1) = \frac{1}{\sqrt{2}}(1,0) - \frac{1}{\sqrt{2}}(1,0)$.
  \end{itemize}
\end{exercise}

\begin{exercise}
  Prove that if two linear maps are unitary then their tensor is also
  unitary.
\end{exercise}

Consider a linear map $f : V \to W$ between inner product spaces $V$
and $W$. There exists a unique linear map $f^\dagger : W \to V$ such
that for all $v \in V$ and $w \in W$ the equation,
\begin{align*}
  \langle f (v), w \rangle = \langle v, f^\dagger (w) \rangle
\end{align*}
holds. This map is precisely the functional representation of
$M_f^\dagger$.

%% Bibliography
\bibliographystyle{alpha}
\bibliography{biblioTeaching}

\end{document}